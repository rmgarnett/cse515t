\documentclass{article}

\usepackage[T1]{fontenc}
\usepackage[osf]{libertine}
\usepackage[scaled=0.8]{beramono}
\usepackage[margin=1.5in]{geometry}
\usepackage{url}
\usepackage{booktabs}
\usepackage{amsmath}
\usepackage{amssymb}
\usepackage{nicefrac}
\usepackage{microtype}
\usepackage{bm}

\usepackage{sectsty}
\sectionfont{\large}
\subsectionfont{\normalsize}

\usepackage{titlesec}
\titlespacing{\section}{0pt}{10pt plus 2pt minus 2pt}{0pt plus 2pt minus 0pt}
\titlespacing{\subsection}{0pt}{5pt plus 2pt minus 2pt}{0pt plus 2pt minus 0pt}

\usepackage{pgfplots}
\pgfplotsset{
  compat=newest,
  plot coordinates/math parser=false,
  tick label style={font=\footnotesize, /pgf/number format/fixed},
  label style={font=\small},
  legend style={font=\small},
  every axis/.append style={
    tick align=outside,
    clip mode=individual,
    scaled ticks=false,
    thick,
    tick style={semithick, black}
  }
}

\pgfkeys{/pgf/number format/.cd, set thousands separator={\,}}

\usepgfplotslibrary{external}
\tikzexternalize[prefix=tikz/]

\newlength\figurewidth
\newlength\figureheight

\setlength{\figurewidth}{8cm}
\setlength{\figureheight}{6cm}

\setlength{\parindent}{0pt}
\setlength{\parskip}{1ex}

\newcommand{\acro}[1]{\textsc{\MakeLowercase{#1}}}
\newcommand{\given}{\mid}
\newcommand{\mc}[1]{\mathcal{#1}}
\newcommand{\data}{\mc{D}}
\newcommand{\intd}[1]{\,\mathrm{d}{#1}}

\begin{document}

{\large \textbf{CSE 515T (Spring 2015) Assignment 1 Solutions}} \\

\begin{enumerate}
\item
  (Barber.)
  Suppose that a study shows that 90\% of people who have contracted
  Creutzfeldt--Jakob disease (``mad cow disease'') ate hamburgers
  prior to contracting the disease.  Creutzfeldt--Jakob disease is
  incredibly rare; suppose only one in a million people have the
  disease.

  If you eat hamburgers, should you be worried?  Does this depend on
  how many other people eat hamburgers?
\end{enumerate}

\subsection*{Solution}
Let CJ be the random variaable ``has Creutzfeldt--Jakob disease,'' and
let H be the random variable ``eats hamburgers.''  From the problem,
we know $\Pr(\mathrm{H} \given \mathrm{CJ}) = 0.9$ and
$\Pr(\mathrm{CJ}) = 10^{-6}$.  From Bayes' theorem, we may compute the
posterior probability of having Creutzfeldt--Jakob disease given that
you eat hamburgers:
\begin{equation*}
  \Pr(\mathrm{CJ} \given \mathrm{H})
  =
  \frac
      {\Pr(\mathrm{H} \given \mathrm{CJ}) \Pr(\mathrm{CJ})}
      {\Pr(\mathrm{H})}.
\end{equation*}
The denominator, $\Pr(\mathrm{H})$, is the probability that a random
person eats hamburgers, so whether you should be worried does depend
on this value.  I estimate $\Pr(\mathrm{H})$ might be around
\nicefrac{1}{2}; plugging in this value, the posterior probability
of having Creutzfeldt--Jakob disease is
\begin{equation*}
  \frac{0.9 \times 10^{-6}}{\nicefrac{1}{2}}
  =
  1.8 \times 10^{-6},
\end{equation*}
so your probability of having the disease has only increased from
$10^{-6}$ to $1.8 \times 10^{-6} = 0.0000018$ as a result of eating
hamburgers.  You can sleep safe.

If hamburger eating were rare, say $\Pr(\mathrm{H}) = 10^{-5}$, then
we would instead have $\Pr(\mathrm{CJ} \given \mathrm{H}) = 9\%$, and
maybe you should be worried!

\clearpage

\begin{enumerate}
\setcounter{enumi}{1}
\item
  (O'Hagan and Forster.)
  Suppose $x$ has a Poisson distribution with unknown mean
  $\theta$:
  \begin{equation*}
    p(x \given \theta) = \frac{\theta^x}{x!} \exp(-\theta),
    \qquad
    x = 0, 1, \dotsc
  \end{equation*}
  Let the prior for $\theta$ be a gamma distribution:
  \begin{equation*}
    p(\theta \given \alpha, \beta)
    =
    \frac{\beta^\alpha \theta^{\alpha - 1}}{\Gamma(\alpha)} \exp(-\beta\theta),
    \qquad \theta > 0
  \end{equation*}
  where $\Gamma$ is the gamma function.  Show that, given an
  observation $x$, the posterior $p(\theta \given x, \alpha, \beta)$
  is a gamma distribution with updated parameters $(\alpha', \beta') =
  (\alpha + x, \beta + 1)$.
\end{enumerate}

\subsection*{Solution}
From Bayes' theorem, we have:
\begin{align*}
  p(\theta \given x)
  &\propto
  p(x \given \theta)
  p(\theta)
  \\
  &\propto
  \bigl(\theta^{x} \exp(-\theta)\bigr)
  \bigl(\theta^{\alpha - 1} \exp(-\beta\theta)\bigr)
  \\
  &=
  \theta^{x + \alpha - 1}
  \exp\bigl(-(\beta + 1)\theta\bigr)
  \\
  &\propto
  \mc{G}(\alpha + x, \beta + 1).
\end{align*}

Here we exploit a common trick: we manipulate the numerator, ignoring
constants independent of $\theta$.  If we can recognize the functional
form as belonging to a distribution family we know, we can simply
identify the parameters and trust that the distribution normalizes!

\clearpage

\begin{enumerate}
\setcounter{enumi}{2}
\item
  (Optimal \emph{Price is Right} bidding.)
  Suppose you have a standard normal belief about an unknown parameter
  $\theta$, $p(\theta) = \mc{N}(\theta; 0, 1^2)$.  You are asked to
  give a point estimate $\hat{\theta}$ of $\theta$, but are told that
  there is a heavy penalty for guessing too high.  The loss function is
  \begin{equation*}
    \ell(\hat{\theta}, \theta; c)
    =
    \begin{cases}
      (\theta - \hat{\theta})^2 & \hat{\theta}  <   \theta; \\
      c                         & \hat{\theta} \geq \theta
    \end{cases},
  \end{equation*}
  where $c > 0$ is a constant cost for overestimating.  What is the
  Bayesian estimator in this case?  How does it change as a function
  of $c$?
\end{enumerate}

\subsection*{Solution}
Let $\phi(x) = \mc{N}(x; 0, 1^2)$ be the standard normal \acro{PDF}
evaluated at $x$, and let $\Phi(x) = \int_{-\infty}^x \phi(x)
\intd{x}$ be the standard normal \acro{CDF} evaluated at $x$.  If we
fix a point $\hat{\theta}$, the expected loss is:
\begin{align*}
  \mathbb{E}\bigl[\ell(\hat{\theta}, \theta; c)\bigr]
  &=
  \int
  \ell(\hat{\theta}, \theta; c)
  p(\theta)
  \intd{\theta}
  \\
  &=
  \int_{-\infty}^{\hat{\theta}}
  c
  \phi(\theta)
  \intd{\theta}
  +
  \int_{\hat{\theta}}^{\infty}
  (\theta - \hat{\theta})^2
  \phi(\theta)
  \intd{\theta}.
\end{align*}
The first term is proportional to the standard normal \acro{CDF}
$\Phi$ evaluated at $\hat{\theta}$:
\begin{equation*}
  \int_{-\infty}^{\hat{\theta}}
  c
  \phi(\theta)
  \intd{\theta}
  =
  c\,\Phi(\hat{\theta}).
\end{equation*}

We may also compute the second integral using the following
antiderivatives:\footnote{\url{http://en.wikipedia.org/wiki/List_of_integrals_of_Gaussian_functions}
  is useful here! (Wolfram alpha also works.)}
\begin{equation*}
  \int \theta \phi(\theta) \intd{\theta} = -\phi(\theta) + C;
  \qquad
  \int \theta^2 \phi(\theta) \intd{\theta} = \Phi(\theta) - \theta \phi(\theta) + C.
\end{equation*}
Using the fundamental theorem of calculus, we may use these to calculate
\begin{equation*}
  \int_{\hat{\theta}}^{\infty}
  (\theta - \hat{\theta})^2
  \phi(\theta)
  \intd{\theta}
  =
  (\hat{\theta}^2 + 1) (1 - \Phi(\hat{\theta}))
  - \hat{\theta}\phi(\hat{\theta}).
\end{equation*}
Finally, the entire expected loss is
\begin{equation*}
  \mathbb{E}\bigl[\ell(\hat{\theta}, \theta; c)\bigr]
  =
  (c - \hat{\theta}^2 - 1)\Phi(\hat{\theta})
  - \hat{\theta}\phi(\hat{\theta})
  + \hat{\theta}^2 + 1.
\end{equation*}
The derivative with respect to $\hat{\theta}$ is
\begin{equation*}
  \frac{\partial \mathbb{E}\bigl[\ell(\hat{\theta}, \theta; c)\bigr]}
       {\partial \hat{\theta}}
  =
  2 \hat{\theta} \bigl(1 - \Phi(\hat{\theta})\bigr)
  + (c - 2) \phi(\hat{\theta}).
\end{equation*}
Unfortunately, I do not believe we may find a root explicitly, so we
would have to rely on numerical root finding.  For $c = 1$, the
minimial expected loss is achieved at $\hat{\theta} = 0.612$; for $c =
10$, the Bayes action is $\hat{\theta} = -1.0615$; for $c = 100$, the
Bayes action is $\hat{\theta} = -2.1167$.  In general, the larger $c$,
the smaller your estimate should be, due to the potentially high cost
of overestimation.  For $c = 0$, there is no Bayes action, because we
may continue to decrease the expected loss when taking $\hat{\theta}
\to \infty$ (there is no reason not to!).

\clearpage

\begin{enumerate}
\setcounter{enumi}{3}
\item
  (Maximum-likelihood estimation.)
  Suppose you flip a coin with unknown bias $\theta$, $\Pr(x =
  \text{H}) = \theta$, three times and observe the outcome HHH.  What
  is the maximum likelihood estimator for $\theta$?  Do you think this
  is a good estimator?  Would you want to use it to make predictions?

  Consider a Bayesian analysis of $\theta$ with a beta prior $p(\theta
  \given \alpha, \beta) = \mc{B}(\theta; \alpha, \beta)$.  What is the
  posterior mean of $\theta$?  What is the posterior mode?  Consider
  $(\alpha, \beta) = (\nicefrac{1}{5}, \nicefrac{1}{5})$.  Plot the
  posterior density in this case.  Is the posterior mean a good
  summary of the distribution?
\end{enumerate}

\subsection*{Solution}
The likelihood of HHH is proportional to $\theta^3$, which over the
domain $\theta \in [0, 1]$ is maximized at $\theta = 1$.  Whether you
think this is a good estimator is subjective; however, I certainly
wouldn't use it for prediction, because it completely discounts the
(in my opinion, still rather likely) possiblity of a tails event.

With a $\mc{B}(\alpha, \beta)$ prior, the posterior is $\mc{B}(\alpha
+ 3, \beta)$.  We may calculate the mean of an arbitrary beta
distribution:
\begin{equation*}
  \mathbb{E}[\theta \given \alpha, \beta]
  =
  \int
  \theta
  \mc{B}(\theta \given \alpha, \beta)
  \intd{\theta}
  =
  \frac{1}{B(\alpha, \beta)}
  \int_{0}^{1}
  \theta
  \bigl(
  \theta^{\alpha - 1}
  (1 - \theta)^{\beta - 1}
  \bigr)
  \intd{\theta}
  =
  \frac{B(\alpha + 1, \beta)}{B(\alpha, \beta)}
  =
  \frac{\alpha}{\alpha + \beta}.
\end{equation*}

Therefore the posterior mean is $\frac{\alpha + 3}{\alpha + \beta +
  3}$.  The posterior mode is $\frac{\alpha + 2}{\alpha + \beta +
  1}$.\footnote{\url{http://goo.gl/zigQER} -- this is how I would
  answer this type of question!} Note that the posterior mode only
exists for $\alpha, \beta > 1$.

The prior and posterior distributions are plotted below.  An
interesting point about the prior is that its mean and median are both
$\nicefrac{1}{2}$, but this is simultaneously the anti-mode!  It's an
unusual estimator.

\begin{figure}[h]
  \centering
  % This file was created by matlab2tikz.
% Minimal pgfplots version: 1.3
%
\tikzsetnextfilename{problem_4}
\definecolor{mycolor1}{rgb}{0.12157,0.47059,0.70588}%
%
\begin{tikzpicture}

\begin{axis}[%
width=0.95092\figurewidth,
height=\figureheight,
at={(0\figurewidth,0\figureheight)},
scale only axis,
xmin=0,
xmax=10,
xlabel={$\sigma$},
ymin=0.5,
ymax=1.05,
ylabel={$\Pr(w_2 > 0 \given \data, \sigma^2)$},
axis x line*=bottom,
axis y line*=left
]
\addplot [color=mycolor1,solid,forget plot]
  table[row sep=crcr]{%
0.01	1\\
0.02	0.999999999992479\\
0.03	0.999996595966551\\
0.04	0.999632318134581\\
0.05	0.996556025677947\\
0.06	0.987878259818996\\
0.07	0.973368167687833\\
0.08	0.954735487167244\\
0.09	0.933964947476636\\
0.1	0.912583952534793\\
0.11	0.891574753363322\\
0.12	0.871496490590082\\
0.13	0.852626389646931\\
0.14	0.835068176630924\\
0.15	0.818824723732505\\
0.16	0.80384371699183\\
0.17	0.790045318348458\\
0.18	0.777338451294903\\
0.19	0.765630100748129\\
0.2	0.754830397766391\\
0.21	0.744855196659661\\
0.22	0.735627183930615\\
0.23	0.727076147021121\\
0.24	0.719138779835368\\
0.25	0.711758249529762\\
0.26	0.704883656624097\\
0.27	0.698469464559199\\
0.28	0.692474941101449\\
0.29	0.686863633782152\\
0.3	0.68160288958312\\
0.31	0.676663422118278\\
0.32	0.672018925643197\\
0.33	0.667645733148085\\
0.34	0.663522514809105\\
0.35	0.659630012737433\\
0.36	0.65595080799485\\
0.37	0.652469116070091\\
0.38	0.64917060733047\\
0.39	0.646042249317254\\
0.4	0.643072168107797\\
0.41	0.640249526302906\\
0.42	0.637564415504967\\
0.43	0.635007761429093\\
0.44	0.632571240033259\\
0.45	0.630247203268365\\
0.46	0.628028613235669\\
0.47	0.625908983700946\\
0.48	0.623882328055273\\
0.49	0.621943112932756\\
0.5	0.620086216800299\\
0.51	0.618306892924071\\
0.52	0.616600736194957\\
0.53	0.61496365336201\\
0.54	0.613391836280597\\
0.55	0.611881737831659\\
0.56	0.610430050211411\\
0.57	0.609033685328275\\
0.58	0.607689757075817\\
0.59	0.606395565278646\\
0.6	0.605148581132367\\
0.61	0.603946433980035\\
0.62	0.602786899285812\\
0.63	0.601667887682682\\
0.64	0.600587434985228\\
0.65	0.599543693070632\\
0.66	0.598534921541953\\
0.67	0.597559480097205\\
0.68	0.596615821536055\\
0.69	0.595702485343312\\
0.7	0.594818091794892\\
0.71	0.59396133653757\\
0.72	0.593130985598999\\
0.73	0.592325870788863\\
0.74	0.591544885456041\\
0.75	0.590786980570195\\
0.76	0.590051161099329\\
0.77	0.589336482657622\\
0.78	0.588642048400429\\
0.79	0.587967006145426\\
0.8	0.587310545701021\\
0.81	0.586671896384804\\
0.82	0.586050324716486\\
0.83	0.585445132271193\\
0.84	0.584855653680229\\
0.85	0.584281254767648\\
0.86	0.583721330811953\\
0.87	0.583175304923218\\
0.88	0.582642626526779\\
0.89	0.582122769945367\\
0.9	0.581615233072313\\
0.91	0.581119536129002\\
0.92	0.580635220500419\\
0.93	0.580161847643043\\
0.94	0.579698998059897\\
0.95	0.579246270337938\\
0.96	0.578803280243383\\
0.97	0.578369659870906\\
0.98	0.577945056842964\\
0.99	0.577529133555825\\
1	0.577121566469098\\
1.01	0.576722045435849\\
1.02	0.576330273070582\\
1.03	0.575945964152599\\
1.04	0.575568845062403\\
1.05	0.575198653249009\\
1.06	0.574835136726189\\
1.07	0.574478053595776\\
1.08	0.574127171596369\\
1.09	0.573782267675797\\
1.1	0.573443127585926\\
1.11	0.573109545498387\\
1.12	0.572781323640002\\
1.13	0.572458271946675\\
1.14	0.572140207734693\\
1.15	0.571826955388362\\
1.16	0.571518346063062\\
1.17	0.571214217402792\\
1.18	0.570914413271393\\
1.19	0.570618783496663\\
1.2	0.570327183626627\\
1.21	0.570039474697299\\
1.22	0.569755523011273\\
1.23	0.569475199926574\\
1.24	0.569198381655192\\
1.25	0.568924949070776\\
1.26	0.568654787525009\\
1.27	0.568387786672189\\
1.28	0.568123840301593\\
1.29	0.567862846177215\\
1.3	0.567604705884497\\
1.31	0.56734932468369\\
1.32	0.567096611369519\\
1.33	0.566846478136817\\
1.34	0.566598840451852\\
1.35	0.566353616929042\\
1.36	0.566110729212808\\
1.37	0.565870101864315\\
1.38	0.565631662252851\\
1.39	0.565395340451642\\
1.4	0.565161069137879\\
1.41	0.564928783496756\\
1.42	0.564698421129346\\
1.43	0.564469921964122\\
1.44	0.564243228171969\\
1.45	0.56401828408452\\
1.46	0.563795036115661\\
1.47	0.563573432686076\\
1.48	0.563353424150694\\
1.49	0.563134962728894\\
1.5	0.562918002437374\\
1.51	0.562702499025545\\
1.52	0.562488409913356\\
1.53	0.562275694131433\\
1.54	0.562064312263457\\
1.55	0.561854226390653\\
1.56	0.561645400038337\\
1.57	0.561437798124408\\
1.58	0.561231386909722\\
1.59	0.561026133950265\\
1.6	0.560822008051053\\
1.61	0.560618979221686\\
1.62	0.560417018633497\\
1.63	0.560216098578229\\
1.64	0.560016192428176\\
1.65	0.559817274597741\\
1.66	0.55961932050634\\
1.67	0.559422306542622\\
1.68	0.559226210029928\\
1.69	0.55903100919297\\
1.7	0.558836683125661\\
1.71	0.558643211760061\\
1.72	0.558450575836403\\
1.73	0.558258756874152\\
1.74	0.558067737144055\\
1.75	0.557877499641162\\
1.76	0.557688028058759\\
1.77	0.557499306763203\\
1.78	0.55731132076961\\
1.79	0.557124055718372\\
1.8	0.556937497852475\\
1.81	0.556751633995586\\
1.82	0.556566451530882\\
1.83	0.556381938380596\\
1.84	0.556198082986257\\
1.85	0.556014874289595\\
1.86	0.55583230171409\\
1.87	0.555650355147147\\
1.88	0.555469024922868\\
1.89	0.555288301805403\\
1.9	0.555108176972869\\
1.91	0.554928642001797\\
1.92	0.554749688852118\\
1.93	0.55457130985264\\
1.94	0.554393497687021\\
1.95	0.554216245380214\\
1.96	0.554039546285362\\
1.97	0.553863394071142\\
1.98	0.553687782709528\\
1.99	0.553512706463973\\
2	0.553338159877984\\
2.01	0.553164137764081\\
2.02	0.55299063519314\\
2.03	0.55281764748408\\
2.04	0.552645170193913\\
2.05	0.552473199108126\\
2.06	0.552301730231381\\
2.07	0.552130759778549\\
2.08	0.551960284166034\\
2.09	0.551790300003396\\
2.1	0.551620804085267\\
2.11	0.551451793383532\\
2.12	0.551283265039792\\
2.13	0.551115216358071\\
2.14	0.55094764479779\\
2.15	0.550780547966965\\
2.16	0.550613923615661\\
2.17	0.550447769629657\\
2.18	0.550282084024337\\
2.19	0.550116864938798\\
2.2	0.549952110630157\\
2.21	0.549787819468065\\
2.22	0.549623989929406\\
2.23	0.549460620593194\\
2.24	0.549297710135641\\
2.25	0.549135257325404\\
2.26	0.548973261019006\\
2.27	0.548811720156414\\
2.28	0.548650633756784\\
2.29	0.548490000914351\\
2.3	0.548329820794474\\
2.31	0.548170092629829\\
2.32	0.54801081571673\\
2.33	0.547851989411596\\
2.34	0.547693613127545\\
2.35	0.547535686331113\\
2.36	0.547378208539098\\
2.37	0.547221179315521\\
2.38	0.547064598268706\\
2.39	0.546908465048462\\
2.4	0.546752779343382\\
2.41	0.546597540878244\\
2.42	0.546442749411507\\
2.43	0.546288404732911\\
2.44	0.546134506661167\\
2.45	0.545981055041742\\
2.46	0.545828049744729\\
2.47	0.545675490662803\\
2.48	0.545523377709261\\
2.49	0.54537171081614\\
2.5	0.545220489932412\\
2.51	0.545069715022257\\
2.52	0.544919386063405\\
2.53	0.544769503045548\\
2.54	0.54462006596882\\
2.55	0.544471074842343\\
2.56	0.544322529682835\\
2.57	0.54417443051328\\
2.58	0.544026777361657\\
2.59	0.543879570259722\\
2.6	0.543732809241851\\
2.61	0.543586494343933\\
2.62	0.543440625602314\\
2.63	0.543295203052792\\
2.64	0.543150226729657\\
2.65	0.543005696664783\\
2.66	0.542861612886757\\
2.67	0.542717975420059\\
2.68	0.542574784284275\\
2.69	0.542432039493359\\
2.7	0.542289741054929\\
2.71	0.542147888969598\\
2.72	0.542006483230347\\
2.73	0.541865523821928\\
2.74	0.541725010720303\\
2.75	0.541584943892115\\
2.76	0.541445323294188\\
2.77	0.541306148873063\\
2.78	0.541167420564551\\
2.79	0.541029138293331\\
2.8	0.540891301972559\\
2.81	0.540753911503511\\
2.82	0.540616966775249\\
2.83	0.540480467664313\\
2.84	0.540344414034434\\
2.85	0.540208805736264\\
2.86	0.540073642607141\\
2.87	0.539938924470863\\
2.88	0.539804651137481\\
2.89	0.539670822403121\\
2.9	0.539537438049816\\
2.91	0.539404497845355\\
2.92	0.539272001543158\\
2.93	0.539139948882152\\
2.94	0.539008339586678\\
2.95	0.538877173366404\\
2.96	0.538746449916252\\
2.97	0.538616168916343\\
2.98	0.538486330031952\\
2.99	0.538356932913477\\
3	0.538227977196419\\
3.01	0.538099462501371\\
3.02	0.537971388434027\\
3.03	0.537843754585188\\
3.04	0.537716560530789\\
3.05	0.537589805831933\\
3.06	0.537463490034927\\
3.07	0.53733761267134\\
3.08	0.537212173258054\\
3.09	0.537087171297339\\
3.1	0.536962606276918\\
3.11	0.536838477670056\\
3.12	0.536714784935642\\
3.13	0.536591527518288\\
3.14	0.536468704848426\\
3.15	0.536346316342416\\
3.16	0.536224361402658\\
3.17	0.536102839417708\\
3.18	0.535981749762399\\
3.19	0.535861091797969\\
3.2	0.53574086487219\\
3.21	0.535621068319505\\
3.22	0.535501701461165\\
3.23	0.535382763605369\\
3.24	0.535264254047414\\
3.25	0.535146172069841\\
3.26	0.535028516942585\\
3.27	0.534911287923132\\
3.28	0.534794484256676\\
3.29	0.534678105176274\\
3.3	0.534562149903011\\
3.31	0.534446617646162\\
3.32	0.534331507603356\\
3.33	0.534216818960743\\
3.34	0.534102550893163\\
3.35	0.533988702564312\\
3.36	0.533875273126917\\
3.37	0.533762261722903\\
3.38	0.533649667483567\\
3.39	0.533537489529753\\
3.4	0.533425726972022\\
3.41	0.533314378910829\\
3.42	0.533203444436694\\
3.43	0.533092922630382\\
3.44	0.532982812563073\\
3.45	0.532873113296541\\
3.46	0.532763823883325\\
3.47	0.532654943366907\\
3.48	0.532546470781887\\
3.49	0.532438405154155\\
3.5	0.532330745501065\\
3.51	0.532223490831612\\
3.52	0.532116640146603\\
3.53	0.532010192438827\\
3.54	0.531904146693232\\
3.55	0.531798501887092\\
3.56	0.53169325699018\\
3.57	0.531588410964935\\
3.58	0.531483962766634\\
3.59	0.531379911343555\\
3.6	0.53127625563715\\
3.61	0.531172994582204\\
3.62	0.531070127107004\\
3.63	0.530967652133503\\
3.64	0.530865568577477\\
3.65	0.530763875348693\\
3.66	0.530662571351065\\
3.67	0.530561655482812\\
3.68	0.530461126636618\\
3.69	0.530360983699785\\
3.7	0.530261225554392\\
3.71	0.530161851077444\\
3.72	0.530062859141024\\
3.73	0.529964248612449\\
3.74	0.529866018354411\\
3.75	0.529768167225131\\
3.76	0.529670694078502\\
3.77	0.529573597764234\\
3.78	0.529476877127997\\
3.79	0.529380531011565\\
3.8	0.529284558252951\\
3.81	0.529188957686551\\
3.82	0.529093728143277\\
3.83	0.528998868450694\\
3.84	0.528904377433155\\
3.85	0.528810253911929\\
3.86	0.528716496705337\\
3.87	0.528623104628877\\
3.88	0.528530076495355\\
3.89	0.528437411115006\\
3.9	0.528345107295625\\
3.91	0.528253163842683\\
3.92	0.528161579559452\\
3.93	0.528070353247125\\
3.94	0.527979483704931\\
3.95	0.527888969730254\\
3.96	0.527798810118745\\
3.97	0.527709003664441\\
3.98	0.527619549159869\\
3.99	0.52753044539616\\
4	0.52744169116316\\
4.01	0.527353285249533\\
4.02	0.527265226442865\\
4.03	0.527177513529775\\
4.04	0.52709014529601\\
4.05	0.527003120526548\\
4.06	0.5269164380057\\
4.07	0.526830096517205\\
4.08	0.526744094844326\\
4.09	0.526658431769947\\
4.1	0.526573106076663\\
4.11	0.526488116546875\\
4.12	0.52640346196288\\
4.13	0.526319141106958\\
4.14	0.526235152761459\\
4.15	0.526151495708892\\
4.16	0.526068168732008\\
4.17	0.525985170613882\\
4.18	0.525902500137997\\
4.19	0.525820156088322\\
4.2	0.525738137249393\\
4.21	0.52565644240639\\
4.22	0.525575070345214\\
4.23	0.525494019852558\\
4.24	0.525413289715987\\
4.25	0.525332878724006\\
4.26	0.52525278566613\\
4.27	0.525173009332959\\
4.28	0.525093548516242\\
4.29	0.525014402008943\\
4.3	0.524935568605313\\
4.31	0.524857047100948\\
4.32	0.524778836292856\\
4.33	0.52470093497952\\
4.34	0.524623341960957\\
4.35	0.524546056038777\\
4.36	0.524469076016244\\
4.37	0.524392400698335\\
4.38	0.52431602889179\\
4.39	0.524239959405175\\
4.4	0.52416419104893\\
4.41	0.524088722635426\\
4.42	0.524013552979014\\
4.43	0.523938680896078\\
4.44	0.523864105205086\\
4.45	0.523789824726632\\
4.46	0.523715838283493\\
4.47	0.523642144700668\\
4.48	0.523568742805429\\
4.49	0.523495631427361\\
4.5	0.52342280939841\\
4.51	0.523350275552923\\
4.52	0.52327802872769\\
4.53	0.523206067761985\\
4.54	0.523134391497605\\
4.55	0.523062998778913\\
4.56	0.522991888452868\\
4.57	0.52292105936907\\
4.58	0.522850510379793\\
4.59	0.522780240340019\\
4.6	0.522710248107473\\
4.61	0.522640532542659\\
4.62	0.522571092508889\\
4.63	0.52250192687232\\
4.64	0.522433034501977\\
4.65	0.522364414269793\\
4.66	0.522296065050631\\
4.67	0.522227985722315\\
4.68	0.52216017516566\\
4.69	0.522092632264497\\
4.7	0.522025355905697\\
4.71	0.521958344979204\\
4.72	0.521891598378053\\
4.73	0.521825114998395\\
4.74	0.521758893739525\\
4.75	0.521692933503901\\
4.76	0.521627233197167\\
4.77	0.521561791728173\\
4.78	0.521496608008999\\
4.79	0.521431680954974\\
4.8	0.521367009484693\\
4.81	0.521302592520039\\
4.82	0.5212384289862\\
4.83	0.521174517811686\\
4.84	0.521110857928347\\
4.85	0.52104744827139\\
4.86	0.520984287779391\\
4.87	0.520921375394317\\
4.88	0.520858710061533\\
4.89	0.520796290729824\\
4.9	0.5207341163514\\
4.91	0.520672185881917\\
4.92	0.520610498280486\\
4.93	0.520549052509682\\
4.94	0.520487847535561\\
4.95	0.520426882327669\\
4.96	0.520366155859051\\
4.97	0.520305667106261\\
4.98	0.520245415049374\\
4.99	0.520185398671993\\
5	0.520125616961257\\
5.01	0.520066068907853\\
5.02	0.520006753506019\\
5.03	0.519947669753552\\
5.04	0.519888816651817\\
5.05	0.519830193205754\\
5.06	0.519771798423879\\
5.07	0.519713631318296\\
5.08	0.519655690904695\\
5.09	0.519597976202365\\
5.1	0.519540486234192\\
5.11	0.519483220026664\\
5.12	0.519426176609878\\
5.13	0.519369355017538\\
5.14	0.519312754286964\\
5.15	0.519256373459089\\
5.16	0.519200211578464\\
5.17	0.51914426769326\\
5.18	0.519088540855269\\
5.19	0.519033030119903\\
5.2	0.518977734546199\\
5.21	0.518922653196817\\
5.22	0.51886778513804\\
5.23	0.518813129439775\\
5.24	0.518758685175554\\
5.25	0.518704451422528\\
5.26	0.518650427261472\\
5.27	0.518596611776782\\
5.28	0.518543004056472\\
5.29	0.518489603192173\\
5.3	0.518436408279131\\
5.31	0.518383418416206\\
5.32	0.518330632705868\\
5.33	0.518278050254194\\
5.34	0.518225670170865\\
5.35	0.518173491569166\\
5.36	0.518121513565977\\
5.37	0.518069735281773\\
5.38	0.518018155840619\\
5.39	0.517966774370168\\
5.4	0.517915590001651\\
5.41	0.517864601869879\\
5.42	0.517813809113233\\
5.43	0.517763210873664\\
5.44	0.517712806296682\\
5.45	0.517662594531355\\
5.46	0.517612574730304\\
5.47	0.517562746049692\\
5.48	0.517513107649223\\
5.49	0.517463658692135\\
5.5	0.517414398345193\\
5.51	0.517365325778681\\
5.52	0.517316440166399\\
5.53	0.517267740685654\\
5.54	0.517219226517251\\
5.55	0.517170896845493\\
5.56	0.517122750858164\\
5.57	0.51707478774653\\
5.58	0.517027006705328\\
5.59	0.516979406932758\\
5.6	0.516931987630475\\
5.61	0.516884748003584\\
5.62	0.516837687260627\\
5.63	0.516790804613582\\
5.64	0.516744099277847\\
5.65	0.516697570472237\\
5.66	0.516651217418972\\
5.67	0.516605039343673\\
5.68	0.516559035475346\\
5.69	0.516513205046382\\
5.7	0.51646754729254\\
5.71	0.516422061452944\\
5.72	0.516376746770071\\
5.73	0.516331602489741\\
5.74	0.516286627861109\\
5.75	0.516241822136658\\
5.76	0.516197184572184\\
5.77	0.51615271442679\\
5.78	0.516108410962878\\
5.79	0.516064273446135\\
5.8	0.516020301145526\\
5.81	0.515976493333283\\
5.82	0.515932849284896\\
5.83	0.515889368279102\\
5.84	0.515846049597876\\
5.85	0.51580289252642\\
5.86	0.515759896353153\\
5.87	0.5157170603697\\
5.88	0.515674383870884\\
5.89	0.515631866154712\\
5.9	0.515589506522368\\
5.91	0.515547304278201\\
5.92	0.515505258729715\\
5.93	0.515463369187557\\
5.94	0.515421634965508\\
5.95	0.515380055380472\\
5.96	0.515338629752463\\
5.97	0.515297357404599\\
5.98	0.515256237663087\\
5.99	0.515215269857213\\
6	0.515174453319333\\
6.01	0.515133787384861\\
6.02	0.515093271392256\\
6.03	0.515052904683015\\
6.04	0.515012686601658\\
6.05	0.514972616495721\\
6.06	0.514932693715741\\
6.07	0.514892917615248\\
6.08	0.514853287550752\\
6.09	0.514813802881734\\
6.1	0.514774462970631\\
6.11	0.514735267182829\\
6.12	0.51469621488665\\
6.13	0.51465730545334\\
6.14	0.514618538257061\\
6.15	0.514579912674874\\
6.16	0.514541428086736\\
6.17	0.514503083875478\\
6.18	0.514464879426807\\
6.19	0.514426814129281\\
6.2	0.514388887374308\\
6.21	0.514351098556131\\
6.22	0.514313447071815\\
6.23	0.514275932321239\\
6.24	0.514238553707082\\
6.25	0.514201310634815\\
6.26	0.514164202512686\\
6.27	0.514127228751711\\
6.28	0.514090388765661\\
6.29	0.514053681971053\\
6.3	0.514017107787139\\
6.31	0.51398066563589\\
6.32	0.513944354941991\\
6.33	0.513908175132824\\
6.34	0.513872125638463\\
6.35	0.513836205891656\\
6.36	0.513800415327819\\
6.37	0.513764753385023\\
6.38	0.513729219503981\\
6.39	0.51369381312804\\
6.4	0.513658533703168\\
6.41	0.513623380677942\\
6.42	0.513588353503539\\
6.43	0.513553451633725\\
6.44	0.513518674524839\\
6.45	0.513484021635788\\
6.46	0.513449492428034\\
6.47	0.513415086365581\\
6.48	0.513380802914965\\
6.49	0.513346641545244\\
6.5	0.513312601727986\\
6.51	0.513278682937258\\
6.52	0.513244884649616\\
6.53	0.513211206344091\\
6.54	0.513177647502183\\
6.55	0.513144207607844\\
6.56	0.513110886147475\\
6.57	0.513077682609905\\
6.58	0.51304459648639\\
6.59	0.513011627270596\\
6.6	0.512978774458589\\
6.61	0.512946037548827\\
6.62	0.512913416042147\\
6.63	0.512880909441754\\
6.64	0.512848517253211\\
6.65	0.512816238984429\\
6.66	0.512784074145654\\
6.67	0.512752022249459\\
6.68	0.512720082810733\\
6.69	0.512688255346669\\
6.7	0.512656539376753\\
6.71	0.512624934422757\\
6.72	0.512593440008725\\
6.73	0.512562055660962\\
6.74	0.512530780908028\\
6.75	0.512499615280722\\
6.76	0.512468558312078\\
6.77	0.512437609537348\\
6.78	0.512406768493997\\
6.79	0.512376034721687\\
6.8	0.512345407762274\\
6.81	0.512314887159791\\
6.82	0.512284472460444\\
6.83	0.512254163212595\\
6.84	0.512223958966759\\
6.85	0.512193859275587\\
6.86	0.512163863693861\\
6.87	0.512133971778482\\
6.88	0.512104183088461\\
6.89	0.512074497184906\\
6.9	0.512044913631018\\
6.91	0.512015431992074\\
6.92	0.511986051835421\\
6.93	0.511956772730468\\
6.94	0.511927594248672\\
6.95	0.51189851596353\\
6.96	0.51186953745057\\
6.97	0.511840658287341\\
6.98	0.511811878053402\\
6.99	0.511783196330315\\
7	0.511754612701632\\
7.01	0.511726126752888\\
7.02	0.511697738071591\\
7.03	0.511669446247211\\
7.04	0.511641250871174\\
7.05	0.511613151536848\\
7.06	0.511585147839537\\
7.07	0.511557239376471\\
7.08	0.511529425746796\\
7.09	0.511501706551563\\
7.1	0.511474081393723\\
7.11	0.511446549878116\\
7.12	0.511419111611459\\
7.13	0.511391766202342\\
7.14	0.511364513261212\\
7.15	0.511337352400373\\
7.16	0.511310283233968\\
7.17	0.511283305377977\\
7.18	0.511256418450202\\
7.19	0.511229622070264\\
7.2	0.51120291585959\\
7.21	0.511176299441405\\
7.22	0.511149772440723\\
7.23	0.511123334484342\\
7.24	0.511096985200827\\
7.25	0.511070724220512\\
7.26	0.51104455117548\\
7.27	0.511018465699565\\
7.28	0.510992467428336\\
7.29	0.51096655599909\\
7.3	0.510940731050846\\
7.31	0.510914992224335\\
7.32	0.510889339161991\\
7.33	0.510863771507941\\
7.34	0.510838288908002\\
7.35	0.510812891009668\\
7.36	0.510787577462101\\
7.37	0.510762347916128\\
7.38	0.510737202024227\\
7.39	0.510712139440522\\
7.4	0.510687159820775\\
7.41	0.510662262822374\\
7.42	0.510637448104333\\
7.43	0.510612715327273\\
7.44	0.510588064153425\\
7.45	0.510563494246613\\
7.46	0.510539005272251\\
7.47	0.510514596897335\\
7.48	0.510490268790433\\
7.49	0.510466020621678\\
7.5	0.510441852062762\\
7.51	0.510417762786925\\
7.52	0.510393752468949\\
7.53	0.510369820785151\\
7.54	0.510345967413375\\
7.55	0.510322192032982\\
7.56	0.510298494324846\\
7.57	0.510274873971344\\
7.58	0.510251330656349\\
7.59	0.510227864065224\\
7.6	0.510204473884812\\
7.61	0.51018115980343\\
7.62	0.510157921510862\\
7.63	0.510134758698351\\
7.64	0.510111671058592\\
7.65	0.510088658285723\\
7.66	0.510065720075321\\
7.67	0.510042856124393\\
7.68	0.510020066131368\\
7.69	0.509997349796091\\
7.7	0.509974706819816\\
7.71	0.509952136905198\\
7.72	0.509929639756287\\
7.73	0.509907215078519\\
7.74	0.509884862578713\\
7.75	0.509862581965059\\
7.76	0.509840372947116\\
7.77	0.509818235235801\\
7.78	0.509796168543384\\
7.79	0.509774172583483\\
7.8	0.509752247071052\\
7.81	0.509730391722381\\
7.82	0.509708606255084\\
7.83	0.509686890388094\\
7.84	0.509665243841657\\
7.85	0.509643666337325\\
7.86	0.50962215759795\\
7.87	0.509600717347675\\
7.88	0.509579345311931\\
7.89	0.509558041217428\\
7.9	0.509536804792148\\
7.91	0.509515635765343\\
7.92	0.509494533867523\\
7.93	0.509473498830453\\
7.94	0.509452530387145\\
7.95	0.509431628271853\\
7.96	0.509410792220067\\
7.97	0.509390021968506\\
7.98	0.50936931725511\\
7.99	0.509348677819037\\
8	0.509328103400655\\
8.01	0.509307593741537\\
8.02	0.509287148584455\\
8.03	0.509266767673369\\
8.04	0.509246450753431\\
8.05	0.50922619757097\\
8.06	0.509206007873489\\
8.07	0.509185881409661\\
8.08	0.509165817929319\\
8.09	0.509145817183456\\
8.1	0.509125878924212\\
8.11	0.509106002904875\\
8.12	0.50908618887987\\
8.13	0.509066436604756\\
8.14	0.50904674583622\\
8.15	0.509027116332071\\
8.16	0.509007547851233\\
8.17	0.508988040153742\\
8.18	0.508968593000739\\
8.19	0.508949206154463\\
8.2	0.508929879378248\\
8.21	0.508910612436517\\
8.22	0.508891405094775\\
8.23	0.508872257119604\\
8.24	0.508853168278659\\
8.25	0.50883413834066\\
8.26	0.50881516707539\\
8.27	0.508796254253687\\
8.28	0.508777399647438\\
8.29	0.508758603029578\\
8.3	0.50873986417408\\
8.31	0.508721182855951\\
8.32	0.508702558851231\\
8.33	0.508683991936979\\
8.34	0.508665481891277\\
8.35	0.50864702849322\\
8.36	0.508628631522911\\
8.37	0.508610290761458\\
8.38	0.508592005990965\\
8.39	0.508573776994533\\
8.4	0.50855560355625\\
8.41	0.508537485461187\\
8.42	0.508519422495395\\
8.43	0.508501414445898\\
8.44	0.508483461100688\\
8.45	0.508465562248723\\
8.46	0.508447717679918\\
8.47	0.508429927185144\\
8.48	0.508412190556221\\
8.49	0.508394507585912\\
8.5	0.508376878067923\\
8.51	0.508359301796891\\
8.52	0.508341778568388\\
8.53	0.508324308178909\\
8.54	0.508306890425871\\
8.55	0.508289525107607\\
8.56	0.508272212023362\\
8.57	0.50825495097329\\
8.58	0.508237741758445\\
8.59	0.508220584180781\\
8.6	0.508203478043145\\
8.61	0.508186423149273\\
8.62	0.508169419303788\\
8.63	0.508152466312189\\
8.64	0.508135563980856\\
8.65	0.508118712117035\\
8.66	0.508101910528845\\
8.67	0.508085159025262\\
8.68	0.508068457416125\\
8.69	0.508051805512125\\
8.7	0.508035203124803\\
8.71	0.508018650066546\\
8.72	0.508002146150581\\
8.73	0.507985691190974\\
8.74	0.507969285002624\\
8.75	0.507952927401257\\
8.76	0.507936618203424\\
8.77	0.507920357226499\\
8.78	0.507904144288668\\
8.79	0.507887979208934\\
8.8	0.507871861807105\\
8.81	0.507855791903794\\
8.82	0.507839769320415\\
8.83	0.507823793879177\\
8.84	0.507807865403083\\
8.85	0.507791983715921\\
8.86	0.507776148642267\\
8.87	0.507760360007476\\
8.88	0.507744617637678\\
8.89	0.507728921359777\\
8.9	0.507713271001447\\
8.91	0.507697666391124\\
8.92	0.507682107358008\\
8.93	0.507666593732054\\
8.94	0.507651125343973\\
8.95	0.507635702025223\\
8.96	0.507620323608011\\
8.97	0.507604989925284\\
8.98	0.50758970081073\\
8.99	0.507574456098768\\
9	0.507559255624553\\
9.01	0.507544099223966\\
9.02	0.50752898673361\\
9.03	0.507513917990812\\
9.04	0.507498892833613\\
9.05	0.507483911100769\\
9.06	0.507468972631745\\
9.07	0.507454077266714\\
9.08	0.507439224846549\\
9.09	0.507424415212825\\
9.1	0.507409648207811\\
9.11	0.507394923674469\\
9.12	0.50738024145645\\
9.13	0.507365601398091\\
9.14	0.507351003344411\\
9.15	0.507336447141107\\
9.16	0.507321932634553\\
9.17	0.507307459671792\\
9.18	0.50729302810054\\
9.19	0.507278637769176\\
9.2	0.507264288526741\\
9.21	0.507249980222935\\
9.22	0.507235712708116\\
9.23	0.50722148583329\\
9.24	0.507207299450117\\
9.25	0.5071931534109\\
9.26	0.507179047568586\\
9.27	0.507164981776761\\
9.28	0.507150955889648\\
9.29	0.507136969762103\\
9.3	0.507123023249612\\
9.31	0.507109116208289\\
9.32	0.507095248494873\\
9.33	0.507081419966721\\
9.34	0.507067630481811\\
9.35	0.507053879898733\\
9.36	0.507040168076693\\
9.37	0.507026494875501\\
9.38	0.507012860155577\\
9.39	0.506999263777942\\
9.4	0.506985705604217\\
9.41	0.50697218549662\\
9.42	0.506958703317965\\
9.43	0.506945258931655\\
9.44	0.506931852201681\\
9.45	0.506918482992622\\
9.46	0.506905151169638\\
9.47	0.506891856598468\\
9.48	0.50687859914543\\
9.49	0.506865378677413\\
9.5	0.506852195061881\\
9.51	0.506839048166865\\
9.52	0.506825937860959\\
9.53	0.506812864013324\\
9.54	0.50679982649368\\
9.55	0.506786825172304\\
9.56	0.506773859920027\\
9.57	0.506760930608233\\
9.58	0.506748037108856\\
9.59	0.506735179294376\\
9.6	0.506722357037817\\
9.61	0.506709570212745\\
9.62	0.506696818693263\\
9.63	0.506684102354014\\
9.64	0.50667142107017\\
9.65	0.506658774717437\\
9.66	0.506646163172048\\
9.67	0.506633586310764\\
9.68	0.506621044010866\\
9.69	0.506608536150158\\
9.7	0.506596062606962\\
9.71	0.506583623260115\\
9.72	0.506571217988969\\
9.73	0.506558846673383\\
9.74	0.506546509193729\\
9.75	0.506534205430883\\
9.76	0.506521935266222\\
9.77	0.506509698581627\\
9.78	0.506497495259477\\
9.79	0.506485325182646\\
9.8	0.506473188234503\\
9.81	0.506461084298908\\
9.82	0.506449013260209\\
9.83	0.506436975003243\\
9.84	0.506424969413329\\
9.85	0.506412996376268\\
9.86	0.506401055778344\\
9.87	0.506389147506314\\
9.88	0.506377271447413\\
9.89	0.506365427489347\\
9.9	0.506353615520294\\
9.91	0.5063418354289\\
9.92	0.506330087104275\\
9.93	0.506318370435997\\
9.94	0.506306685314101\\
9.95	0.506295031629084\\
9.96	0.506283409271901\\
9.97	0.506271818133958\\
9.98	0.506260258107119\\
9.99	0.506248729083695\\
10	0.506237230956447\\
};
\end{axis}
\end{tikzpicture}%
  \caption{The prior and posterior distributions over $\theta$ for the
    coin-flipping problem.}
\end{figure}

\clearpage

\begin{enumerate}
\setcounter{enumi}{4}
\item
  (Gaussian with unknown mean.)
  Let $\bm{x} = \{ x_i \}_{i = 1}^N$ be independent, identically
  distributed real-valued random variables with distribution $p(x_i
  \given \theta) = \mc{N}(x_i; \theta, \sigma^2)$.  Suppose the
  variance $\sigma^2$ is known but the mean $\theta$ is unknown with
  prior distribution $p(\theta) = \mc{N}(\theta; 0, 1^2)$.
  \begin{itemize}
  \item
    What is the likelihood of the full observation vector $p(\bm{x}
    \given \theta)$?
  \item
    After observing $\bm{x}$, what is the posterior distribution of
    $\theta$, $p(\theta \given \bm{x}, \sigma^2)$?  (Note: you might
    find it more convenient in this case to work with the
    \emph{precision} $\tau = \sigma^{-2}$.)
  \item
    Interpret how the posterior changes as a function of $N$.  What
    happens if $N = 0$?  What happens if $N \to \infty$?  Does this
    agree with your intuition?
  \end{itemize}
\end{enumerate}

\subsection*{Solution}
In the first part, we use the definition of independence:
\begin{equation*}
  p(\bm{x} \given \theta)
  =
  \prod_{i = 1}^N
  p(x_i \given \theta)
  =
  \prod_{i = 1}^N
  \mc{N}(x_i; \theta, \sigma^2).
\end{equation*}

We will consider the slightly more general case with an arbitrary
Gaussian prior on $\theta$: $p(\theta \given m, s^2) = \mc{N}(\theta;
m, s^2)$.  For convenience, we will parameterize the Gaussians on the
$\{x_i\}$ and $\theta$ with the \emph{precision} parameters $\tau =
\sigma^{-2}; t = s^{-2}$.

By Bayes' theorem, we have
\begin{align*}
  p(\theta \given \bm{x})
  &\propto
  p(\bm{x} \given \theta)
  p(\theta)
  \\
  &=
  \prod \mc{N}(x_i; \theta, \tau)
  \,
  \mc{N}(\theta; m, t)
  \\
  &\propto
  \exp\Biggl(
  -\frac{1}{2}
  \biggl(
  \tau
  \sum_{i = 1}^N (x_i - \theta)^2
  +
  t
  (\theta - m)^2
  \biggr)
  \Biggr).
\end{align*}
This is an exponentiated, negative quadratic function of $\theta$ and
is therefore proportional to a Gaussian distribution over $\theta$.
Our goal is to identify the mean and precision of this distribution.

We expand:
\begin{equation*}
  \tau
  \sum_{i = 1}^N (x_i - \theta)^2
  +
  t
  (\theta - m)^2
  =
  (\tau N + t)\theta^2
  - 2(\tau \textstyle\sum x_i + tm) \theta
  + (\tau \sum x_i^2 + t m^2).
\end{equation*}
Notice that $\sum x_i = N \bar{x}$, where $\bar{x}$ is the sample mean
of the measurements.  Notice also that the last term is a constant
independent of $\theta$, which will be absorbed into the normalizing
constant.  To arrive at a more-familiar form, we ``complete the
square:''
\begin{equation*}
  (\tau N + t)\theta^2
  - 2(\tau N \bar{x} + tm) \theta
  =
  (\tau N + t) \biggl(
  \theta -
  \frac{\tau N \bar{x} + tm}{\tau N + t}
  \biggr)^2
  +
  c,
\end{equation*}
where $c$ is another constant independent of $\theta$.  Therefore:
\begin{align*}
  p(\theta \given \bm{x})
  &\propto
  \exp\bigl(
  \textstyle -\frac{t'}{2}
  (\theta - m')^2
  \bigr),
\end{align*}
which is a Gaussian distribution with mean and precision
\begin{equation*}
  m' = \frac{\tau N \bar{x} + t m}{\tau N + t};
  \qquad
  t' = \tau N + t,
\end{equation*}
respectively.  We recognize that the new precision is the sum of the
precisions of each measurement (the $N$ independent measurements
$\{x_i\}$ with precision $\tau$ aplus one additional prior
``measurement.'' with precision $t$).  The posterior mean can be
recognized as a precision-weighted average of the measurements,
including the ``measurement'' at the prior mean $m$.

As $N \to \infty$, the sample mean $\bar{x}$ dominates, and the
influence of the prior diminishes to zero.  If $N = 0$, we rely only
on our prior knowledge.

\clearpage

\begin{enumerate}
\setcounter{enumi}{5}
\item
  (Spike and slab priors.)
  Suppose $\theta$ is a real-valued random variable that is expected
  to either be near zero (with probability $\pi$) or to have a wide
  range of potential values (with probability $(1 - \pi)$).  Such
  scenarios happen a lot in practice: for example, $\theta$ could be
  the coefficient of a feature in a regression model.  We either
  expect the feature to be useless for predicting the output (and have
  a value close to zero) or to be useful, in which case we expect a
  value with larger magnitude but can't say much else.

  A common approach in this case is to use a so-called \emph{spike and
    slab prior.} Let $f \in \{0, 1\}$ be a discrete random variable
  serving as a flag.  We define the following conditional prior:
  \begin{equation*}
    p(\theta \given f, \sigma_{\text{spike}}^2, \sigma_{\text{slab}}^2)
    =
    \begin{cases}
      \mc{N}(\theta; 0, \sigma_{\text{spike}}^2) & f = 0 \\
      \mc{N}(\theta; 0, \sigma_{\text{slab}}^2)  & f = 1,
    \end{cases}
  \end{equation*}
  where $\sigma_{\text{spike}}$ is the width of a narrow ``spike'' at
  zero, and $\sigma_{\text{slab}} > \sigma_{\text{spike}}$ is the
  width of a ``slab'' supporting values with larger magnitude.

  In practice, we will never observe the flag variable $f$; instead,
  we must infer it or marginalize it, as required.
  \begin{itemize}
  \item
    Suppose we choose a prior $\Pr(f = 1) = \pi = \nicefrac{1}{2}$,
    expressing no \emph{a priori} preference for the spike or the
    slab.  What is the marginal prior $p(\theta \given
    \sigma_{\text{spike}}^2, \sigma_{\text{slab}}^2$)?  Plot the
    marginal prior distribution for $(\sigma_{\text{spike}}^2,
    \sigma_{\text{slab}}^2) = (1^2, 10^2)$.
  \item
    Suppose that we can make a noisy observation $x$ of $\theta$, with
    distribution $p(x \given \theta, \omega^2) = \mc{N}(x; \theta,
    \omega^2)$, with known variance $\omega^2$.  Given $x$, what is
    the posterior distribution of the flag parameter, $\Pr(f = 1
    \given x, \sigma_{\text{spike}}^2, \sigma_{\text{slab}}^2,
    \omega^2)$?  Plot this distribution as a function of $x$.  What
    observation would teach us the most about $f$?  What teaches us
    the least?
  \item
    Given an observation $x$ as in the last part, what is the
    posterior distribution of $\theta$, $p(\theta \given x,
    \sigma_{\text{spike}}^2, \sigma_{\text{slab}}^2, \omega^2)$?
    (Hint: use the sum rule to eliminate $f$ and use the result
    above.)
  \item
    Suppose the noise variance is $\omega^2 = 0.1^2$ and we make an
    observation $x = 3$.  Plot the posterior distribution of $\theta$,
    using the parameters from the first part.
  \end{itemize}
\end{enumerate}

\subsection*{Solution}
For the first part, we use the sum rule:
\begin{equation*}
  p(\theta \given \sigma_{\text{spike}}^2, \sigma_{\text{slab}}^2)
  =
  \sum_f
  \Pr(f)p(\theta \given f, \sigma_{\text{spike}}^2, \sigma_{\text{slab}}^2)
  =
  \frac{1}{2}\mc{N}(\theta; 0, \sigma_{\text{spike}}^2) +
  \frac{1}{2}\mc{N}(\theta; 0, \sigma_{\text{slab}}^2).
\end{equation*}
The prior density is plotted below.

\begin{figure}[h]
  \centering
  % This file was created by matlab2tikz.
% Minimal pgfplots version: 1.3
%
\tikzsetnextfilename{problem_6_prior}
\definecolor{mycolor1}{rgb}{0.12157,0.47059,0.70588}%
%
\begin{tikzpicture}

\begin{axis}[%
width=0.95092\figurewidth,
height=\figureheight,
at={(0\figurewidth,0\figureheight)},
scale only axis,
xmin=-30,
xmax=30,
xlabel={$\theta$},
ymin=0,
ymax=0.25,
axis x line*=bottom,
axis y line*=left,
legend style={at={(0.03,0.97)},anchor=north west,legend cell align=left,align=left,fill=none,draw=none}
]
\addplot [color=mycolor1,solid]
  table[row sep=crcr]{%
-30	0.0002215924205969\\
-29.9399399399399	0.000225617194431115\\
-29.8798798798799	0.000229706783906147\\
-29.8198198198198	0.000233862066176456\\
-29.7597597597598	0.000238083927106706\\
-29.6996996996997	0.000242373261300351\\
-29.6396396396396	0.000246730972127154\\
-29.5795795795796	0.00025115797174961\\
-29.5195195195195	0.000255655181148236\\
-29.4594594594595	0.000260223530145722\\
-29.3993993993994	0.000264863957429895\\
-29.3393393393393	0.000269577410575485\\
-29.2792792792793	0.000274364846064665\\
-29.2192192192192	0.000279227229306332\\
-29.1591591591592	0.000284165534654112\\
-29.0990990990991	0.000289180745423046\\
-29.039039039039	0.000294273853904956\\
-28.978978978979	0.000299445861382437\\
-28.9189189189189	0.000304697778141464\\
-28.8588588588589	0.000310030623482585\\
-28.7987987987988	0.000315445425730669\\
-28.7387387387387	0.000320943222243179\\
-28.6786786786787	0.000326525059416953\\
-28.6186186186186	0.000332191992693455\\
-28.5585585585586	0.000337945086562472\\
-28.4984984984985	0.00034378541456423\\
-28.4384384384384	0.000349714059289901\\
-28.3783783783784	0.000355732112380472\\
-28.3183183183183	0.000361840674523944\\
-28.2582582582583	0.000368040855450846\\
-28.1981981981982	0.00037433377392802\\
-28.1381381381381	0.000380720557750653\\
-28.0780780780781	0.000387202343732544\\
-28.018018018018	0.000393780277694546\\
-27.957957957958	0.00040045551445119\\
-27.8978978978979	0.000407229217795432\\
-27.8378378378378	0.000414102560481516\\
-27.7777777777778	0.000421076724205918\\
-27.7177177177177	0.000428152899586331\\
-27.6576576576577	0.000435332286138691\\
-27.5975975975976	0.000442616092252179\\
-27.5375375375375	0.000450005535162202\\
-27.4774774774775	0.000457501840921312\\
-27.4174174174174	0.000465106244368028\\
-27.3573573573574	0.000472819989093549\\
-27.2972972972973	0.000480644327406314\\
-27.2372372372372	0.000488580520294397\\
-27.1771771771772	0.000496629837385689\\
-27.1171171171171	0.000504793556905862\\
-27.0570570570571	0.000513072965634072\\
-26.996996996997	0.000521469358856374\\
-26.9369369369369	0.000529984040316832\\
-26.8768768768769	0.000538618322166284\\
-26.8168168168168	0.000547373524908743\\
-26.7567567567568	0.000556250977345413\\
-26.6966966966967	0.000565252016516269\\
-26.6366366366366	0.000574377987639202\\
-26.5765765765766	0.000583630244046695\\
-26.5165165165165	0.000593010147119985\\
-26.4564564564565	0.000602519066220718\\
-26.3963963963964	0.000612158378620038\\
-26.3363363363363	0.000621929469425108\\
-26.2762762762763	0.000631833731503031\\
-26.2162162162162	0.000641872565402138\\
-26.1561561561562	0.00065204737927063\\
-26.0960960960961	0.000662359588772542\\
-26.036036036036	0.000672810617001008\\
-25.975975975976	0.000683401894388802\\
-25.9159159159159	0.00069413485861613\\
-25.8558558558559	0.000705010954515658\\
-25.7957957957958	0.000716031633974735\\
-25.7357357357357	0.000727198355834811\\
-25.6756756756757	0.000738512585788013\\
-25.6156156156156	0.000749975796270856\\
-25.5555555555556	0.000761589466355083\\
-25.4954954954955	0.000773355081635594\\
-25.4354354354354	0.000785274134115452\\
-25.3753753753754	0.000797348122087952\\
-25.3153153153153	0.000809578550015724\\
-25.2552552552553	0.000821966928406852\\
-25.1951951951952	0.000834514773687991\\
-25.1351351351351	0.000847223608074471\\
-25.0750750750751	0.000860094959437356\\
-25.015015015015	0.000873130361167452\\
-24.954954954955	0.000886331352036248\\
-24.8948948948949	0.000899699476053755\\
-24.8348348348348	0.000913236282323261\\
-24.7747747747748	0.000926943324892946\\
-24.7147147147147	0.000940822162604383\\
-24.6546546546547	0.000954874358937879\\
-24.5945945945946	0.000969101481854654\\
-24.5345345345345	0.000983505103635855\\
-24.4744744744745	0.000998086800718375\\
-24.4144144144144	0.00101284815352748\\
-24.3543543543544	0.00102779074630622\\
-24.2942942942943	0.00104291616694162\\
-24.2342342342342	0.00105822600678767\\
-24.1741741741742	0.00107372186048501\\
-24.1141141141141	0.00108940532577745\\
-24.0540540540541	0.00110527800332518\\
-23.993993993994	0.00112134149651472\\
-23.9339339339339	0.00113759741126567\\
-23.8738738738739	0.00115404735583405\\
-23.8138138138138	0.00117069294061252\\
-23.7537537537538	0.00118753577792723\\
-23.6936936936937	0.00120457748183136\\
-23.6336336336336	0.00122181966789546\\
-23.5735735735736	0.00123926395299443\\
-23.5135135135135	0.00125691195509125\\
-23.4534534534535	0.00127476529301739\\
-23.3933933933934	0.00129282558624993\\
-23.3333333333333	0.00131109445468548\\
-23.2732732732733	0.00132957351841064\\
-23.2132132132132	0.00134826439746938\\
-23.1531531531532	0.001367168711627\\
-23.0930930930931	0.00138628808013086\\
-23.033033033033	0.00140562412146789\\
-22.972972972973	0.00142517845311876\\
-22.9129129129129	0.00144495269130888\\
-22.8528528528529	0.00146494845075611\\
-22.7927927927928	0.00148516734441522\\
-22.7327327327327	0.00150561098321923\\
-22.6726726726727	0.0015262809758174\\
-22.6126126126126	0.00154717892831015\\
-22.5525525525526	0.00156830644398073\\
-22.4924924924925	0.00158966512302373\\
-22.4324324324324	0.00161125656227044\\
-22.3723723723724	0.00163308235491111\\
-22.3123123123123	0.00165514409021405\\
-22.2522522522523	0.00167744335324164\\
-22.1921921921922	0.00169998172456328\\
-22.1321321321321	0.0017227607799653\\
-22.0720720720721	0.00174578209015774\\
-22.012012012012	0.00176904722047827\\
-21.951951951952	0.00179255773059297\\
-21.8918918918919	0.00181631517419426\\
-21.8318318318318	0.00184032109869583\\
-21.7717717717718	0.0018645770449247\\
-21.7117117117117	0.00188908454681037\\
-21.6516516516517	0.00191384513107113\\
-21.5915915915916	0.00193886031689755\\
-21.5315315315315	0.0019641316156332\\
-21.4714714714715	0.00198966053045261\\
-21.4114114114114	0.00201544855603646\\
-21.3513513513514	0.00204149717824415\\
-21.2912912912913	0.0020678078737837\\
-21.2312312312312	0.00209438210987902\\
-21.1711711711712	0.00212122134393464\\
-21.1111111111111	0.00214832702319786\\
-21.0510510510511	0.00217570058441848\\
-20.990990990991	0.002203343453506\\
-20.9309309309309	0.00223125704518446\\
-20.8708708708709	0.00225944276264494\\
-20.8108108108108	0.00228790199719567\\
-20.7507507507508	0.00231663612790995\\
-20.6906906906907	0.0023456465212718\\
-20.6306306306306	0.00237493453081945\\
-20.5705705705706	0.00240450149678674\\
-20.5105105105105	0.00243434874574238\\
-20.4504504504505	0.00246447759022726\\
-20.3903903903904	0.00249488932838976\\
-20.3303303303303	0.00252558524361917\\
-20.2702702702703	0.00255656660417727\\
-20.2102102102102	0.00258783466282806\\
-20.1501501501502	0.0026193906564658\\
-20.0900900900901	0.00265123580574138\\
-20.03003003003	0.00268337131468706\\
-19.96996996997	0.00271579837033963\\
-19.9099099099099	0.00274851814236212\\
-19.8498498498498	0.00278153178266406\\
-19.7897897897898	0.00281484042502043\\
-19.7297297297297	0.00284844518468919\\
-19.6696696696697	0.00288234715802774\\
-19.6096096096096	0.00291654742210813\\
-19.5495495495495	0.00295104703433119\\
-19.4894894894895	0.00298584703203966\\
-19.4294294294294	0.00302094843213038\\
-19.3693693693694	0.0030563522306656\\
-19.3093093093093	0.00309205940248347\\
-19.2492492492492	0.00312807090080786\\
-19.1891891891892	0.00316438765685751\\
-19.1291291291291	0.00320101057945457\\
-19.0690690690691	0.00323794055463275\\
-19.009009009009	0.00327517844524496\\
-18.9489489489489	0.00331272509057066\\
-18.8888888888889	0.00335058130592301\\
-18.8288288288288	0.00338874788225576\\
-18.7687687687688	0.00342722558577016\\
-18.7087087087087	0.00346601515752179\\
-18.6486486486486	0.00350511731302755\\
-18.5885885885886	0.00354453274187276\\
-18.5285285285285	0.00358426210731858\\
-18.4684684684685	0.00362430604590976\\
-18.4084084084084	0.00366466516708284\\
-18.3483483483483	0.00370534005277482\\
-18.2882882882883	0.00374633125703258\\
-18.2282282282282	0.00378763930562291\\
-18.1681681681682	0.00382926469564336\\
-18.1081081081081	0.00387120789513405\\
-18.048048048048	0.00391346934269039\\
-17.987987987988	0.003956049447077\\
-17.9279279279279	0.00399894858684269\\
-17.8678678678679	0.00404216710993684\\
-17.8078078078078	0.00408570533332707\\
-17.7477477477477	0.00412956354261848\\
-17.6876876876877	0.00417374199167434\\
-17.6276276276276	0.00421824090223863\\
-17.5675675675676	0.00426306046356019\\
-17.5075075075075	0.00430820083201885\\
-17.4474474474474	0.00435366213075351\\
-17.3873873873874	0.00439944444929231\\
-17.3273273273273	0.00444554784318498\\
-17.2672672672673	0.00449197233363745\\
-17.2072072072072	0.00453871790714891\\
-17.1471471471471	0.00458578451515132\\
-17.0870870870871	0.00463317207365153\\
-17.027027027027	0.00468088046287611\\
-16.966966966967	0.00472890952691898\\
-16.9069069069069	0.004777259073392\\
-16.8468468468468	0.00482592887307846\\
-16.7867867867868	0.00487491865958987\\
-16.7267267267267	0.00492422812902586\\
-16.6666666666667	0.00497385693963744\\
-16.6066066066066	0.00502380471149376\\
-16.5465465465465	0.00507407102615237\\
-16.4864864864865	0.00512465542633319\\
-16.4264264264264	0.00517555741559619\\
-16.3663663663664	0.00522677645802302\\
-16.3063063063063	0.0052783119779026\\
-16.2462462462462	0.00533016335942073\\
-16.1861861861862	0.00538232994635409\\
-16.1261261261261	0.00543481104176831\\
-16.0660660660661	0.00548760590772068\\
-16.006006006006	0.00554071376496723\\
-15.9459459459459	0.00559413379267453\\
-15.8858858858859	0.00564786512813618\\
-15.8258258258258	0.00570190686649423\\
-15.7657657657658	0.00575625806046545\\
-15.7057057057057	0.00581091772007273\\
-15.6456456456456	0.00586588481238169\\
-15.5855855855856	0.00592115826124242\\
-15.5255255255255	0.00597673694703682\\
-15.4654654654655	0.00603261970643122\\
-15.4054054054054	0.00608880533213473\\
-15.3453453453453	0.00614529257266324\\
-15.2852852852853	0.00620208013210924\\
-15.2252252252252	0.00625916666991756\\
-15.1651651651652	0.00631655080066705\\
-15.1051051051051	0.0063742310938585\\
-15.045045045045	0.00643220607370869\\
-14.984984984985	0.00649047421895075\\
-14.9249249249249	0.00654903396264101\\
-14.8648648648649	0.00660788369197236\\
-14.8048048048048	0.00666702174809411\\
-14.7447447447447	0.00672644642593878\\
-14.6846846846847	0.00678615597405554\\
-14.6246246246246	0.00684614859445061\\
-14.5645645645646	0.00690642244243471\\
-14.5045045045045	0.00696697562647758\\
-14.4444444444444	0.00702780620806973\\
-14.3843843843844	0.00708891220159142\\
-14.3243243243243	0.00715029157418916\\
-14.2642642642643	0.0072119422456595\\
-14.2042042042042	0.00727386208834058\\
-14.1441441441441	0.00733604892701122\\
-14.0840840840841	0.00739850053879777\\
-14.024024024024	0.00746121465308885\\
-13.963963963964	0.00752418895145797\\
-13.9039039039039	0.00758742106759409\\
-13.8438438438438	0.00765090858724042\\
-13.7837837837838	0.00771464904814124\\
-13.7237237237237	0.00777863993999704\\
-13.6636636636637	0.00784287870442798\\
-13.6036036036036	0.00790736273494572\\
-13.5435435435435	0.0079720893769338\\
-13.4834834834835	0.00803705592763646\\
-13.4234234234234	0.00810225963615623\\
-13.3633633633634	0.00816769770346009\\
-13.3033033033033	0.00823336728239447\\
-13.2432432432432	0.00829926547770908\\
-13.1831831831832	0.00836538934608957\\
-13.1231231231231	0.00843173589619928\\
-13.0630630630631	0.00849830208872991\\
-13.003003003003	0.00856508483646135\\
-12.9429429429429	0.00863208100433057\\
-12.8828828828829	0.00869928740950985\\
-12.8228228228228	0.00876670082149414\\
-12.7627627627628	0.0088343179621978\\
-12.7027027027027	0.00890213550606072\\
-12.6426426426426	0.0089701500801638\\
-12.5825825825826	0.00903835826435395\\
-12.5225225225225	0.00910675659137854\\
-12.4624624624625	0.00917534154702946\\
-12.4024024024024	0.00924410957029677\\
-12.3423423423423	0.00931305705353189\\
-12.2822822822823	0.00938218034262059\\
-12.2222222222222	0.00945147573716561\\
-12.1621621621622	0.00952093949067904\\
-12.1021021021021	0.00959056781078441\\
-12.042042042042	0.0096603568594287\\
-11.981981981982	0.00973030275310403\\
-11.9219219219219	0.00980040156307933\\
-11.8618618618619	0.00987064931564172\\
-11.8018018018018	0.00994104199234798\\
-11.7417417417417	0.0100115755302857\\
-11.6816816816817	0.0100822458223445\\
-11.6216216216216	0.0101530487174971\\
-11.5615615615616	0.0102239800210906\\
-11.5015015015015	0.0102950354951469\\
-11.4414414414414	0.0103662108586743\\
-11.3813813813814	0.010437501787988\\
-11.3213213213213	0.010508903917041\\
-11.2612612612613	0.0105804128377648\\
-11.2012012012012	0.0106520241004205\\
-11.1411411411411	0.0107237332139589\\
-11.0810810810811	0.0107955356463915\\
-11.021021021021	0.010867426825171\\
-10.960960960961	0.0109394021375814\\
-10.9009009009009	0.0110114569311385\\
-10.8408408408408	0.0110835865139999\\
-10.7807807807808	0.011155786155385\\
-10.7207207207207	0.0112280510860048\\
-10.6606606606607	0.0113003764985012\\
-10.6006006006006	0.0113727575478968\\
-10.5405405405405	0.0114451893520533\\
-10.4804804804805	0.0115176669921402\\
-10.4204204204204	0.0115901855131136\\
-10.3603603603604	0.0116627399242037\\
-10.3003003003003	0.011735325199412\\
-10.2402402402402	0.0118079362780187\\
-10.1801801801802	0.0118805680650987\\
-10.1201201201201	0.0119532154320477\\
-10.0600600600601	0.0120258732171167\\
-10	0.0120985362259572\\
-9.93993993993994	0.0121711992321739\\
-9.87987987987988	0.0122438569778881\\
-9.81981981981982	0.0123165041743092\\
-9.75975975975976	0.0123891355023157\\
-9.6996996996997	0.0124617456130447\\
-9.63963963963964	0.0125343291284909\\
-9.57957957957958	0.0126068806421139\\
-9.51951951951952	0.0126793947194542\\
-9.45945945945946	0.0127518658987581\\
-9.3993993993994	0.0128242886916112\\
-9.33933933933934	0.0128966575835798\\
-9.27927927927928	0.0129689670348614\\
-9.21921921921922	0.0130412114809426\\
-9.15915915915916	0.0131133853332664\\
-9.0990990990991	0.0131854829799063\\
-9.03903903903904	0.013257498786249\\
-8.97897897897898	0.0133294270956849\\
-8.91891891891892	0.0134012622303068\\
-8.85885885885886	0.0134729984916149\\
-8.7987987987988	0.0135446301612312\\
-8.73873873873874	0.0136161515016196\\
-8.67867867867868	0.0136875567568146\\
-8.61861861861862	0.0137588401531565\\
-8.55855855855856	0.0138299959000339\\
-8.4984984984985	0.0139010181906327\\
-8.43843843843844	0.013971901202693\\
-8.37837837837838	0.0140426390992714\\
-8.31831831831832	0.0141132260295108\\
-8.25825825825826	0.0141836561294161\\
-8.1981981981982	0.0142539235226364\\
-8.13813813813814	0.0143240223212534\\
-8.07807807807808	0.0143939466265759\\
-8.01801801801802	0.0144636905299395\\
-7.95795795795796	0.014533248113513\\
-7.8978978978979	0.01460261345111\\
-7.83783783783784	0.0146717806090055\\
-7.77777777777778	0.0147407436467584\\
-7.71771771771772	0.0148094966180392\\
-7.65765765765766	0.0148780335714621\\
-7.5975975975976	0.0149463485514227\\
-7.53753753753754	0.0150144355989405\\
-7.47747747747748	0.0150822887525053\\
-7.41741741741742	0.0151499020489299\\
-7.35735735735736	0.0152172695242059\\
-7.2972972972973	0.0152843852143656\\
-7.23723723723724	0.0153512431563485\\
-7.17717717717718	0.0154178373888737\\
-7.11711711711712	0.0154841619533188\\
-7.05705705705706	0.015550210894606\\
-6.996996996997	0.0156159782620981\\
-6.93693693693694	0.0156814581105062\\
-6.87687687687688	0.0157466445008135\\
-6.81681681681682	0.0158115315012205\\
-6.75675675675676	0.0158761131881198\\
-6.6966966966967	0.0159403836471115\\
-6.63663663663664	0.0160043369740757\\
-6.57657657657658	0.0160679672763242\\
-6.51651651651652	0.0161312686738633\\
-6.45645645645646	0.0161942353008106\\
-6.3963963963964	0.0162568613070271\\
-6.33633633633634	0.0163191408600472\\
-6.27627627627628	0.0163810681474205\\
-6.21621621621622	0.01644263737962\\
-6.15615615615616	0.0165038427937272\\
-6.0960960960961	0.016564678658178\\
-6.03603603603604	0.0166251392789484\\
-5.97597597597598	0.0166852190076917\\
-5.91591591591592	0.016744912252501\\
-5.85585585585586	0.0168042134921941\\
-5.7957957957958	0.0168631172953\\
-5.73573573573574	0.0169216183452914\\
-5.67567567567568	0.0169797114740778\\
-5.61561561561562	0.0170373917063713\\
-5.55555555555556	0.0170946543182964\\
-5.4954954954955	0.0171514949145688\\
-5.43543543543544	0.0172079095297669\\
-5.37537537537538	0.0172638947607091\\
-5.31531531531532	0.0173194479387918\\
-5.25525525525526	0.0173745673534108\\
-5.1951951951952	0.01742925254035\\
-5.13513513513514	0.0174835046523747\\
-5.07507507507508	0.0175373269333\\
-5.01501501501502	0.0175907253216206\\
-4.95495495495495	0.0176437092154977\\
-4.89489489489489	0.0176962924376079\\
-4.83483483483483	0.0177484944461754\\
-4.77477477477477	0.0178003418475304\\
-4.71471471471471	0.0178518702758508\\
-4.65465465465465	0.0179031267173973\\
-4.59459459459459	0.0179541723695834\\
-4.53453453453453	0.0180050861395827\\
-4.47447447447447	0.0180559689027916\\
-4.41441441441441	0.0181069486581469\\
-4.35435435435435	0.0181581867347768\\
-4.29429429429429	0.0182098852223368\\
-4.23423423423423	0.0182622958151149\\
-4.17417417417417	0.0183157302768777\\
-4.11411411411411	0.0183705727485972\\
-4.05405405405405	0.0184272941335678\\
-3.99399399399399	0.0184864688027269\\
-3.93393393393393	0.018548793865743\\
-3.87387387387387	0.0186151112489613\\
-3.81381381381381	0.0186864328077373\\
-3.75375375375375	0.018763968676049\\
-3.69369369369369	0.01884915901845\\
-3.63363363363363	0.0189437092963032\\
-3.57357357357357	0.0190496290897216\\
-3.51351351351351	0.0191692744268705\\
-3.45345345345345	0.0193053934616122\\
-3.39339339339339	0.0194611752077546\\
-3.33333333333333	0.0196403008827952\\
-3.27327327327327	0.0198469972362465\\
-3.21321321321321	0.0200860910384963\\
-3.15315315315315	0.0203630636879553\\
-3.09309309309309	0.0206841046604741\\
-3.03303303303303	0.0210561622805956\\
-2.97297297297297	0.0214869900455463\\
-2.91291291291291	0.0219851864879207\\
-2.85285285285285	0.0225602263312442\\
-2.79279279279279	0.0232224804849215\\
-2.73273273273273	0.0239832222536311\\
-2.67267267267267	0.0248546170141434\\
-2.61261261261261	0.0258496925535233\\
-2.55255255255255	0.0269822872805678\\
-2.49249249249249	0.0282669736304942\\
-2.43243243243243	0.0297189541936005\\
-2.37237237237237	0.0313539284222862\\
-2.31231231231231	0.0331879282152832\\
-2.25225225225225	0.0352371212476485\\
-2.19219219219219	0.0375175816103483\\
-2.13213213213213	0.0400450281396492\\
-2.07207207207207	0.0428345317442067\\
-2.01201201201201	0.0459001940611082\\
-1.95195195195195	0.0492548008696528\\
-1.89189189189189	0.0529094548358687\\
-1.83183183183183	0.056873193318652\\
-1.77177177177177	0.0611525981019613\\
-1.71171171171171	0.0657514049847253\\
-1.65165165165165	0.0706701221162344\\
-1.59159159159159	0.0759056667637859\\
-1.53153153153153	0.0814510307957571\\
-1.47147147147147	0.0872949855140875\\
-1.41141141141141	0.093421836536934\\
-1.35135135135135	0.0998112391832054\\
-1.29129129129129	0.106438084222637\\
-1.23123123123123	0.113272462915342\\
-1.17117117117117	0.120279718972681\\
-1.11111111111111	0.12742059343927\\
-1.05105105105105	0.134651466550188\\
-0.990990990990991	0.141924698398076\\
-0.930930930930931	0.14918906780518\\
-0.870870870870871	0.156390306201099\\
-0.810810810810811	0.163471720634027\\
-0.75075075075075	0.170374897375464\\
-0.69069069069069	0.177040475004946\\
-0.63063063063063	0.183408973472913\\
-0.57057057057057	0.189421663525664\\
-0.51051051051051	0.195021459119906\\
-0.45045045045045	0.200153814130103\\
-0.39039039039039	0.204767603821724\\
-0.33033033033033	0.208815971273696\\
-0.27027027027027	0.212257119212412\\
-0.21021021021021	0.215055028575825\\
-0.15015015015015	0.217180086547454\\
-0.0900900900900901	0.218609608753297\\
-0.03003003003003	0.219328242746816\\
0.03003003003003	0.219328242746816\\
0.0900900900900901	0.218609608753297\\
0.15015015015015	0.217180086547454\\
0.21021021021021	0.215055028575825\\
0.27027027027027	0.212257119212412\\
0.33033033033033	0.208815971273696\\
0.39039039039039	0.204767603821724\\
0.45045045045045	0.200153814130103\\
0.51051051051051	0.195021459119906\\
0.57057057057057	0.189421663525664\\
0.63063063063063	0.183408973472913\\
0.69069069069069	0.177040475004946\\
0.75075075075075	0.170374897375464\\
0.810810810810811	0.163471720634027\\
0.870870870870871	0.156390306201099\\
0.930930930930931	0.14918906780518\\
0.990990990990991	0.141924698398076\\
1.05105105105105	0.134651466550188\\
1.11111111111111	0.12742059343927\\
1.17117117117117	0.120279718972681\\
1.23123123123123	0.113272462915342\\
1.29129129129129	0.106438084222637\\
1.35135135135135	0.0998112391832054\\
1.41141141141141	0.093421836536934\\
1.47147147147147	0.0872949855140875\\
1.53153153153153	0.0814510307957571\\
1.59159159159159	0.0759056667637859\\
1.65165165165165	0.0706701221162344\\
1.71171171171171	0.0657514049847253\\
1.77177177177177	0.0611525981019613\\
1.83183183183183	0.056873193318652\\
1.89189189189189	0.0529094548358687\\
1.95195195195195	0.0492548008696528\\
2.01201201201201	0.045900194061108\\
2.07207207207207	0.0428345317442066\\
2.13213213213213	0.0400450281396491\\
2.19219219219219	0.0375175816103481\\
2.25225225225226	0.0352371212476484\\
2.31231231231232	0.0331879282152831\\
2.37237237237238	0.0313539284222861\\
2.43243243243244	0.0297189541936004\\
2.4924924924925	0.0282669736304941\\
2.55255255255256	0.0269822872805677\\
2.61261261261262	0.0258496925535233\\
2.67267267267268	0.0248546170141434\\
2.73273273273274	0.023983222253631\\
2.7927927927928	0.0232224804849215\\
2.85285285285286	0.0225602263312442\\
2.91291291291292	0.0219851864879206\\
2.97297297297298	0.0214869900455463\\
3.03303303303304	0.0210561622805956\\
3.0930930930931	0.0206841046604741\\
3.15315315315316	0.0203630636879553\\
3.21321321321322	0.0200860910384963\\
3.27327327327328	0.0198469972362464\\
3.33333333333334	0.0196403008827952\\
3.3933933933934	0.0194611752077546\\
3.45345345345346	0.0193053934616122\\
3.51351351351352	0.0191692744268705\\
3.57357357357358	0.0190496290897216\\
3.63363363363364	0.0189437092963032\\
3.6936936936937	0.01884915901845\\
3.75375375375376	0.0187639686760489\\
3.81381381381382	0.0186864328077373\\
3.87387387387388	0.0186151112489613\\
3.93393393393394	0.018548793865743\\
3.993993993994	0.0184864688027269\\
4.05405405405406	0.0184272941335678\\
4.11411411411412	0.0183705727485972\\
4.17417417417418	0.0183157302768777\\
4.23423423423424	0.0182622958151149\\
4.2942942942943	0.0182098852223368\\
4.35435435435436	0.0181581867347768\\
4.41441441441442	0.0181069486581469\\
4.47447447447448	0.0180559689027916\\
4.53453453453454	0.0180050861395827\\
4.5945945945946	0.0179541723695834\\
4.65465465465466	0.0179031267173973\\
4.71471471471472	0.0178518702758508\\
4.77477477477478	0.0178003418475304\\
4.83483483483484	0.0177484944461754\\
4.8948948948949	0.0176962924376079\\
4.95495495495496	0.0176437092154977\\
5.01501501501502	0.0175907253216206\\
5.07507507507508	0.0175373269333\\
5.13513513513514	0.0174835046523747\\
5.1951951951952	0.01742925254035\\
5.25525525525526	0.0173745673534108\\
5.31531531531532	0.0173194479387918\\
5.37537537537538	0.0172638947607091\\
5.43543543543544	0.0172079095297669\\
5.4954954954955	0.0171514949145688\\
5.55555555555556	0.0170946543182964\\
5.61561561561562	0.0170373917063713\\
5.67567567567568	0.0169797114740778\\
5.73573573573574	0.0169216183452914\\
5.7957957957958	0.0168631172953\\
5.85585585585586	0.0168042134921941\\
5.91591591591592	0.016744912252501\\
5.97597597597598	0.0166852190076917\\
6.03603603603604	0.0166251392789484\\
6.0960960960961	0.016564678658178\\
6.15615615615616	0.0165038427937272\\
6.21621621621622	0.01644263737962\\
6.27627627627628	0.0163810681474205\\
6.33633633633634	0.0163191408600472\\
6.3963963963964	0.0162568613070271\\
6.45645645645646	0.0161942353008106\\
6.51651651651652	0.0161312686738633\\
6.57657657657658	0.0160679672763242\\
6.63663663663664	0.0160043369740757\\
6.6966966966967	0.0159403836471115\\
6.75675675675676	0.0158761131881198\\
6.81681681681682	0.0158115315012205\\
6.87687687687688	0.0157466445008135\\
6.93693693693694	0.0156814581105062\\
6.996996996997	0.0156159782620981\\
7.05705705705706	0.015550210894606\\
7.11711711711712	0.0154841619533188\\
7.17717717717718	0.0154178373888737\\
7.23723723723724	0.0153512431563485\\
7.2972972972973	0.0152843852143656\\
7.35735735735736	0.0152172695242059\\
7.41741741741742	0.0151499020489299\\
7.47747747747748	0.0150822887525053\\
7.53753753753754	0.0150144355989405\\
7.5975975975976	0.0149463485514227\\
7.65765765765766	0.0148780335714621\\
7.71771771771772	0.0148094966180392\\
7.77777777777778	0.0147407436467584\\
7.83783783783784	0.0146717806090055\\
7.8978978978979	0.01460261345111\\
7.95795795795796	0.014533248113513\\
8.01801801801802	0.0144636905299395\\
8.07807807807808	0.0143939466265759\\
8.13813813813814	0.0143240223212534\\
8.1981981981982	0.0142539235226364\\
8.25825825825826	0.0141836561294161\\
8.31831831831832	0.0141132260295108\\
8.37837837837838	0.0140426390992714\\
8.43843843843844	0.013971901202693\\
8.4984984984985	0.0139010181906327\\
8.55855855855856	0.0138299959000339\\
8.61861861861862	0.0137588401531565\\
8.67867867867868	0.0136875567568146\\
8.73873873873874	0.0136161515016196\\
8.7987987987988	0.0135446301612312\\
8.85885885885886	0.0134729984916149\\
8.91891891891892	0.0134012622303068\\
8.97897897897898	0.0133294270956849\\
9.03903903903904	0.013257498786249\\
9.0990990990991	0.0131854829799063\\
9.15915915915916	0.0131133853332664\\
9.21921921921922	0.0130412114809426\\
9.27927927927928	0.0129689670348614\\
9.33933933933934	0.0128966575835798\\
9.3993993993994	0.0128242886916112\\
9.45945945945946	0.0127518658987581\\
9.51951951951952	0.0126793947194542\\
9.57957957957958	0.0126068806421139\\
9.63963963963964	0.0125343291284909\\
9.6996996996997	0.0124617456130447\\
9.75975975975976	0.0123891355023157\\
9.81981981981982	0.0123165041743092\\
9.87987987987988	0.0122438569778881\\
9.93993993993994	0.0121711992321739\\
10	0.0120985362259572\\
10.0600600600601	0.0120258732171167\\
10.1201201201201	0.0119532154320477\\
10.1801801801802	0.0118805680650987\\
10.2402402402402	0.0118079362780187\\
10.3003003003003	0.011735325199412\\
10.3603603603604	0.0116627399242037\\
10.4204204204204	0.0115901855131136\\
10.4804804804805	0.0115176669921402\\
10.5405405405405	0.0114451893520533\\
10.6006006006006	0.0113727575478968\\
10.6606606606607	0.0113003764985012\\
10.7207207207207	0.0112280510860048\\
10.7807807807808	0.011155786155385\\
10.8408408408408	0.0110835865139999\\
10.9009009009009	0.0110114569311385\\
10.960960960961	0.0109394021375814\\
11.021021021021	0.010867426825171\\
11.0810810810811	0.0107955356463915\\
11.1411411411411	0.0107237332139589\\
11.2012012012012	0.0106520241004205\\
11.2612612612613	0.0105804128377648\\
11.3213213213213	0.010508903917041\\
11.3813813813814	0.010437501787988\\
11.4414414414414	0.0103662108586743\\
11.5015015015015	0.0102950354951469\\
11.5615615615616	0.0102239800210906\\
11.6216216216216	0.0101530487174971\\
11.6816816816817	0.0100822458223445\\
11.7417417417417	0.0100115755302857\\
11.8018018018018	0.00994104199234798\\
11.8618618618619	0.00987064931564172\\
11.9219219219219	0.00980040156307933\\
11.981981981982	0.00973030275310403\\
12.042042042042	0.0096603568594287\\
12.1021021021021	0.00959056781078441\\
12.1621621621622	0.00952093949067904\\
12.2222222222222	0.00945147573716561\\
12.2822822822823	0.00938218034262059\\
12.3423423423423	0.00931305705353189\\
12.4024024024024	0.00924410957029677\\
12.4624624624625	0.00917534154702946\\
12.5225225225225	0.00910675659137854\\
12.5825825825826	0.00903835826435395\\
12.6426426426426	0.0089701500801638\\
12.7027027027027	0.00890213550606072\\
12.7627627627628	0.0088343179621978\\
12.8228228228228	0.00876670082149414\\
12.8828828828829	0.00869928740950985\\
12.9429429429429	0.00863208100433057\\
13.003003003003	0.00856508483646135\\
13.0630630630631	0.00849830208872991\\
13.1231231231231	0.00843173589619928\\
13.1831831831832	0.00836538934608957\\
13.2432432432432	0.00829926547770908\\
13.3033033033033	0.00823336728239447\\
13.3633633633634	0.00816769770346009\\
13.4234234234234	0.00810225963615623\\
13.4834834834835	0.00803705592763646\\
13.5435435435435	0.0079720893769338\\
13.6036036036036	0.00790736273494572\\
13.6636636636637	0.00784287870442798\\
13.7237237237237	0.00777863993999704\\
13.7837837837838	0.00771464904814124\\
13.8438438438438	0.00765090858724042\\
13.9039039039039	0.00758742106759409\\
13.963963963964	0.00752418895145797\\
14.024024024024	0.00746121465308886\\
14.0840840840841	0.00739850053879777\\
14.1441441441441	0.00733604892701122\\
14.2042042042042	0.00727386208834058\\
14.2642642642643	0.0072119422456595\\
14.3243243243243	0.00715029157418916\\
14.3843843843844	0.00708891220159143\\
14.4444444444444	0.00702780620806973\\
14.5045045045045	0.00696697562647758\\
14.5645645645646	0.00690642244243471\\
14.6246246246246	0.00684614859445061\\
14.6846846846847	0.00678615597405554\\
14.7447447447447	0.00672644642593879\\
14.8048048048048	0.00666702174809411\\
14.8648648648649	0.00660788369197236\\
14.9249249249249	0.00654903396264102\\
14.984984984985	0.00649047421895075\\
15.045045045045	0.00643220607370869\\
15.1051051051051	0.0063742310938585\\
15.1651651651652	0.00631655080066705\\
15.2252252252252	0.00625916666991756\\
15.2852852852853	0.00620208013210925\\
15.3453453453453	0.00614529257266324\\
15.4054054054054	0.00608880533213473\\
15.4654654654655	0.00603261970643122\\
15.5255255255255	0.00597673694703682\\
15.5855855855856	0.00592115826124243\\
15.6456456456456	0.00586588481238169\\
15.7057057057057	0.00581091772007274\\
15.7657657657658	0.00575625806046545\\
15.8258258258258	0.00570190686649423\\
15.8858858858859	0.00564786512813618\\
15.9459459459459	0.00559413379267453\\
16.006006006006	0.00554071376496723\\
16.0660660660661	0.00548760590772068\\
16.1261261261261	0.00543481104176831\\
16.1861861861862	0.00538232994635409\\
16.2462462462462	0.00533016335942073\\
16.3063063063063	0.0052783119779026\\
16.3663663663664	0.00522677645802302\\
16.4264264264264	0.00517555741559619\\
16.4864864864865	0.00512465542633319\\
16.5465465465465	0.00507407102615237\\
16.6066066066066	0.00502380471149376\\
16.6666666666667	0.00497385693963744\\
16.7267267267267	0.00492422812902586\\
16.7867867867868	0.00487491865958987\\
16.8468468468468	0.00482592887307846\\
16.9069069069069	0.004777259073392\\
16.966966966967	0.00472890952691898\\
17.027027027027	0.00468088046287611\\
17.0870870870871	0.00463317207365153\\
17.1471471471471	0.00458578451515132\\
17.2072072072072	0.00453871790714891\\
17.2672672672673	0.00449197233363745\\
17.3273273273273	0.00444554784318498\\
17.3873873873874	0.00439944444929231\\
17.4474474474474	0.00435366213075351\\
17.5075075075075	0.00430820083201885\\
17.5675675675676	0.00426306046356019\\
17.6276276276276	0.00421824090223863\\
17.6876876876877	0.00417374199167434\\
17.7477477477477	0.00412956354261848\\
17.8078078078078	0.00408570533332708\\
17.8678678678679	0.00404216710993684\\
17.9279279279279	0.00399894858684269\\
17.987987987988	0.003956049447077\\
18.048048048048	0.0039134693426904\\
18.1081081081081	0.00387120789513405\\
18.1681681681682	0.00382926469564336\\
18.2282282282282	0.00378763930562291\\
18.2882882882883	0.00374633125703258\\
18.3483483483483	0.00370534005277482\\
18.4084084084084	0.00366466516708284\\
18.4684684684685	0.00362430604590977\\
18.5285285285285	0.00358426210731858\\
18.5885885885886	0.00354453274187276\\
18.6486486486486	0.00350511731302755\\
18.7087087087087	0.00346601515752179\\
18.7687687687688	0.00342722558577016\\
18.8288288288288	0.00338874788225577\\
18.8888888888889	0.00335058130592301\\
18.9489489489489	0.00331272509057067\\
19.009009009009	0.00327517844524496\\
19.0690690690691	0.00323794055463276\\
19.1291291291291	0.00320101057945457\\
19.1891891891892	0.00316438765685751\\
19.2492492492492	0.00312807090080786\\
19.3093093093093	0.00309205940248347\\
19.3693693693694	0.0030563522306656\\
19.4294294294294	0.00302094843213039\\
19.4894894894895	0.00298584703203967\\
19.5495495495495	0.00295104703433119\\
19.6096096096096	0.00291654742210813\\
19.6696696696697	0.00288234715802774\\
19.7297297297297	0.00284844518468919\\
19.7897897897898	0.00281484042502043\\
19.8498498498498	0.00278153178266407\\
19.9099099099099	0.00274851814236212\\
19.96996996997	0.00271579837033963\\
20.03003003003	0.00268337131468706\\
20.0900900900901	0.00265123580574138\\
20.1501501501502	0.00261939065646579\\
20.2102102102102	0.00258783466282805\\
20.2702702702703	0.00255656660417727\\
20.3303303303303	0.00252558524361917\\
20.3903903903904	0.00249488932838975\\
20.4504504504505	0.00246447759022725\\
20.5105105105105	0.00243434874574238\\
20.5705705705706	0.00240450149678674\\
20.6306306306306	0.00237493453081945\\
20.6906906906907	0.00234564652127179\\
20.7507507507508	0.00231663612790994\\
20.8108108108108	0.00228790199719566\\
20.8708708708709	0.00225944276264494\\
20.9309309309309	0.00223125704518446\\
20.990990990991	0.00220334345350599\\
21.0510510510511	0.00217570058441848\\
21.1111111111111	0.00214832702319786\\
21.1711711711712	0.00212122134393464\\
21.2312312312312	0.00209438210987902\\
21.2912912912913	0.0020678078737837\\
21.3513513513514	0.00204149717824415\\
21.4114114114114	0.00201544855603646\\
21.4714714714715	0.00198966053045261\\
21.5315315315315	0.0019641316156332\\
21.5915915915916	0.00193886031689754\\
21.6516516516517	0.00191384513107112\\
21.7117117117117	0.00188908454681037\\
21.7717717717718	0.0018645770449247\\
21.8318318318318	0.00184032109869583\\
21.8918918918919	0.00181631517419426\\
21.951951951952	0.00179255773059297\\
22.012012012012	0.00176904722047827\\
22.0720720720721	0.00174578209015774\\
22.1321321321321	0.0017227607799653\\
22.1921921921922	0.00169998172456328\\
22.2522522522523	0.00167744335324163\\
22.3123123123123	0.00165514409021404\\
22.3723723723724	0.00163308235491111\\
22.4324324324324	0.00161125656227044\\
22.4924924924925	0.00158966512302373\\
22.5525525525526	0.00156830644398073\\
22.6126126126126	0.00154717892831015\\
22.6726726726727	0.0015262809758174\\
22.7327327327327	0.00150561098321922\\
22.7927927927928	0.00148516734441522\\
22.8528528528529	0.00146494845075611\\
22.9129129129129	0.00144495269130888\\
22.972972972973	0.00142517845311876\\
23.033033033033	0.00140562412146789\\
23.0930930930931	0.00138628808013086\\
23.1531531531532	0.001367168711627\\
23.2132132132132	0.00134826439746938\\
23.2732732732733	0.00132957351841064\\
23.3333333333333	0.00131109445468547\\
23.3933933933934	0.00129282558624993\\
23.4534534534535	0.00127476529301739\\
23.5135135135135	0.00125691195509125\\
23.5735735735736	0.00123926395299443\\
23.6336336336336	0.00122181966789546\\
23.6936936936937	0.00120457748183136\\
23.7537537537538	0.00118753577792723\\
23.8138138138138	0.00117069294061252\\
23.8738738738739	0.00115404735583405\\
23.9339339339339	0.00113759741126567\\
23.993993993994	0.00112134149651472\\
24.0540540540541	0.00110527800332518\\
24.1141141141141	0.00108940532577745\\
24.1741741741742	0.00107372186048501\\
24.2342342342342	0.00105822600678767\\
24.2942942942943	0.00104291616694162\\
24.3543543543544	0.00102779074630622\\
24.4144144144144	0.00101284815352748\\
24.4744744744745	0.000998086800718375\\
24.5345345345345	0.000983505103635855\\
24.5945945945946	0.000969101481854654\\
24.6546546546547	0.000954874358937879\\
24.7147147147147	0.000940822162604383\\
24.7747747747748	0.000926943324892946\\
24.8348348348348	0.000913236282323261\\
24.8948948948949	0.000899699476053755\\
24.954954954955	0.000886331352036248\\
25.015015015015	0.000873130361167452\\
25.0750750750751	0.000860094959437356\\
25.1351351351351	0.000847223608074471\\
25.1951951951952	0.000834514773687991\\
25.2552552552553	0.000821966928406852\\
25.3153153153153	0.000809578550015724\\
25.3753753753754	0.000797348122087952\\
25.4354354354354	0.000785274134115452\\
25.4954954954955	0.000773355081635594\\
25.5555555555556	0.000761589466355083\\
25.6156156156156	0.000749975796270856\\
25.6756756756757	0.000738512585788013\\
25.7357357357357	0.000727198355834811\\
25.7957957957958	0.000716031633974735\\
25.8558558558559	0.000705010954515658\\
25.9159159159159	0.00069413485861613\\
25.975975975976	0.000683401894388802\\
26.036036036036	0.000672810617001008\\
26.0960960960961	0.000662359588772542\\
26.1561561561562	0.00065204737927063\\
26.2162162162162	0.000641872565402138\\
26.2762762762763	0.000631833731503031\\
26.3363363363363	0.000621929469425108\\
26.3963963963964	0.000612158378620038\\
26.4564564564565	0.000602519066220718\\
26.5165165165165	0.000593010147119985\\
26.5765765765766	0.000583630244046695\\
26.6366366366366	0.000574377987639202\\
26.6966966966967	0.000565252016516269\\
26.7567567567568	0.000556250977345413\\
26.8168168168168	0.000547373524908743\\
26.8768768768769	0.000538618322166284\\
26.9369369369369	0.000529984040316832\\
26.996996996997	0.000521469358856374\\
27.0570570570571	0.000513072965634072\\
27.1171171171171	0.000504793556905862\\
27.1771771771772	0.000496629837385689\\
27.2372372372372	0.000488580520294397\\
27.2972972972973	0.000480644327406314\\
27.3573573573574	0.000472819989093549\\
27.4174174174174	0.000465106244368028\\
27.4774774774775	0.000457501840921312\\
27.5375375375375	0.000450005535162202\\
27.5975975975976	0.000442616092252179\\
27.6576576576577	0.000435332286138691\\
27.7177177177177	0.000428152899586331\\
27.7777777777778	0.000421076724205918\\
27.8378378378378	0.000414102560481516\\
27.8978978978979	0.000407229217795432\\
27.957957957958	0.00040045551445119\\
28.018018018018	0.000393780277694546\\
28.0780780780781	0.000387202343732544\\
28.1381381381381	0.000380720557750653\\
28.1981981981982	0.00037433377392802\\
28.2582582582583	0.000368040855450846\\
28.3183183183183	0.000361840674523944\\
28.3783783783784	0.000355732112380472\\
28.4384384384384	0.000349714059289901\\
28.4984984984985	0.00034378541456423\\
28.5585585585586	0.000337945086562472\\
28.6186186186186	0.000332191992693455\\
28.6786786786787	0.000326525059416953\\
28.7387387387387	0.000320943222243179\\
28.7987987987988	0.000315445425730669\\
28.8588588588589	0.000310030623482585\\
28.9189189189189	0.000304697778141464\\
28.978978978979	0.000299445861382437\\
29.039039039039	0.000294273853904956\\
29.0990990990991	0.000289180745423046\\
29.1591591591592	0.000284165534654112\\
29.2192192192192	0.000279227229306332\\
29.2792792792793	0.000274364846064665\\
29.3393393393393	0.000269577410575485\\
29.3993993993994	0.000264863957429895\\
29.4594594594595	0.000260223530145722\\
29.5195195195195	0.000255655181148236\\
29.5795795795796	0.00025115797174961\\
29.6396396396396	0.000246730972127154\\
29.6996996996997	0.000242373261300351\\
29.7597597597598	0.000238083927106706\\
29.8198198198198	0.000233862066176456\\
29.8798798798799	0.000229706783906147\\
29.9399399399399	0.000225617194431115\\
30	0.0002215924205969\\
};
\addlegendentry{$p(\theta \given \sigma_{\text{spike}}^2, \sigma_{\text{slab}}^2)$};

\end{axis}
\end{tikzpicture}%
  \caption{The prior distribution over $\theta$ for the spike-and-slab
    prior problem.}
\end{figure}

For the second part, we have from Bayes' theorem that:
\begin{equation*}
  \Pr(f = 1 \given x, \sigma_{\text{spike}}^2, \sigma_{\text{slab}}^2, \omega^2)
  =
  \frac{p(x \given f = 1, \sigma_{\text{slab}}^2, \omega^2)\Pr(f = 1)}
       {\sum_f p(x \given f, \sigma_{\text{spike}}^2, \sigma_{\text{slab}}^2, \omega^2)\Pr(f)}.
\end{equation*}

We must derive the likelihood $p(x \given f, \sigma_{\text{spike}}^2,
\sigma_{\text{slab}}^2)$.  We apply the sum rule:
\begin{equation*}
  p(x \given f, \sigma_{\text{spike}}^2, \sigma_{\text{slab}}^2, \omega^2)
  =
  \int
  p(x \given \theta, \omega^2)
  p(\theta \given f, \sigma_{\text{slab}}^2, \omega^2)
  \intd{\theta}.
\end{equation*}
For $f = 1$, this becomes:
\begin{align*}
  p(x \given f = 1, \sigma_{\text{spike}}^2, \sigma_{\text{slab}}^2, \omega^2)
  &=
  \int
  p(x \given \theta, \omega^2)
  p(\theta \given f = 1, \sigma_{\text{slab}}^2, \omega^2)
  \intd{\theta}
  \\
  &=
  \int
  \mc{N}(x; \theta, \omega^2)
  \mc{N}(\theta; 0, \sigma_{\text{slab}}^2)
  \intd{\theta}
  \\
  &=
  \mc{N}(x; 0, \sigma_{\text{slab}}^2 + \omega^2).
\end{align*}
A similar expression holds for $f = 0$.  The final result is
\begin{equation*}
  \Pr(f = 1 \given x, \sigma_{\text{spike}}^2, \sigma_{\text{slab}}^2, \omega^2)
  =
  \frac{\mc{N}(x; 0, \sigma_{\text{slab}}^2 + \omega^2)}
       {\mc{N}(x; 0, \sigma_{\text{spike}}^2 + \omega^2) + \mc{N}(x; 0, \sigma_{\text{slab}}^2 + \omega^2)}.
\end{equation*}
This value is plotted as a function of $x$ on the range $x \in [-10,
  10]$ below for $\omega^2 = 0.1^2$.  Extreme values of $x$ teach us
the most, because we can conclude with near certainty that the
observation came from the slab.  The observations that would teach us
the least are $x \approx \pm 2.165$, either of which result in a
posterior slab probability of $\nicefrac{1}{2}$ -- the same as our
prior!

\begin{figure}[h]
  \centering
  % This file was created by matlab2tikz.
% Minimal pgfplots version: 1.3
%
\tikzsetnextfilename{problem_6_f_posterior}
\definecolor{mycolor1}{rgb}{0.12157,0.47059,0.70588}%
%
\begin{tikzpicture}

\begin{axis}[%
width=0.95092\figurewidth,
height=\figureheight,
at={(0\figurewidth,0\figureheight)},
scale only axis,
xmin=-10,
xmax=10,
xlabel={$x$},
ymin=0,
ymax=1.2,
axis x line*=bottom,
axis y line*=left,
legend style={at={(0.03,0.97)},anchor=north west,legend cell align=left,align=left,fill=none,draw=none}
]
\addplot [color=mycolor1,solid]
  table[row sep=crcr]{%
-10	1\\
-9.97997997997998	1\\
-9.95995995995996	1\\
-9.93993993993994	1\\
-9.91991991991992	1\\
-9.8998998998999	1\\
-9.87987987987988	1\\
-9.85985985985986	1\\
-9.83983983983984	1\\
-9.81981981981982	1\\
-9.7997997997998	1\\
-9.77977977977978	1\\
-9.75975975975976	1\\
-9.73973973973974	1\\
-9.71971971971972	1\\
-9.6996996996997	1\\
-9.67967967967968	1\\
-9.65965965965966	1\\
-9.63963963963964	1\\
-9.61961961961962	1\\
-9.5995995995996	1\\
-9.57957957957958	1\\
-9.55955955955956	1\\
-9.53953953953954	1\\
-9.51951951951952	1\\
-9.4994994994995	1\\
-9.47947947947948	1\\
-9.45945945945946	1\\
-9.43943943943944	1\\
-9.41941941941942	1\\
-9.3993993993994	1\\
-9.37937937937938	1\\
-9.35935935935936	1\\
-9.33933933933934	1\\
-9.31931931931932	1\\
-9.2992992992993	1\\
-9.27927927927928	1\\
-9.25925925925926	1\\
-9.23923923923924	1\\
-9.21921921921922	1\\
-9.1991991991992	1\\
-9.17917917917918	1\\
-9.15915915915916	1\\
-9.13913913913914	1\\
-9.11911911911912	1\\
-9.0990990990991	1\\
-9.07907907907908	1\\
-9.05905905905906	1\\
-9.03903903903904	1\\
-9.01901901901902	1\\
-8.998998998999	1\\
-8.97897897897898	1\\
-8.95895895895896	1\\
-8.93893893893894	1\\
-8.91891891891892	1\\
-8.8988988988989	1\\
-8.87887887887888	1\\
-8.85885885885886	1\\
-8.83883883883884	1\\
-8.81881881881882	1\\
-8.7987987987988	1\\
-8.77877877877878	1\\
-8.75875875875876	0.999999999999999\\
-8.73873873873874	0.999999999999999\\
-8.71871871871872	0.999999999999999\\
-8.6986986986987	0.999999999999999\\
-8.67867867867868	0.999999999999999\\
-8.65865865865866	0.999999999999999\\
-8.63863863863864	0.999999999999999\\
-8.61861861861862	0.999999999999998\\
-8.5985985985986	0.999999999999998\\
-8.57857857857858	0.999999999999998\\
-8.55855855855856	0.999999999999997\\
-8.53853853853854	0.999999999999997\\
-8.51851851851852	0.999999999999996\\
-8.4984984984985	0.999999999999996\\
-8.47847847847848	0.999999999999995\\
-8.45845845845846	0.999999999999994\\
-8.43843843843844	0.999999999999993\\
-8.41841841841842	0.999999999999992\\
-8.3983983983984	0.99999999999999\\
-8.37837837837838	0.999999999999989\\
-8.35835835835836	0.999999999999987\\
-8.33833833833834	0.999999999999984\\
-8.31831831831832	0.999999999999981\\
-8.2982982982983	0.999999999999978\\
-8.27827827827828	0.999999999999974\\
-8.25825825825826	0.99999999999997\\
-8.23823823823824	0.999999999999964\\
-8.21821821821822	0.999999999999958\\
-8.1981981981982	0.999999999999951\\
-8.17817817817818	0.999999999999942\\
-8.15815815815816	0.999999999999932\\
-8.13813813813814	0.99999999999992\\
-8.11811811811812	0.999999999999906\\
-8.0980980980981	0.99999999999989\\
-8.07807807807808	0.999999999999871\\
-8.05805805805806	0.999999999999849\\
-8.03803803803804	0.999999999999823\\
-8.01801801801802	0.999999999999793\\
-7.997997997998	0.999999999999758\\
-7.97797797797798	0.999999999999717\\
-7.95795795795796	0.999999999999669\\
-7.93793793793794	0.999999999999613\\
-7.91791791791792	0.999999999999548\\
-7.8978978978979	0.999999999999472\\
-7.87787787787788	0.999999999999384\\
-7.85785785785786	0.999999999999281\\
-7.83783783783784	0.999999999999161\\
-7.81781781781782	0.999999999999022\\
-7.7977977977978	0.99999999999886\\
-7.77777777777778	0.999999999998672\\
-7.75775775775776	0.999999999998453\\
-7.73773773773774	0.999999999998199\\
-7.71771771771772	0.999999999997905\\
-7.6976976976977	0.999999999997563\\
-7.67767767767768	0.999999999997166\\
-7.65765765765766	0.999999999996705\\
-7.63763763763764	0.999999999996172\\
-7.61761761761762	0.999999999995554\\
-7.5975975975976	0.999999999994838\\
-7.57757757757758	0.99999999999401\\
-7.55755755755756	0.999999999993051\\
-7.53753753753754	0.999999999991941\\
-7.51751751751752	0.999999999990659\\
-7.4974974974975	0.999999999989176\\
-7.47747747747748	0.999999999987463\\
-7.45745745745746	0.999999999985485\\
-7.43743743743744	0.999999999983201\\
-7.41741741741742	0.999999999980565\\
-7.3973973973974	0.999999999977525\\
-7.37737737737738	0.999999999974019\\
-7.35735735735736	0.999999999969978\\
-7.33733733733734	0.999999999965322\\
-7.31731731731732	0.99999999995996\\
-7.2972972972973	0.999999999953788\\
-7.27727727727728	0.999999999946684\\
-7.25725725725726	0.999999999938513\\
-7.23723723723724	0.999999999929117\\
-7.21721721721722	0.999999999918317\\
-7.1971971971972	0.999999999905909\\
-7.17717717717718	0.999999999891659\\
-7.15715715715716	0.999999999875299\\
-7.13713713713714	0.999999999856525\\
-7.11711711711712	0.99999999983499\\
-7.0970970970971	0.999999999810298\\
-7.07707707707708	0.999999999781995\\
-7.05705705705706	0.999999999749569\\
-7.03703703703704	0.999999999712432\\
-7.01701701701702	0.999999999669918\\
-6.996996996997	0.999999999621268\\
-6.97697697697698	0.999999999565618\\
-6.95695695695696	0.999999999501986\\
-6.93693693693694	0.999999999429257\\
-6.91691691691692	0.999999999346164\\
-6.8968968968969	0.999999999251268\\
-6.87687687687688	0.999999999142936\\
-6.85685685685686	0.999999999019314\\
-6.83683683683684	0.999999998878303\\
-6.81681681681682	0.999999998717519\\
-6.7967967967968	0.999999998534265\\
-6.77677677677678	0.999999998325483\\
-6.75675675675676	0.999999998087714\\
-6.73673673673674	0.99999999781704\\
-6.71671671671672	0.999999997509033\\
-6.6966966966967	0.999999997158684\\
-6.67667667667668	0.999999996760331\\
-6.65665665665666	0.999999996307581\\
-6.63663663663664	0.99999999579321\\
-6.61661661661662	0.999999995209068\\
-6.5965965965966	0.999999994545956\\
-6.57657657657658	0.999999993793502\\
-6.55655655655656	0.999999992940011\\
-6.53653653653654	0.999999991972305\\
-6.51651651651652	0.999999990875543\\
-6.4964964964965	0.999999989633011\\
-6.47647647647648	0.999999988225902\\
-6.45645645645646	0.999999986633059\\
-6.43643643643644	0.99999998483069\\
-6.41641641641642	0.999999982792055\\
-6.3963963963964	0.99999998048711\\
-6.37637637637638	0.999999977882115\\
-6.35635635635636	0.999999974939197\\
-6.33633633633634	0.999999971615859\\
-6.31631631631632	0.999999967864435\\
-6.2962962962963	0.999999963631489\\
-6.27627627627628	0.999999958857139\\
-6.25625625625626	0.999999953474305\\
-6.23623623623624	0.999999947407884\\
-6.21621621621622	0.99999994057382\\
-6.1961961961962	0.999999932878079\\
-6.17617617617618	0.999999924215508\\
-6.15615615615616	0.999999914468575\\
-6.13613613613614	0.999999903505964\\
-6.11611611611612	0.999999891181026\\
-6.0960960960961	0.999999877330051\\
-6.07607607607608	0.999999861770372\\
-6.05605605605606	0.999999844298252\\
-6.03603603603604	0.999999824686549\\
-6.01601601601602	0.999999802682146\\
-5.995995995996	0.999999778003092\\
-5.97597597597598	0.999999750335459\\
-5.95595595595596	0.999999719329867\\
-5.93593593593594	0.999999684597644\\
-5.91591591591592	0.9999996457066\\
-5.8958958958959	0.99999960217636\\
-5.87587587587588	0.999999553473223\\
-5.85585585585586	0.999999499004501\\
-5.83583583583584	0.999999438112287\\
-5.81581581581582	0.99999937006659\\
-5.7957957957958	0.9999992940578\\
-5.77577577577578	0.999999209188384\\
-5.75575575575576	0.999999114463779\\
-5.73573573573574	0.999999008782381\\
-5.71571571571572	0.999998890924558\\
-5.6956956956957	0.999998759540591\\
-5.67567567567568	0.999998613137461\\
-5.65565565565566	0.999998450064347\\
-5.63563563563564	0.999998268496757\\
-5.61561561561562	0.999998066419135\\
-5.5955955955956	0.999997841605836\\
-5.57557557557558	0.999997591600303\\
-5.55555555555556	0.999997313692308\\
-5.53553553553554	0.999997004893069\\
-5.51551551551552	0.999996661908076\\
-5.4954954954955	0.999996281107423\\
-5.47547547547548	0.999995858493425\\
-5.45545545545546	0.999995389665306\\
-5.43543543543544	0.999994869780697\\
-5.41541541541542	0.99999429351369\\
-5.3953953953954	0.999993655009149\\
-5.37537537537538	0.999992947832987\\
-5.35535535535536	0.999992164918055\\
-5.33533533533534	0.999991298505322\\
-5.31531531531532	0.999990340079927\\
-5.2952952952953	0.999989280301731\\
-5.27527527527528	0.999988108929903\\
-5.25525525525526	0.999986814741094\\
-5.23523523523524	0.999985385440685\\
-5.21521521521522	0.999983807566575\\
-5.1951951951952	0.999982066384951\\
-5.17517517517518	0.999980145777409\\
-5.15515515515516	0.999978028118784\\
-5.13513513513514	0.999975694145007\\
-5.11511511511512	0.999973122810226\\
-5.0950950950951	0.999970291132417\\
-5.07507507507508	0.999967174026655\\
-5.05505505505506	0.999963744125133\\
-5.03503503503504	0.999959971583007\\
-5.01501501501502	0.999955823869049\\
-4.99499499499499	0.999951265540051\\
-4.97497497497497	0.999946257997851\\
-4.95495495495495	0.999940759227796\\
-4.93493493493493	0.999934723517363\\
-4.91491491491491	0.999928101153637\\
-4.89489489489489	0.999920838098211\\
-4.87487487487487	0.99991287563804\\
-4.85485485485485	0.999904150010694\\
-4.83483483483483	0.999894592002349\\
-4.81481481481481	0.999884126516818\\
-4.79479479479479	0.999872672113778\\
-4.77477477477477	0.999860140514314\\
-4.75475475475475	0.999846436071793\\
-4.73473473473473	0.999831455205961\\
-4.71471471471471	0.999815085798117\\
-4.69469469469469	0.999797206545074\\
-4.67467467467467	0.999777686269554\\
-4.65465465465465	0.999756383184559\\
-4.63463463463463	0.99973314410915\\
-4.61461461461461	0.999707803633017\\
-4.59459459459459	0.999680183227064\\
-4.57457457457457	0.999650090297224\\
-4.55455455455455	0.999617317178551\\
-4.53453453453453	0.999581640066626\\
-4.51451451451451	0.999542817883188\\
-4.49449449449449	0.99950059107286\\
-4.47447447447447	0.999454680327758\\
-4.45445445445445	0.999404785236749\\
-4.43443443443443	0.999350582856043\\
-4.41441441441441	0.999291726197815\\
-4.39439439439439	0.999227842633511\\
-4.37437437437437	0.999158532208531\\
-4.35435435435435	0.999083365864962\\
-4.33433433433433	0.99900188356913\\
-4.31431431431431	0.998913592340784\\
-4.29429429429429	0.998817964180835\\
-4.27427427427427	0.998714433894716\\
-4.25425425425425	0.998602396808598\\
-4.23423423423423	0.998481206375909\\
-4.21421421421421	0.998350171671868\\
-4.19419419419419	0.99820855477404\\
-4.17417417417417	0.998055568027324\\
-4.15415415415415	0.997890371192149\\
-4.13413413413413	0.997712068475223\\
-4.11411411411411	0.997519705442681\\
-4.09409409409409	0.997312265816174\\
-4.07407407407407	0.997088668153164\\
-4.05405405405405	0.996847762413516\\
-4.03403403403403	0.996588326415426\\
-4.01401401401401	0.996309062184774\\
-3.99399399399399	0.996008592203152\\
-3.97397397397397	0.995685455561123\\
-3.95395395395395	0.995338104024695\\
-3.93393393393393	0.994964898024588\\
-3.91391391391391	0.994564102579591\\
-3.89389389389389	0.994133883167212\\
-3.87387387387387	0.993672301556906\\
-3.85385385385385	0.993177311623375\\
-3.83383383383383	0.992646755159901\\
-3.81381381381381	0.992078357714258\\
-3.79379379379379	0.991469724472571\\
-3.77377377377377	0.990818336219458\\
-3.75375375375375	0.990121545406018\\
-3.73373373373373	0.989376572360539\\
-3.71371371371371	0.9885805016804\\
-3.69369369369369	0.987730278847286\\
-3.67367367367367	0.986822707111759\\
-3.65365365365365	0.985854444697138\\
-3.63363363363363	0.984822002376762\\
-3.61361361361361	0.983721741482839\\
-3.59359359359359	0.982549872409209\\
-3.57357357357357	0.981302453674533\\
-3.55355355355355	0.979975391616384\\
-3.53353353353353	0.978564440790643\\
-3.51351351351351	0.977065205154219\\
-3.49349349349349	0.975473140112392\\
-3.47347347347347	0.973783555514992\\
-3.45345345345345	0.971991619687922\\
-3.43343343343343	0.970092364588214\\
-3.41341341341341	0.968080692171661\\
-3.39339339339339	0.965951382061994\\
-3.37337337337337	0.963699100609413\\
-3.35335335335335	0.961318411423826\\
-3.33333333333333	0.958803787464359\\
-3.31331331331331	0.956149624761253\\
-3.29329329329329	0.953350257839142\\
-3.27327327327327	0.950399976901651\\
-3.25325325325325	0.947293046826211\\
-3.23323323323323	0.944023728004737\\
-3.21321321321321	0.940586299050369\\
-3.19319319319319	0.936975081372639\\
-3.17317317317317	0.933184465603295\\
-3.15315315315315	0.929208939832502\\
-3.13313313313313	0.925043119590348\\
-3.11311311311311	0.920681779481688\\
-3.09309309309309	0.91611988635347\\
-3.07307307307307	0.911352633843142\\
-3.05305305305305	0.906375478124893\\
-3.03303303303303	0.901184174637665\\
-3.01301301301301	0.895774815545793\\
-2.99299299299299	0.890143867650157\\
-2.97297297297297	0.884288210435716\\
-2.95295295295295	0.878205173910904\\
-2.93293293293293	0.871892575866358\\
-2.91291291291291	0.865348758155779\\
-2.89289289289289	0.858572621581103\\
-2.87287287287287	0.851563658948564\\
-2.85285285285285	0.844321985852406\\
-2.83283283283283	0.836848368739754\\
-2.81281281281281	0.829144249814173\\
-2.79279279279279	0.821211768347228\\
-2.77277277277277	0.813053777987426\\
-2.75275275275275	0.804673859684386\\
-2.73273273273273	0.796076329883076\\
-2.71271271271271	0.787266243688244\\
-2.69269269269269	0.778249392752288\\
-2.67267267267267	0.769032297700219\\
-2.65265265265265	0.759622194971987\\
-2.63263263263263	0.750027018034326\\
-2.61261261261261	0.740255372989926\\
-2.59259259259259	0.730316508689794\\
-2.57257257257257	0.720220281533386\\
-2.55255255255255	0.709977115218841\\
-2.53253253253253	0.699597955780562\\
-2.51251251251251	0.689094222321907\\
-2.49249249249249	0.678477753915005\\
-2.47247247247247	0.667760753196416\\
-2.45245245245245	0.65695572723497\\
-2.43243243243243	0.646075426285715\\
-2.41241241241241	0.635132781070568\\
-2.39239239239239	0.624140839241452\\
-2.37237237237237	0.613112701685273\\
-2.35235235235235	0.602061459322061\\
-2.33233233233233	0.591000131028388\\
-2.31231231231231	0.579941603288513\\
-2.29229229229229	0.56889857213646\\
-2.27227227227227	0.55788348790461\\
-2.25225225225225	0.546908503239727\\
-2.23223223223223	0.535985424787068\\
-2.21221221221221	0.525125668878843\\
-2.19219219219219	0.514340221496483\\
-2.17217217217217	0.503639602708242\\
-2.15215215215215	0.493033835716301\\
-2.13213213213213	0.482532420581879\\
-2.11211211211211	0.472144312634275\\
-2.09209209209209	0.461877905511172\\
-2.07207207207207	0.451741018723849\\
-2.05205205205205	0.441740889592846\\
-2.03203203203203	0.431884169357536\\
-2.01201201201201	0.422176923227332\\
-1.99199199199199	0.412624634112964\\
-1.97197197197197	0.40323220975337\\
-1.95195195195195	0.3940039929371\\
-1.93193193193193	0.384943774506386\\
-1.91191191191191	0.376054808826832\\
-1.89189189189189	0.367339831405446\\
-1.87187187187187	0.358801078344135\\
-1.85185185185185	0.35044030732399\\
-1.83183183183183	0.342258819827341\\
-1.81181181181181	0.334257484318986\\
-1.79179179179179	0.326436760124575\\
-1.77177177177177	0.318796721762496\\
-1.75175175175175	0.311337083505041\\
-1.73173173173173	0.304057223964856\\
-1.71171171171171	0.296956210523209\\
-1.69169169169169	0.290032823437051\\
-1.67167167167167	0.283285579481956\\
-1.65165165165165	0.276712755007545\\
-1.63163163163163	0.27031240830061\\
-1.61161161161161	0.264082401168856\\
-1.59159159159159	0.258020419674684\\
-1.57157157157157	0.252123993963801\\
-1.55155155155155	0.246390517147489\\
-1.53153153153153	0.240817263210139\\
-1.51151151151151	0.235401403925192\\
-1.49149149149149	0.230140024772821\\
-1.47147147147147	0.225030139861723\\
-1.45145145145145	0.220068705865242\\
-1.43143143143143	0.215252634988741\\
-1.41141141141141	0.210578806990903\\
-1.39139139139139	0.206044080286311\\
-1.37137137137137	0.201645302160577\\
-1.35135135135135	0.197379318132307\\
-1.33133133133133	0.193242980498536\\
-1.31131131131131	0.18923315610195\\
-1.29129129129129	0.185346733359327\\
-1.27127127127127	0.18158062859123\\
-1.25125125125125	0.177931791693154\\
-1.23123123123123	0.174397211188144\\
-1.21121121121121	0.170973918700338\\
-1.19119119119119	0.167658992888148\\
-1.17117117117117	0.164449562874725\\
-1.15115115115115	0.161342811212221\\
-1.13113113113113	0.158335976414973\\
-1.11111111111111	0.155426355095361\\
-1.09109109109109	0.152611303734525\\
-1.07107107107107	0.149888240118598\\
-1.05105105105105	0.147254644469523\\
-1.03103103103103	0.144708060297892\\
-1.01101101101101	0.142246095003688\\
-0.990990990990991	0.139866420249228\\
-0.970970970970971	0.137566772127051\\
-0.950950950950951	0.135344951144024\\
-0.930930930930931	0.133198822041469\\
-0.910910910910911	0.131126313469714\\
-0.890890890890891	0.129125417534173\\
-0.870870870870871	0.127194189228717\\
-0.850850850850851	0.125330745770948\\
-0.830830830830831	0.123533265852809\\
-0.810810810810811	0.121799988818863\\
-0.790790790790791	0.120129213783587\\
-0.77077077077077	0.118519298698039\\
-0.75075075075075	0.116968659375373\\
-0.73073073073073	0.115475768483821\\
-0.71071071071071	0.114039154515008\\
-0.69069069069069	0.112657400734703\\
-0.67067067067067	0.111329144122468\\
-0.65065065065065	0.110053074306022\\
-0.63063063063063	0.108827932495573\\
-0.61061061061061	0.107652510422836\\
-0.59059059059059	0.106525649288962\\
-0.57057057057057	0.105446238725174\\
-0.55055055055055	0.104413215769473\\
-0.53053053053053	0.103425563862424\\
-0.51051051051051	0.102482311864667\\
-0.49049049049049	0.101582533098522\\
-0.47047047047047	0.100725344415737\\
-0.45045045045045	0.0999099052932061\\
-0.43043043043043	0.0991354169582323\\
-0.41041041041041	0.0984011215447149\\
-0.39039039039039	0.0977063012814533\\
-0.37037037037037	0.0970502777135889\\
-0.35035035035035	0.096432410958064\\
-0.33033033033033	0.095852098993841\\
-0.31031031031031	0.0953087769875114\\
-0.29029029029029	0.0948019166548199\\
-0.27027027027027	0.0943310256585398\\
-0.25025025025025	0.0938956470430591\\
-0.23023023023023	0.0934953587059653\\
-0.21021021021021	0.0931297729068643\\
-0.19019019019019	0.0927985358136146\\
-0.17017017017017	0.0925013270861186\\
-0.15015015015015	0.092237859497779\\
-0.13013013013013	0.092007878594698\\
-0.11011011011011	0.0918111623926754\\
-0.0900900900900901	0.0916475211120448\\
-0.07007007007007	0.0915167969503696\\
-0.05005005005005	0.0914188638930166\\
-0.03003003003003	0.0913536275616115\\
-0.01001001001001	0.0913210251003809\\
0.01001001001001	0.0913210251003809\\
0.03003003003003	0.0913536275616115\\
0.05005005005005	0.0914188638930166\\
0.07007007007007	0.0915167969503696\\
0.0900900900900901	0.0916475211120448\\
0.11011011011011	0.0918111623926754\\
0.13013013013013	0.092007878594698\\
0.15015015015015	0.092237859497779\\
0.17017017017017	0.0925013270861186\\
0.19019019019019	0.0927985358136146\\
0.21021021021021	0.0931297729068643\\
0.23023023023023	0.0934953587059653\\
0.25025025025025	0.0938956470430591\\
0.27027027027027	0.0943310256585398\\
0.29029029029029	0.0948019166548199\\
0.31031031031031	0.0953087769875114\\
0.33033033033033	0.095852098993841\\
0.35035035035035	0.096432410958064\\
0.37037037037037	0.0970502777135889\\
0.39039039039039	0.0977063012814533\\
0.41041041041041	0.0984011215447149\\
0.43043043043043	0.0991354169582323\\
0.45045045045045	0.0999099052932061\\
0.47047047047047	0.100725344415737\\
0.49049049049049	0.101582533098522\\
0.51051051051051	0.102482311864667\\
0.53053053053053	0.103425563862424\\
0.55055055055055	0.104413215769473\\
0.57057057057057	0.105446238725174\\
0.59059059059059	0.106525649288962\\
0.61061061061061	0.107652510422836\\
0.63063063063063	0.108827932495573\\
0.65065065065065	0.110053074306022\\
0.67067067067067	0.111329144122468\\
0.69069069069069	0.112657400734703\\
0.71071071071071	0.114039154515008\\
0.73073073073073	0.115475768483821\\
0.75075075075075	0.116968659375373\\
0.77077077077077	0.118519298698039\\
0.790790790790791	0.120129213783587\\
0.810810810810811	0.121799988818863\\
0.830830830830831	0.123533265852809\\
0.850850850850851	0.125330745770948\\
0.870870870870871	0.127194189228717\\
0.890890890890891	0.129125417534173\\
0.910910910910911	0.131126313469714\\
0.930930930930931	0.133198822041469\\
0.950950950950951	0.135344951144024\\
0.970970970970971	0.137566772127051\\
0.990990990990991	0.139866420249228\\
1.01101101101101	0.142246095003688\\
1.03103103103103	0.144708060297892\\
1.05105105105105	0.147254644469523\\
1.07107107107107	0.149888240118598\\
1.09109109109109	0.152611303734525\\
1.11111111111111	0.155426355095361\\
1.13113113113113	0.158335976414973\\
1.15115115115115	0.161342811212221\\
1.17117117117117	0.164449562874725\\
1.19119119119119	0.167658992888148\\
1.21121121121121	0.170973918700338\\
1.23123123123123	0.174397211188144\\
1.25125125125125	0.177931791693154\\
1.27127127127127	0.18158062859123\\
1.29129129129129	0.185346733359327\\
1.31131131131131	0.18923315610195\\
1.33133133133133	0.193242980498536\\
1.35135135135135	0.197379318132307\\
1.37137137137137	0.201645302160577\\
1.39139139139139	0.206044080286311\\
1.41141141141141	0.210578806990903\\
1.43143143143143	0.215252634988741\\
1.45145145145145	0.220068705865242\\
1.47147147147147	0.225030139861723\\
1.49149149149149	0.230140024772821\\
1.51151151151151	0.235401403925192\\
1.53153153153153	0.240817263210139\\
1.55155155155155	0.246390517147489\\
1.57157157157157	0.252123993963801\\
1.59159159159159	0.258020419674684\\
1.61161161161161	0.264082401168856\\
1.63163163163163	0.27031240830061\\
1.65165165165165	0.276712755007545\\
1.67167167167167	0.283285579481956\\
1.69169169169169	0.290032823437051\\
1.71171171171171	0.296956210523209\\
1.73173173173173	0.304057223964856\\
1.75175175175175	0.311337083505041\\
1.77177177177177	0.318796721762496\\
1.79179179179179	0.326436760124575\\
1.81181181181181	0.334257484318986\\
1.83183183183183	0.342258819827341\\
1.85185185185185	0.35044030732399\\
1.87187187187187	0.358801078344135\\
1.89189189189189	0.367339831405446\\
1.91191191191191	0.376054808826832\\
1.93193193193193	0.384943774506386\\
1.95195195195195	0.3940039929371\\
1.97197197197197	0.40323220975337\\
1.99199199199199	0.412624634112964\\
2.01201201201201	0.422176923227332\\
2.03203203203203	0.431884169357535\\
2.05205205205205	0.441740889592846\\
2.07207207207207	0.451741018723849\\
2.09209209209209	0.461877905511171\\
2.11211211211211	0.472144312634274\\
2.13213213213213	0.482532420581878\\
2.15215215215215	0.493033835716301\\
2.17217217217217	0.503639602708242\\
2.19219219219219	0.514340221496482\\
2.21221221221221	0.525125668878843\\
2.23223223223223	0.535985424787067\\
2.25225225225225	0.546908503239727\\
2.27227227227227	0.557883487904609\\
2.29229229229229	0.56889857213646\\
2.31231231231231	0.579941603288513\\
2.33233233233233	0.591000131028387\\
2.35235235235235	0.602061459322061\\
2.37237237237237	0.613112701685273\\
2.39239239239239	0.624140839241451\\
2.41241241241241	0.635132781070568\\
2.43243243243243	0.646075426285715\\
2.45245245245245	0.65695572723497\\
2.47247247247247	0.667760753196416\\
2.49249249249249	0.678477753915004\\
2.51251251251251	0.689094222321907\\
2.53253253253253	0.699597955780563\\
2.55255255255255	0.709977115218841\\
2.57257257257257	0.720220281533387\\
2.59259259259259	0.730316508689794\\
2.61261261261261	0.740255372989927\\
2.63263263263263	0.750027018034327\\
2.65265265265265	0.759622194971987\\
2.67267267267267	0.769032297700219\\
2.69269269269269	0.778249392752289\\
2.71271271271271	0.787266243688244\\
2.73273273273273	0.796076329883076\\
2.75275275275275	0.804673859684386\\
2.77277277277277	0.813053777987427\\
2.79279279279279	0.821211768347229\\
2.81281281281281	0.829144249814173\\
2.83283283283283	0.836848368739754\\
2.85285285285285	0.844321985852406\\
2.87287287287287	0.851563658948565\\
2.89289289289289	0.858572621581103\\
2.91291291291291	0.865348758155779\\
2.93293293293293	0.871892575866358\\
2.95295295295295	0.878205173910904\\
2.97297297297297	0.884288210435716\\
2.99299299299299	0.890143867650157\\
3.01301301301301	0.895774815545793\\
3.03303303303303	0.901184174637665\\
3.05305305305305	0.906375478124893\\
3.07307307307307	0.911352633843142\\
3.09309309309309	0.91611988635347\\
3.11311311311311	0.920681779481689\\
3.13313313313313	0.925043119590349\\
3.15315315315315	0.929208939832502\\
3.17317317317317	0.933184465603295\\
3.19319319319319	0.936975081372639\\
3.21321321321321	0.94058629905037\\
3.23323323323323	0.944023728004737\\
3.25325325325325	0.947293046826211\\
3.27327327327327	0.950399976901651\\
3.29329329329329	0.953350257839142\\
3.31331331331331	0.956149624761254\\
3.33333333333333	0.958803787464359\\
3.35335335335335	0.961318411423826\\
3.37337337337337	0.963699100609413\\
3.39339339339339	0.965951382061994\\
3.41341341341341	0.968080692171661\\
3.43343343343343	0.970092364588214\\
3.45345345345345	0.971991619687922\\
3.47347347347347	0.973783555514992\\
3.49349349349349	0.975473140112392\\
3.51351351351351	0.97706520515422\\
3.53353353353353	0.978564440790644\\
3.55355355355355	0.979975391616384\\
3.57357357357357	0.981302453674533\\
3.59359359359359	0.982549872409209\\
3.61361361361361	0.983721741482839\\
3.63363363363363	0.984822002376762\\
3.65365365365365	0.985854444697138\\
3.67367367367367	0.986822707111759\\
3.69369369369369	0.987730278847286\\
3.71371371371371	0.9885805016804\\
3.73373373373373	0.989376572360539\\
3.75375375375375	0.990121545406018\\
3.77377377377377	0.990818336219458\\
3.79379379379379	0.991469724472571\\
3.81381381381381	0.992078357714258\\
3.83383383383383	0.992646755159901\\
3.85385385385385	0.993177311623375\\
3.87387387387387	0.993672301556906\\
3.89389389389389	0.994133883167212\\
3.91391391391391	0.994564102579591\\
3.93393393393393	0.994964898024588\\
3.95395395395395	0.995338104024695\\
3.97397397397397	0.995685455561123\\
3.99399399399399	0.996008592203152\\
4.01401401401401	0.996309062184774\\
4.03403403403403	0.996588326415426\\
4.05405405405405	0.996847762413516\\
4.07407407407407	0.997088668153164\\
4.09409409409409	0.997312265816174\\
4.11411411411411	0.997519705442681\\
4.13413413413413	0.997712068475223\\
4.15415415415415	0.997890371192149\\
4.17417417417417	0.998055568027324\\
4.19419419419419	0.99820855477404\\
4.21421421421421	0.998350171671868\\
4.23423423423423	0.998481206375909\\
4.25425425425425	0.998602396808598\\
4.27427427427427	0.998714433894716\\
4.29429429429429	0.998817964180835\\
4.31431431431431	0.998913592340784\\
4.33433433433433	0.99900188356913\\
4.35435435435435	0.999083365864962\\
4.37437437437437	0.999158532208531\\
4.39439439439439	0.999227842633511\\
4.41441441441441	0.999291726197815\\
4.43443443443443	0.999350582856043\\
4.45445445445445	0.999404785236749\\
4.47447447447447	0.999454680327758\\
4.49449449449449	0.99950059107286\\
4.51451451451451	0.999542817883188\\
4.53453453453453	0.999581640066626\\
4.55455455455455	0.999617317178551\\
4.57457457457457	0.999650090297224\\
4.59459459459459	0.999680183227064\\
4.61461461461461	0.999707803633017\\
4.63463463463463	0.99973314410915\\
4.65465465465465	0.999756383184559\\
4.67467467467467	0.999777686269554\\
4.69469469469469	0.999797206545074\\
4.71471471471471	0.999815085798117\\
4.73473473473473	0.999831455205961\\
4.75475475475475	0.999846436071793\\
4.77477477477477	0.999860140514314\\
4.79479479479479	0.999872672113778\\
4.81481481481481	0.999884126516818\\
4.83483483483483	0.999894592002349\\
4.85485485485485	0.999904150010694\\
4.87487487487487	0.99991287563804\\
4.89489489489489	0.999920838098211\\
4.91491491491491	0.999928101153637\\
4.93493493493493	0.999934723517363\\
4.95495495495495	0.999940759227796\\
4.97497497497497	0.999946257997851\\
4.99499499499499	0.999951265540051\\
5.01501501501502	0.999955823869049\\
5.03503503503504	0.999959971583007\\
5.05505505505506	0.999963744125133\\
5.07507507507508	0.999967174026655\\
5.0950950950951	0.999970291132417\\
5.11511511511512	0.999973122810226\\
5.13513513513514	0.999975694145007\\
5.15515515515516	0.999978028118784\\
5.17517517517518	0.999980145777409\\
5.1951951951952	0.999982066384951\\
5.21521521521522	0.999983807566575\\
5.23523523523524	0.999985385440685\\
5.25525525525526	0.999986814741094\\
5.27527527527528	0.999988108929903\\
5.2952952952953	0.999989280301731\\
5.31531531531532	0.999990340079927\\
5.33533533533534	0.999991298505322\\
5.35535535535536	0.999992164918055\\
5.37537537537538	0.999992947832987\\
5.3953953953954	0.999993655009149\\
5.41541541541542	0.99999429351369\\
5.43543543543544	0.999994869780697\\
5.45545545545546	0.999995389665306\\
5.47547547547548	0.999995858493425\\
5.4954954954955	0.999996281107423\\
5.51551551551552	0.999996661908076\\
5.53553553553554	0.999997004893069\\
5.55555555555556	0.999997313692308\\
5.57557557557558	0.999997591600303\\
5.5955955955956	0.999997841605836\\
5.61561561561562	0.999998066419135\\
5.63563563563564	0.999998268496757\\
5.65565565565566	0.999998450064347\\
5.67567567567568	0.999998613137461\\
5.6956956956957	0.999998759540591\\
5.71571571571572	0.999998890924558\\
5.73573573573574	0.999999008782381\\
5.75575575575576	0.999999114463779\\
5.77577577577578	0.999999209188384\\
5.7957957957958	0.9999992940578\\
5.81581581581582	0.99999937006659\\
5.83583583583584	0.999999438112287\\
5.85585585585586	0.999999499004501\\
5.87587587587588	0.999999553473223\\
5.8958958958959	0.99999960217636\\
5.91591591591592	0.9999996457066\\
5.93593593593594	0.999999684597644\\
5.95595595595596	0.999999719329867\\
5.97597597597598	0.999999750335459\\
5.995995995996	0.999999778003092\\
6.01601601601602	0.999999802682146\\
6.03603603603604	0.999999824686549\\
6.05605605605606	0.999999844298252\\
6.07607607607608	0.999999861770372\\
6.0960960960961	0.999999877330051\\
6.11611611611612	0.999999891181026\\
6.13613613613614	0.999999903505964\\
6.15615615615616	0.999999914468575\\
6.17617617617618	0.999999924215508\\
6.1961961961962	0.999999932878079\\
6.21621621621622	0.99999994057382\\
6.23623623623624	0.999999947407884\\
6.25625625625626	0.999999953474305\\
6.27627627627628	0.999999958857139\\
6.2962962962963	0.999999963631489\\
6.31631631631632	0.999999967864435\\
6.33633633633634	0.999999971615859\\
6.35635635635636	0.999999974939197\\
6.37637637637638	0.999999977882115\\
6.3963963963964	0.99999998048711\\
6.41641641641642	0.999999982792055\\
6.43643643643644	0.99999998483069\\
6.45645645645646	0.999999986633059\\
6.47647647647648	0.999999988225902\\
6.4964964964965	0.999999989633011\\
6.51651651651652	0.999999990875543\\
6.53653653653654	0.999999991972305\\
6.55655655655656	0.999999992940011\\
6.57657657657658	0.999999993793502\\
6.5965965965966	0.999999994545956\\
6.61661661661662	0.999999995209068\\
6.63663663663664	0.99999999579321\\
6.65665665665666	0.999999996307581\\
6.67667667667668	0.999999996760331\\
6.6966966966967	0.999999997158684\\
6.71671671671672	0.999999997509033\\
6.73673673673674	0.99999999781704\\
6.75675675675676	0.999999998087714\\
6.77677677677678	0.999999998325483\\
6.7967967967968	0.999999998534265\\
6.81681681681682	0.999999998717519\\
6.83683683683684	0.999999998878303\\
6.85685685685686	0.999999999019314\\
6.87687687687688	0.999999999142936\\
6.8968968968969	0.999999999251268\\
6.91691691691692	0.999999999346164\\
6.93693693693694	0.999999999429257\\
6.95695695695696	0.999999999501986\\
6.97697697697698	0.999999999565618\\
6.996996996997	0.999999999621268\\
7.01701701701702	0.999999999669918\\
7.03703703703704	0.999999999712432\\
7.05705705705706	0.999999999749569\\
7.07707707707708	0.999999999781995\\
7.0970970970971	0.999999999810298\\
7.11711711711712	0.99999999983499\\
7.13713713713714	0.999999999856525\\
7.15715715715716	0.999999999875299\\
7.17717717717718	0.999999999891659\\
7.1971971971972	0.999999999905909\\
7.21721721721722	0.999999999918317\\
7.23723723723724	0.999999999929117\\
7.25725725725726	0.999999999938513\\
7.27727727727728	0.999999999946684\\
7.2972972972973	0.999999999953788\\
7.31731731731732	0.99999999995996\\
7.33733733733734	0.999999999965322\\
7.35735735735736	0.999999999969978\\
7.37737737737738	0.999999999974019\\
7.3973973973974	0.999999999977525\\
7.41741741741742	0.999999999980565\\
7.43743743743744	0.999999999983201\\
7.45745745745746	0.999999999985485\\
7.47747747747748	0.999999999987463\\
7.4974974974975	0.999999999989176\\
7.51751751751752	0.999999999990659\\
7.53753753753754	0.999999999991941\\
7.55755755755756	0.999999999993051\\
7.57757757757758	0.99999999999401\\
7.5975975975976	0.999999999994838\\
7.61761761761762	0.999999999995554\\
7.63763763763764	0.999999999996172\\
7.65765765765766	0.999999999996705\\
7.67767767767768	0.999999999997166\\
7.6976976976977	0.999999999997563\\
7.71771771771772	0.999999999997905\\
7.73773773773774	0.999999999998199\\
7.75775775775776	0.999999999998453\\
7.77777777777778	0.999999999998672\\
7.7977977977978	0.99999999999886\\
7.81781781781782	0.999999999999022\\
7.83783783783784	0.999999999999161\\
7.85785785785786	0.999999999999281\\
7.87787787787788	0.999999999999384\\
7.8978978978979	0.999999999999472\\
7.91791791791792	0.999999999999548\\
7.93793793793794	0.999999999999613\\
7.95795795795796	0.999999999999669\\
7.97797797797798	0.999999999999717\\
7.997997997998	0.999999999999758\\
8.01801801801802	0.999999999999793\\
8.03803803803804	0.999999999999823\\
8.05805805805806	0.999999999999849\\
8.07807807807808	0.999999999999871\\
8.0980980980981	0.99999999999989\\
8.11811811811812	0.999999999999906\\
8.13813813813814	0.99999999999992\\
8.15815815815816	0.999999999999932\\
8.17817817817818	0.999999999999942\\
8.1981981981982	0.999999999999951\\
8.21821821821822	0.999999999999958\\
8.23823823823824	0.999999999999964\\
8.25825825825826	0.99999999999997\\
8.27827827827828	0.999999999999974\\
8.2982982982983	0.999999999999978\\
8.31831831831832	0.999999999999981\\
8.33833833833834	0.999999999999984\\
8.35835835835836	0.999999999999987\\
8.37837837837838	0.999999999999989\\
8.3983983983984	0.99999999999999\\
8.41841841841842	0.999999999999992\\
8.43843843843844	0.999999999999993\\
8.45845845845846	0.999999999999994\\
8.47847847847848	0.999999999999995\\
8.4984984984985	0.999999999999996\\
8.51851851851852	0.999999999999996\\
8.53853853853854	0.999999999999997\\
8.55855855855856	0.999999999999997\\
8.57857857857858	0.999999999999998\\
8.5985985985986	0.999999999999998\\
8.61861861861862	0.999999999999998\\
8.63863863863864	0.999999999999999\\
8.65865865865866	0.999999999999999\\
8.67867867867868	0.999999999999999\\
8.6986986986987	0.999999999999999\\
8.71871871871872	0.999999999999999\\
8.73873873873874	0.999999999999999\\
8.75875875875876	0.999999999999999\\
8.77877877877878	1\\
8.7987987987988	1\\
8.81881881881882	1\\
8.83883883883884	1\\
8.85885885885886	1\\
8.87887887887888	1\\
8.8988988988989	1\\
8.91891891891892	1\\
8.93893893893894	1\\
8.95895895895896	1\\
8.97897897897898	1\\
8.998998998999	1\\
9.01901901901902	1\\
9.03903903903904	1\\
9.05905905905906	1\\
9.07907907907908	1\\
9.0990990990991	1\\
9.11911911911912	1\\
9.13913913913914	1\\
9.15915915915916	1\\
9.17917917917918	1\\
9.1991991991992	1\\
9.21921921921922	1\\
9.23923923923924	1\\
9.25925925925926	1\\
9.27927927927928	1\\
9.2992992992993	1\\
9.31931931931932	1\\
9.33933933933934	1\\
9.35935935935936	1\\
9.37937937937938	1\\
9.3993993993994	1\\
9.41941941941942	1\\
9.43943943943944	1\\
9.45945945945946	1\\
9.47947947947948	1\\
9.4994994994995	1\\
9.51951951951952	1\\
9.53953953953954	1\\
9.55955955955956	1\\
9.57957957957958	1\\
9.5995995995996	1\\
9.61961961961962	1\\
9.63963963963964	1\\
9.65965965965966	1\\
9.67967967967968	1\\
9.6996996996997	1\\
9.71971971971972	1\\
9.73973973973974	1\\
9.75975975975976	1\\
9.77977977977978	1\\
9.7997997997998	1\\
9.81981981981982	1\\
9.83983983983984	1\\
9.85985985985986	1\\
9.87987987987988	1\\
9.8998998998999	1\\
9.91991991991992	1\\
9.93993993993994	1\\
9.95995995995996	1\\
9.97997997997998	1\\
10	1\\
};
\addlegendentry{$\Pr(f = 1 \given x, \sigma_{\text{spike}}^2, \sigma_{\text{slab}}^2, \omega^2)$};

\end{axis}
\end{tikzpicture}%
  \caption{The posterior probability for the slab flag $(f = 1)$ for
    the spike-and-slab prior problem as a function of the observation
    $x$.}
\end{figure}

For the third part, we use the sum rule:
\begin{equation*}
  p(\theta \given x, \sigma_{\text{spike}}^2, \sigma_{\text{slab}}^2, \omega^2)
  =
  \sum_f
  p(\theta \given f, x, \sigma_{\text{spike}}^2, \sigma_{\text{slab}}^2, \omega^2)
  \Pr(f \given x, \sigma_{\text{spike}}^2, \sigma_{\text{slab}}^2, \omega^2).
\end{equation*}
We need to compute the posterior on $\theta$ given $f$ and $x$.  Here
we may use the result from the last problem:
\begin{align*}
  p(\theta \given f = 0, x, \sigma_{\text{spike}}^2, \omega^2)
  &=
  \mc{N}(\theta;
  \tau_{\text{spike}}^{-1} \omega^{-2} x,
  \tau_{\text{spike}}^{-1}) \\
  p(\theta \given f = 1, x, \sigma_{\text{slab}}^2, \omega^2)
  &=
  \mc{N}(\theta;
  \tau_{\text{slab}}^{-1} \omega^{-2} x,
  \tau_{\text{slab}}^{-1}),
\end{align*}
where $\tau_{\text{spike}} = (\omega^{-2} +
\sigma_{\text{spike}}^{-2})$, and $\tau_{\text{slab}}$ is defined
similarly.

Finally, the posterior for $\theta$ for the observation $x = 3$ with
$\omega^2 = 0.1^2$ is plotted below.

\begin{figure}[h]
  \centering
  % This file was created by matlab2tikz.
% Minimal pgfplots version: 1.3
%
\tikzsetnextfilename{problem_6_theta_posterior}
\definecolor{mycolor1}{rgb}{0.12157,0.47059,0.70588}%
%
\begin{tikzpicture}

\begin{axis}[%
width=0.95092\figurewidth,
height=\figureheight,
at={(0\figurewidth,0\figureheight)},
scale only axis,
xmin=1,
xmax=5,
xlabel={$\theta$},
ymin=0,
ymax=4.9,
axis x line*=bottom,
axis y line*=left,
legend style={at={(0.03,0.97)},anchor=north west,legend cell align=left,align=left,fill=none,draw=none}
]
\addplot [color=mycolor1,solid]
  table[row sep=crcr]{%
1	3.63776039197452e-86\\
1.004004004004	8.06808463651127e-86\\
1.00800800800801	1.78650972705716e-85\\
1.01201201201201	3.94947093063364e-85\\
1.01601601601602	8.7170807161123e-85\\
1.02002002002002	1.92088707801312e-84\\
1.02402402402402	4.22601697604461e-84\\
1.02802802802803	9.28237879343597e-84\\
1.03203203203203	2.03556979162389e-83\\
1.03603603603604	4.45667927113801e-83\\
1.04004004004004	9.74171569945077e-83\\
1.04404404404404	2.12597533244978e-82\\
1.04804804804805	4.63211896225954e-82\\
1.05205205205205	1.00762727436008e-81\\
1.05605605605606	2.18836048767229e-81\\
1.06006006006006	4.7450038476235e-81\\
1.06406406406406	1.02719534097053e-80\\
1.06806806806807	2.22007831377024e-80\\
1.07207207207207	4.79051653910033e-80\\
1.07607607607608	1.03203684647284e-79\\
1.08008008008008	2.21976442165519e-79\\
1.08408408408408	4.76669547929976e-79\\
1.08808808808809	1.02194325689926e-78\\
1.09209209209209	2.18743433330866e-78\\
1.0960960960961	4.67457517719626e-78\\
1.1001001001001	9.97351326735828e-78\\
1.1041041041041	2.1244822300383e-77\\
1.10810810810811	4.51811169252862e-77\\
1.11211211211211	9.59311752878576e-77\\
1.11611611611612	2.03358123114667e-76\\
1.12012012012012	4.30390084696765e-76\\
1.12412412412412	9.09414717919119e-76\\
1.12812812812813	1.91849514100758e-75\\
1.13213213213213	4.04071712761286e-75\\
1.13613613613614	8.49679673938682e-75\\
1.14014014014014	1.78382013445466e-74\\
1.14414414414414	3.73891818465363e-74\\
1.14814814814815	7.82420006009068e-74\\
1.15215215215215	1.63468106379664e-73\\
1.15615615615616	3.40977116753137e-73\\
1.16016016016016	7.10095189056048e-73\\
1.16416416416416	1.47641026842789e-72\\
1.16816816816817	3.06476169764221e-72\\
1.17217217217217	6.35163524913371e-72\\
1.17617617617618	1.31423670050516e-71\\
1.18018018018018	2.71494384086835e-71\\
1.18418418418418	5.59947547447161e-71\\
1.18818818818819	1.15301006162266e-70\\
1.19219219219219	2.37038079234248e-70\\
1.1961961961962	4.86521941519061e-70\\
1.2002002002002	9.96979092098347e-70\\
1.2042042042042	2.03971268151645e-69\\
1.20820820820821	4.16630716382896e-69\\
1.21221221221221	8.49636061958238e-69\\
1.21621621621622	1.72987200419748e-68\\
1.22022022022022	3.51636867715366e-68\\
1.22422422422422	7.13631898015345e-68\\
1.22822822822823	1.44595088891412e-67\\
1.23223223223223	2.92504322200394e-67\\
1.23623623623624	5.90759220736211e-67\\
1.24024024024024	1.19120971130535e-66\\
1.24424424424424	2.39808988101356e-66\\
1.24824824824825	4.81994655019749e-66\\
1.25225225225225	9.67205020822669e-66\\
1.25625625625626	1.93773523651006e-65\\
1.26026026026026	3.87587645754112e-65\\
1.26426426426426	7.74007250994576e-65\\
1.26826826826827	1.54319130650445e-64\\
1.27227227227227	3.0718086760991e-64\\
1.27627627627628	6.10475492422672e-64\\
1.28028028028028	1.21127290457822e-63\\
1.28428428428428	2.39947091197005e-63\\
1.28828828828829	4.74557363136453e-63\\
1.29229229229229	9.37047710675991e-63\\
1.2962962962963	1.84728742665125e-62\\
1.3003003003003	3.63585913408001e-62\\
1.3043043043043	7.14462517176914e-62\\
1.30830830830831	1.40168910582621e-61\\
1.31231231231231	2.74551497415983e-61\\
1.31631631631632	5.36902996706926e-61\\
1.32032032032032	1.04825695893391e-60\\
1.32432432432432	2.04333520674524e-60\\
1.32832832832833	3.97659573283386e-60\\
1.33233233233233	7.72650784076843e-60\\
1.33633633633634	1.49883927139673e-59\\
1.34034034034034	2.90286545590064e-59\\
1.34434434434434	5.61304862143134e-59\\
1.34834834834835	1.08360444503717e-58\\
1.35235235235235	2.08854026358745e-58\\
1.35635635635636	4.01897265232071e-58\\
1.36036036036036	7.7212461620283e-58\\
1.36436436436436	1.4810165746751e-57\\
1.36836836836837	2.83617246779717e-57\\
1.37237237237237	5.42257511507274e-57\\
1.37637637637638	1.03509155058837e-56\\
1.38038038038038	1.97265992586531e-56\\
1.38438438438438	3.75340974876532e-56\\
1.38838838838839	7.13017280704461e-56\\
1.39239239239239	1.35230446272741e-55\\
1.3963963963964	2.5606444519119e-55\\
1.4004004004004	4.84088180383429e-55\\
1.4044044044044	9.13692589937921e-55\\
1.40840840840841	1.72177413017177e-54\\
1.41241241241241	3.2393116398983e-54\\
1.41641641641642	6.0845685658326e-54\\
1.42042042042042	1.14105710245357e-53\\
1.42442442442442	2.1364145440257e-53\\
1.42842842842843	3.99359719283324e-53\\
1.43243243243243	7.45321421326079e-53\\
1.43643643643644	1.38874845440955e-52\\
1.44044044044044	2.58347486972643e-52\\
1.44444444444444	4.7982800260225e-52\\
1.44844844844845	8.89749313172455e-52\\
1.45245245245245	1.64721574520984e-51\\
1.45645645645646	3.04462695839568e-51\\
1.46046046046046	5.61847562033469e-51\\
1.46446446446446	1.03515110571072e-50\\
1.46846846846847	1.90410049685718e-50\\
1.47247247247247	3.49684898965935e-50\\
1.47647647647648	6.41157628868389e-50\\
1.48048048048048	1.17369081173096e-49\\
1.48448448448448	2.14508042285451e-49\\
1.48848848848849	3.91412298792325e-49\\
1.49249249249249	7.1306053415482e-49\\
1.4964964964965	1.29693857221685e-48\\
1.5005005005005	2.35512258949061e-48\\
1.5045045045045	4.26981211025558e-48\\
1.50850850850851	7.72867735438618e-48\\
1.51251251251251	1.39669888764087e-47\\
1.51651651651652	2.52000629427228e-47\\
1.52052052052052	4.53943418206884e-47\\
1.52452452452452	8.16400219891523e-47\\
1.52852852852853	1.46590499823045e-46\\
1.53253253253253	2.62790640507295e-46\\
1.53653653653654	4.70343718743925e-46\\
1.54054054054054	8.40470013974892e-46\\
1.54454454454454	1.49944503529152e-45\\
1.54854854854855	2.67079382093595e-45\\
1.55255255255255	4.74954156841721e-45\\
1.55655655655656	8.43265969349404e-45\\
1.56056056056056	1.49478606413791e-44\\
1.56456456456456	2.64542299460443e-44\\
1.56856856856857	4.67425993647609e-44\\
1.57257257257257	8.24579021188677e-44\\
1.57657657657658	1.4522902530041e-43\\
1.58058058058058	2.55373787735185e-43\\
1.58458458458458	4.48333304961665e-43\\
1.58858858858859	7.85828018865591e-43\\
1.59259259259259	1.37516837227121e-42\\
1.5965965965966	2.4026257901152e-42\\
1.6006006006006	4.19100638403625e-42\\
1.6046046046046	7.2988169913768e-42\\
1.60860860860861	1.26907890815848e-41\\
1.61261261261261	2.20306247877653e-41\\
1.61661661661662	3.81827385155708e-41\\
1.62062062062062	6.60707810570539e-41\\
1.62462462462462	1.1414422741565e-40\\
1.62862862862863	1.96879567295785e-40\\
1.63263263263263	3.39038921239649e-40\\
1.63663663663664	5.82908950541495e-40\\
1.64064064064064	1.00058544859654e-39\\
1.64464464464464	1.71478616607671e-39\\
1.64864864864865	2.934054165674e-39\\
1.65265265265265	5.01220341651877e-39\\
1.65665665665666	8.54853496714145e-39\\
1.66066066066066	1.45565071541446e-38\\
1.66466466466466	2.47471495948714e-38\\
1.66866866866867	4.20044930321834e-38\\
1.67267267267267	7.11817905528249e-38\\
1.67667667667668	1.2043277829146e-37\\
1.68068068068068	2.03433830916838e-37\\
1.68468468468468	3.430870804192e-37\\
1.68868868868869	5.77681294197233e-37\\
1.69269269269269	9.71124787540346e-37\\
1.6966966966967	1.62991369020379e-36\\
1.7007007007007	2.73122229767198e-36\\
1.7047047047047	4.56932874676079e-36\\
1.70870870870871	7.63221709565249e-36\\
1.71271271271271	1.27277612557003e-35\\
1.71671671671672	2.11912395666788e-35\\
1.72072072072072	3.52260363439986e-35\\
1.72472472472472	5.84620869305234e-35\\
1.72872872872873	9.68697121332416e-35\\
1.73273273273273	1.60252533673375e-34\\
1.73673673673674	2.64682386985059e-34\\
1.74074074074074	4.36464036503034e-34\\
1.74474474474474	7.18580145299991e-34\\
1.74874874874875	1.18115075056604e-33\\
1.75275275275275	1.93837972310806e-33\\
1.75675675675676	3.17596596247361e-33\\
1.76076076076076	5.19536801932945e-33\\
1.76476476476476	8.48516580280346e-33\\
1.76876876876877	1.38359164651914e-32\\
1.77277277277277	2.25247072259214e-32\\
1.77677677677678	3.66112069864393e-32\\
1.78078078078078	5.94117918583364e-32\\
1.78478478478478	9.62575959074847e-32\\
1.78878878878879	1.55704502091402e-31\\
1.79279279279279	2.51461309595197e-31\\
1.7967967967968	4.05457243588815e-31\\
1.8008008008008	6.52713937198398e-31\\
1.8048048048048	1.04907052491638e-30\\
1.80880880880881	1.68341233471602e-30\\
1.81281281281281	2.69699654834841e-30\\
1.81681681681682	4.31394284442207e-30\\
1.82082082082082	6.88925941762185e-30\\
1.82482482482482	1.09843633660284e-29\\
1.82882882882883	1.74856382602039e-29\\
1.83283283283283	2.77902452484895e-29\\
1.83683683683684	4.40968539575529e-29\\
1.84084084084084	6.98597803496144e-29\\
1.84484484484484	1.10497195073358e-28\\
1.84884884884885	1.74493705824263e-28\\
1.85285285285285	2.75114068128519e-28\\
1.85685685685686	4.33062320076812e-28\\
1.86086086086086	6.80601043305227e-28\\
1.86486486486486	1.06792187372165e-27\\
1.86886886886887	1.67298119520945e-27\\
1.87287287287287	2.6166605641396e-27\\
1.87687687687688	4.08609555068409e-27\\
1.88088088088088	6.37051427240513e-27\\
1.88488488488488	9.91620190440782e-27\\
1.88888888888889	1.54106575098223e-26\\
1.89289289289289	2.39112325687026e-26\\
1.8968968968969	3.70414341189697e-26\\
1.9009009009009	5.728998433918e-26\\
1.9049049049049	8.84656643563114e-26\\
1.90890890890891	1.36387936763573e-25\\
1.91291291291291	2.09933815871518e-25\\
1.91691691691692	3.22622098492333e-25\\
1.92092092092092	4.95006774972193e-25\\
1.92492492492492	7.58286879243947e-25\\
1.92892892892893	1.1597419180959e-24\\
1.93293293293293	1.77090241799037e-24\\
1.93693693693694	2.69981134473772e-24\\
1.94094094094094	4.10939382281592e-24\\
1.94494494494494	6.24493203897828e-24\\
1.94894894894895	9.47508957152371e-24\\
1.95295295295295	1.43530644531426e-23\\
1.95695695695696	2.17075943490049e-23\\
1.96096096096096	3.27781591152555e-23\\
1.96496496496496	4.94155085013888e-23\\
1.96896896896897	7.43785785929147e-23\\
1.97297297297297	1.11773390476824e-22\\
1.97697697697698	1.67700735699094e-22\\
1.98098098098098	2.5121036357047e-22\\
1.98498498498498	3.75704352013928e-22\\
1.98898898898899	5.609976440754e-22\\
1.99299299299299	8.36338395618992e-22\\
1.996996996997	1.24482776946048e-21\\
2.001001001001	1.8498767054384e-21\\
2.00500500500501	2.74462257157233e-21\\
2.00900900900901	4.06563940920715e-21\\
2.01301301301301	6.01286581883833e-21\\
2.01701701701702	8.87852135762688e-21\\
2.02102102102102	1.30889949402636e-20\\
2.02502502502503	1.9265422107312e-20\\
2.02902902902903	2.83111414197659e-20\\
2.03303303303303	4.15377433696967e-20\\
2.03703703703704	6.08464347311648e-20\\
2.04104104104104	8.89885517456963e-20\\
2.04504504504505	1.29939130209226e-19\\
2.04904904904905	1.89431689857853e-19\\
2.05305305305305	2.75722508602289e-19\\
2.05705705705706	4.00681015946534e-19\\
2.06106106106106	5.81342824786909e-19\\
2.06506506506507	8.42117940603213e-19\\
2.06906906906907	1.21792521923615e-18\\
2.07307307307307	1.7586340398803e-18\\
2.07707707707708	2.53534766994939e-18\\
2.08108108108108	3.64927684986107e-18\\
2.08508508508509	5.24425006158867e-18\\
2.08908908908909	7.52432089654177e-18\\
2.09309309309309	1.07785070523332e-17\\
2.0970970970971	1.5415490304817e-17\\
2.1011011011011	2.20122051889221e-17\\
2.10510510510511	3.13817592952186e-17\\
2.10910910910911	4.46682144367241e-17\\
2.11311311311311	6.34786255988458e-17\\
2.11711711711712	9.00666690305737e-17\\
2.12112112112112	1.27587579160506e-16\\
2.12512512512513	1.80451480116107e-16\\
2.12912912912913	2.5481224916784e-16\\
2.13313313313313	3.59242740541929e-16\\
2.13713713713714	5.05665812700011e-16\\
2.14114114114114	7.10635864858002e-16\\
2.14514514514515	9.97099810950424e-16\\
2.14914914914915	1.3968126944487e-15\\
2.15315315315315	1.95364560578012e-15\\
2.15715715715716	2.72810771054408e-15\\
2.16116116116116	3.80351735415533e-15\\
2.16516516516517	5.29440909917025e-15\\
2.16916916916917	7.35796705788327e-15\\
2.17317317317317	1.02095470417112e-14\\
2.17717717717718	1.41437133472417e-14\\
2.18118118118118	1.95627000584929e-14\\
2.18518518518519	2.70148505101884e-14\\
2.18918918918919	3.72464431267595e-14\\
2.19319319319319	5.12714441415259e-14\\
2.1971971971972	7.04652251622073e-14\\
2.2012012012012	9.66902602468144e-14\\
2.20520520520521	1.32464421565817e-13\\
2.20920920920921	1.81185917736283e-13\\
2.21321321321321	2.47433414590665e-13\\
2.21721721721722	3.37365768036869e-13\\
2.22122122122122	4.59253500551103e-13\\
2.22522522522523	6.24184205061029e-13\\
2.22922922922923	8.46997154800054e-13\\
2.23323323323323	1.14751942355353e-12\\
2.23723723723724	1.55219791532428e-12\\
2.24124124124124	2.09625031987031e-12\\
2.24524524524525	2.8264950052458e-12\\
2.24924924924925	3.80506759049447e-12\\
2.25325325325325	5.11429340835039e-12\\
2.25725725725726	6.8630640678047e-12\\
2.26126126126126	9.19516766354177e-12\\
2.26526526526527	1.23001519157498e-11\\
2.26926926926927	1.6427465293311e-11\\
2.27327327327327	2.19048325677767e-11\\
2.27727727727728	2.91620902910114e-11\\
2.28128128128128	3.876204943129e-11\\
2.28528528528529	5.1440385622417e-11\\
2.28928928928929	6.81571055709229e-11\\
2.29329329329329	9.01628302001777e-11\\
2.2972972972973	1.190840071369e-10\\
2.3013013013013	1.57032261765437e-10\\
2.30530530530531	2.0674447234332e-10\\
2.30930930930931	2.71761878390188e-10\\
2.31331331331331	3.56658686043077e-10\\
2.31731731731732	4.67333321441823e-10\\
2.32132132132132	6.11378828369453e-10\\
2.32532532532533	7.98553061716891e-10\\
2.32932932932933	1.04137447935308e-09\\
2.33333333333333	1.35587570688974e-09\\
2.33733733733734	1.76255491544682e-09\\
2.34134134134134	2.28757474148867e-09\\
2.34534534534535	2.96427038743533e-09\\
2.34934934934935	3.83504314972562e-09\\
2.35335335335335	4.95373350016347e-09\\
2.35735735735736	6.38858968188452e-09\\
2.36136136136136	8.22597439052914e-09\\
2.36536536536537	1.05749844667963e-08\\
2.36936936936937	1.35731977825197e-08\\
2.37337337337337	1.73938090245337e-08\\
2.37737737737738	2.22544734805943e-08\\
2.38138138138138	2.84282471061486e-08\\
2.38538538538539	3.62570943255088e-08\\
2.38938938938939	4.61685347998129e-08\\
2.39339339339339	5.86961198147569e-08\\
2.3973973973974	7.45045715360115e-08\\
2.4014014014014	9.44205882449674e-08\\
2.40540540540541	1.19470520535414e-07\\
2.40940940940941	1.50926362774672e-07\\
2.41341341341341	1.90361787170378e-07\\
2.41741741741742	2.3972028169774e-07\\
2.42142142142142	3.01397846095785e-07\\
2.42542542542543	3.78343161409477e-07\\
2.42942942942943	4.74178688630225e-07\\
2.43343343343343	5.93346782699714e-07\\
2.43743743743744	7.41285642767765e-07\\
2.44144144144144	9.24640772973784e-07\\
2.44544544544545	1.15151861666955e-06\\
2.44944944944945	1.4317901692033e-06\\
2.45345345345345	1.77745369029989e-06\\
2.45745745745746	2.20306714879061e-06\\
2.46146146146146	2.72626276405004e-06\\
2.46546546546547	3.36835798608235e-06\\
2.46946946946947	4.15507950784525e-06\\
2.47347347347347	5.11741945912242e-06\\
2.47747747747748	6.29264582287728e-06\\
2.48148148148148	7.72549237598011e-06\\
2.48548548548549	9.4695571213071e-06\\
2.48948948948949	1.1588942283254e-05\\
2.49349349349349	1.41601735200321e-05\\
2.4974974974975	1.72744411000243e-05\\
2.5015015015015	2.10402114315071e-05\\
2.50550550550551	2.55862635591532e-05\\
2.50950950950951	3.10652120782188e-05\\
2.51351351351351	3.76575853957002e-05\\
2.51751751751752	4.55765364093647e-05\\
2.52152152152152	5.5073271495944e-05\\
2.52552552552553	6.64432932060778e-05\\
2.52952952952953	8.0033562252344e-05\\
2.53353353353353	9.62506952319047e-05\\
2.53753753753754	0.00011557032601595\\
2.54154154154154	0.000138547770815548\\
2.54554554554555	0.000165830149529111\\
2.54954954954955	0.000198170128998381\\
2.55355355355355	0.000236441467129897\\
2.55755755755756	0.000281656550330423\\
2.56156156156156	0.000334986130178795\\
2.56556556556557	0.000397781478499616\\
2.56956956956957	0.000471599192764518\\
2.57357357357357	0.0005582288957329\\
2.57757757757758	0.000659724084122493\\
2.58158158158158	0.000778436390492269\\
2.58558558558559	0.000917053530004028\\
2.58958958958959	0.00107864120883567\\
2.59359359359359	0.00126668927323386\\
2.5975975975976	0.00148516237696336\\
2.6016016016016	0.00173855543963142\\
2.60560560560561	0.00203195415840952\\
2.60960960960961	0.00237110082037256\\
2.61361361361361	0.00276246564133566\\
2.61761761761762	0.0032133238289937\\
2.62162162162162	0.00373183853263643\\
2.62562562562563	0.00432714979803336\\
2.62962962962963	0.00500946959357054\\
2.63363363363363	0.00579018291173394\\
2.63763763763764	0.00668195487800134\\
2.64164164164164	0.00769884371660054\\
2.64564564564565	0.00885641932903282\\
2.64964964964965	0.0101718871364522\\
2.65365365365365	0.0116642167207934\\
2.65765765765766	0.0133542746720214\\
2.66166166166166	0.0152649609102235\\
2.66566566566567	0.0174213476019967\\
2.66966966966967	0.0198508196313709\\
2.67367367367367	0.0225832154173429\\
2.67767767767768	0.0256509666942942\\
2.68168168168168	0.0290892356896463\\
2.68568568568569	0.0329360479470698\\
2.68968968968969	0.0372324188556178\\
2.69369369369369	0.0420224717578882\\
2.6976976976977	0.0473535453267561\\
2.7017017017017	0.0532762877235047\\
2.70570570570571	0.0598447348840363\\
2.70970970970971	0.0671163701280822\\
2.71371371371371	0.0751521621530557\\
2.71771771771772	0.0840165783638996\\
2.72172172172172	0.093777570407273\\
2.72572572572573	0.104506528727511\\
2.72972972972973	0.116278202947501\\
2.73373373373373	0.129170584904467\\
2.73773773773774	0.143264751243283\\
2.74174174174174	0.158644662592193\\
2.74574574574575	0.1753969165218\\
2.74974974974975	0.193610451721143\\
2.75375375375375	0.213376201117173\\
2.75775775775776	0.234786692018588\\
2.76176176176176	0.257935591782227\\
2.76576576576577	0.282917197981284\\
2.76976976976977	0.309825872598146\\
2.77377377377377	0.3387554203691\\
2.77777777777778	0.369798412070827\\
2.78178178178178	0.403045454254567\\
2.78578578578579	0.438584407698303\\
2.78978978978979	0.47649955765299\\
2.79379379379379	0.516870739797335\\
2.7977977977978	0.559772426678315\\
2.8018018018018	0.605272780289435\\
2.80580580580581	0.65343267731502\\
2.80980980980981	0.704304714432994\\
2.81381381381381	0.757932201906546\\
2.81781781781782	0.814348154493347\\
2.82182182182182	0.873574289442831\\
2.82582582582583	0.935620042023657\\
2.82982982982983	1.00048160960862\\
2.83383383383383	1.06814103582737\\
2.83783783783784	1.13856534666548\\
2.84184184184184	1.21170575062499\\
2.84584584584585	1.28749691515741\\
2.84984984984985	1.36585633152065\\
2.85385385385385	1.44668377998976\\
2.85785785785786	1.52986090695886\\
2.86186186186186	1.61525092490344\\
2.86586586586587	1.70269844542605\\
2.86986986986987	1.79202945468483\\
2.87387387387387	1.88305143940486\\
2.87787787787788	1.97555367040636\\
2.88188188188188	2.0693076491566\\
2.88588588588589	2.16406772127988\\
2.88988988988989	2.25957185925594\\
2.89389389389389	2.35554261471908\\
2.8978978978979	2.4516882388625\\
2.9019019019019	2.5477039674738\\
2.90590590590591	2.64327346510866\\
2.90990990990991	2.73807042087496\\
2.91391391391391	2.83176028627869\\
2.91791791791792	2.92400214360762\\
2.92192192192192	3.01445069142632\\
2.92592592592593	3.10275833196003\\
2.92992992992993	3.18857734348338\\
2.93393393393393	3.27156211933106\\
2.93793793793794	3.35137145384102\\
2.94194194194194	3.42767085444717\\
2.94594594594595	3.50013485828414\\
2.94994994994995	3.56844933106689\\
2.95395395395395	3.63231372568007\\
2.95795795795796	3.69144327786605\\
2.96196196196196	3.74557111664421\\
2.96596596596597	3.79445026762902\\
2.96996996996997	3.83785552823917\\
2.97397397397397	3.87558519489714\\
2.97797797797798	3.90746262369681\\
2.98198198198198	3.93333760765016\\
2.98598598598599	3.95308755549207\\
2.98998998998999	3.96661845910074\\
2.99399399399399	3.97386563885248\\
2.997997997998	3.97479425864204\\
3.002002002002	3.96939960483035\\
3.00600600600601	3.95770712599411\\
3.01001001001001	3.93977223300923\\
3.01401401401401	3.91567986166508\\
3.01801801801802	3.88554380263993\\
3.02202202202202	3.84950580623342\\
3.02602602602603	3.80773447171193\\
3.03003003003003	3.76042393344361\\
3.03403403403403	3.70779235814922\\
3.03803803803804	3.65008026954332\\
3.04204204204204	3.58754871836253\\
3.04604604604605	3.5204773172514\\
3.05005005005005	3.4491621611838\\
3.05405405405405	3.37391365502666\\
3.05805805805806	3.29505427049272\\
3.06206206206206	3.21291625507793\\
3.06606606606607	3.12783931563601\\
3.07007007007007	3.04016829901386\\
3.07407407407407	2.95025089166674\\
3.07807807807808	2.85843535940368\\
3.08208208208208	2.76506834740107\\
3.08608608608609	2.67049275938523\\
3.09009009009009	2.57504573344688\\
3.09409409409409	2.47905673033922\\
3.0980980980981	2.38284574835354\\
3.1021021021021	2.28672167699242\\
3.10610610610611	2.19098079970084\\
3.11011011011011	2.09590545389989\\
3.11411411411411	2.00176285452795\\
3.11811811811812	1.9088040852582\\
3.12212212212212	1.81726325955846\\
3.12612612612613	1.72735685181635\\
3.13013013013013	1.63928319689326\\
3.13413413413413	1.55322215471775\\
3.13813813813814	1.4693349349019\\
3.14214214214214	1.38776407487896\\
3.14614614614615	1.30863356373184\\
3.15015015015015	1.23204910271943\\
3.15415415415415	1.15809849251891\\
3.15815815815816	1.08685213639129\\
3.16216216216216	1.01836364784644\\
3.16616616616617	0.952670550929776\\
3.17017017017017	0.88979506097321\\
3.17417417417417	0.829744933539734\\
3.17817817817818	0.77251436933623\\
3.18218218218218	0.718084963061928\\
3.18618618618619	0.66642668448775\\
3.19019019019019	0.617498880511726\\
3.19419419419419	0.571251287492478\\
3.1981981981982	0.527625043811991\\
3.2022022022022	0.486553693344642\\
3.20620620620621	0.44796417129623\\
3.21021021021021	0.411777764709443\\
3.21421421421421	0.377911040795517\\
3.21821821821822	0.346276737131861\\
3.22222222222222	0.316784608648703\\
3.22622622622623	0.289342227201702\\
3.23023023023023	0.263855730381243\\
3.23423423423423	0.240230517032164\\
3.23823823823824	0.21837188774175\\
3.24224224224224	0.198185629291318\\
3.24624624624625	0.179578542751593\\
3.25025025025025	0.162458915529826\\
3.25425425425425	0.146736938243458\\
3.25825825825826	0.132325067799307\\
3.26226226226226	0.119138338497398\\
3.26626626626627	0.107094623354853\\
3.27027027027027	0.0961148481586847\\
3.27427427427427	0.0861231610087228\\
3.27827827827828	0.0770470603057959\\
3.28228228228228	0.0688174842790476\\
3.28628628628629	0.0613688652334083\\
3.29029029029029	0.0546391517381018\\
3.29429429429429	0.0485698019738544\\
3.2982982982983	0.0431057514149401\\
3.3023023023023	0.0381953579469164\\
3.30630630630631	0.0337903274165982\\
3.31031031031031	0.029845622482111\\
3.31431431431431	0.0263193574821531\\
3.31831831831832	0.0231726818792205\\
3.32232232232232	0.0203696546554532\\
3.32632632632633	0.0178771118557118\\
3.33033033033033	0.0156645292838874\\
3.33433433433433	0.0137038821683533\\
3.33833833833834	0.0119695034236346\\
3.34234234234234	0.0104379419501724\\
3.34634634634635	0.00908782223453201\\
3.35035035035035	0.00789970634026399\\
3.35435435435435	0.0068559592162328\\
3.35835835835836	0.0059406180956781\\
3.36236236236236	0.00513926661634395\\
3.36636636636637	0.00443891416023997\\
3.37037037037037	0.0038278807912869\\
3.37437437437437	0.00329568806032406\\
3.37837837837838	0.00283295584963541\\
3.38238238238238	0.00243130534302394\\
3.38638638638639	0.00208326813215131\\
3.39039039039039	0.0017822014048719\\
3.39439439439439	0.00152220910604806\\
3.3983983983984	0.00129806891520396\\
3.4024024024024	0.00110516484767116\\
3.40640640640641	0.000939425255894949\\
3.41041041041041	0.000797265984581859\\
3.41441441441441	0.000675538416660411\\
3.41841841841842	0.000571482135890656\\
3.42242242242242	0.000482681925711831\\
3.42642642642643	0.000407028821909161\\
3.43043043043043	0.000342684938292827\\
3.43443443443443	0.000288051789240621\\
3.43843843843844	0.000241741840127924\\
3.44244244244244	0.000202553025868824\\
3.44644644644645	0.000169445988580783\\
3.45045045045045	0.000141523797368236\\
3.45445445445445	0.000118013926050008\\
3.45845845845846	9.82522780261049e-05\\
3.46246246246246	8.16690611279894e-05\\
3.46646646646647	6.77763289984579e-05\\
3.47047047047047	5.61570191139417e-05\\
3.47447447447447	4.64553308386504e-05\\
3.47847847847848	3.83682997611198e-05\\
3.48248248248248	3.16384369115815e-05\\
3.48648648648649	2.60473132192944e-05\\
3.49049049049049	2.14099806897407e-05\\
3.49449449449449	1.75701322281813e-05\\
3.4984984984985	1.43959117897736e-05\\
3.5025025025025	1.17762955920068e-05\\
3.50650650650651	9.61797348832215e-06\\
3.51051051051051	7.84266728714043e-06\\
3.51451451451451	6.38482983002937e-06\\
3.51851851851852	5.18967504376905e-06\\
3.52252252252252	4.21149498511302e-06\\
3.52652652652653	3.41222513843166e-06\\
3.53053053053053	2.76022394125513e-06\\
3.53453453453453	2.22923673801267e-06\\
3.53853853853854	1.79751813509933e-06\\
3.54254254254254	1.4470900874312e-06\\
3.54654654654655	1.16311602354655e-06\\
3.55055055055055	9.33373947386874e-07\\
3.55455455455455	7.47813772843992e-07\\
3.55855855855856	5.98186182838136e-07\\
3.56256256256256	4.77732086992596e-07\\
3.56656656656657	3.80923307751762e-07\\
3.57057057057057	3.03246478850279e-07\\
3.57457457457457	2.41023315144863e-07\\
3.57857857857858	1.91261429734932e-07\\
3.58258258258258	1.51530751875057e-07\\
3.58658658658659	1.19861354445436e-07\\
3.59059059059059	9.46591479957749e-08\\
3.59459459459459	7.46364533034119e-08\\
3.5985985985986	5.87549381829148e-08\\
3.6026026026026	4.61788077710751e-08\\
3.60660660660661	3.62364802486022e-08\\
3.61061061061061	2.83892703632467e-08\\
3.61461461461461	2.22058485693568e-08\\
3.61861861861862	1.73414505454741e-08\\
3.62262262262262	1.35209859111606e-08\\
3.62662662662663	1.05253410114811e-08\\
3.63063063063063	8.18029288913112e-09\\
3.63463463463463	6.34755365616106e-09\\
3.63863863863864	4.91754953491822e-09\\
3.64264264264264	3.80360953947581e-09\\
3.64664664664665	2.9372974025719e-09\\
3.65065065065065	2.26466886813638e-09\\
3.65465465465465	1.74327652139418e-09\\
3.65865865865866	1.33977731952991e-09\\
3.66266266266266	1.02802510030883e-09\\
3.66666666666667	7.87552581393071e-10\\
3.67067067067067	6.02365557309872e-10\\
3.67467467467467	4.5998685609155e-10\\
3.67867867867868	3.50699720060118e-10\\
3.68268268268268	2.66950113883955e-10\\
3.68668668668669	2.0287544420758e-10\\
3.69069069069069	1.53933635743736e-10\\
3.69469469469469	1.16611727345539e-10\\
3.6986986986987	8.81973581142585e-11\\
3.7027027027027	6.65998972211251e-11\\
3.70670670670671	5.02106871370845e-11\\
3.71071071071071	3.77940455704733e-11\\
3.71471471471471	2.84024105343344e-11\\
3.71871871871872	2.13104003137712e-11\\
3.72272272272272	1.59636645664979e-11\\
3.72672672672673	1.1939280255279e-11\\
3.73073073073073	8.91514181800633e-12\\
3.73473473473473	6.6463454636313e-12\\
3.73873873873874	4.94700217740603e-12\\
3.74274274274274	3.67625659246502e-12\\
3.74674674674675	2.72755831953914e-12\\
3.75075075075075	2.0204438154059e-12\\
3.75475475475475	1.49425254610867e-12\\
3.75875875875876	1.10333073049856e-12\\
3.76276276276276	8.13377019641498e-13\\
3.76676676676677	5.98663176053857e-13\\
3.77077077077077	4.39923990956596e-13\\
3.77477477477477	3.22758131059109e-13\\
3.77877877877878	2.36418344610021e-13\\
3.78278278278278	1.72897843854715e-13\\
3.78678678678679	1.26241573340248e-13\\
3.79079079079079	9.20279242046984e-14\\
3.79479479479479	6.69793997411346e-14\\
3.7987987987988	4.8670663239123e-14\\
3.8028028028028	3.53099950817958e-14\\
3.80680680680681	2.55759880368703e-14\\
3.81081081081081	1.84957328893847e-14\\
3.81481481481481	1.33541122505769e-14\\
3.81881881881882	9.6263765213037e-15\\
3.82282282282282	6.92811363369486e-15\\
3.82682682682683	4.97819022324326e-15\\
3.83083083083083	3.57134899560632e-15\\
3.83483483483483	2.5579815469411e-15\\
3.83883883883884	1.82922372351862e-15\\
3.84284284284284	1.30599206386948e-15\\
3.84684684684685	9.30933239417303e-16\\
3.85085085085085	6.62522780004033e-16\\
3.85485485485485	4.70746787380384e-16\\
3.85885885885886	3.33947485339823e-16\\
3.86286286286286	2.3652292762572e-16\\
3.86686686686687	1.67252478630641e-16\\
3.87087087087087	1.18079938969692e-16\\
3.87487487487487	8.32307696843035e-17\\
3.87887887887888	5.85727922238145e-17\\
3.88288288288288	4.11540104621961e-17\\
3.88688688688689	2.88690578863121e-17\\
3.89089089089089	2.02188878716103e-17\\
3.89489489489489	1.4137939343756e-17\\
3.8988988988989	9.87004551256222e-18\\
3.9029029029029	6.8794921692149e-18\\
3.90690690690691	4.78737870240724e-18\\
3.91091091091091	3.32616168866239e-18\\
3.91491491491491	2.30724171262509e-18\\
3.91891891891892	1.59789040969706e-18\\
3.92292292292292	1.10485436789678e-18\\
3.92692692692693	7.62723677678068e-19\\
3.93093093093093	5.25694652205692e-19\\
3.93493493493493	3.6174623680573e-19\\
3.93893893893894	2.48529880575325e-19\\
3.94294294294294	1.70473673846501e-19\\
3.94694694694695	1.16745498649698e-19\\
3.95095095095095	7.98228218651696e-20\\
3.95495495495495	5.44901635623766e-20\\
3.95895895895896	3.71375494504263e-20\\
3.96296296296296	2.52704203311808e-20\\
3.96696696696697	1.71678446130463e-20\\
3.97097097097097	1.16445621121901e-20\\
3.97497497497497	7.88559783009868e-21\\
3.97897897897898	5.33150910515314e-21\\
3.98298298298298	3.59889975599746e-21\\
3.98698698698699	2.42545601315909e-21\\
3.99099099099099	1.63200358636179e-21\\
3.99499499499499	1.09635919285752e-21\\
3.998998998999	7.35340792478213e-22\\
4.003003003003	4.92411846533748e-22\\
4.00700700700701	3.29209500051919e-22\\
4.01101101101101	2.19745633137224e-22\\
4.01501501501502	1.46444198110179e-22\\
4.01901901901902	9.74379425719875e-23\\
4.02302302302302	6.47273837158399e-23\\
4.02702702702703	4.29291223378319e-23\\
4.03103103103103	2.84262761008828e-23\\
4.03503503503504	1.87928189377789e-23\\
4.03903903903904	1.24041749721908e-23\\
4.04304304304304	8.17424834715836e-24\\
4.04704704704705	5.37813572872081e-24\\
4.05105105105105	3.53280502947862e-24\\
4.05505505505506	2.31692277546167e-24\\
4.05905905905906	1.51707628767467e-24\\
4.06306306306306	9.91761512277604e-25\\
4.06706706706707	6.4730810523844e-25\\
4.07107107107107	4.21811873091671e-25\\
4.07507507507508	2.74429321140821e-25\\
4.07907907907908	1.78256828277099e-25\\
4.08308308308308	1.15602116850753e-25\\
4.08708708708709	7.48495757469659e-26\\
4.09109109109109	4.83856790052796e-26\\
4.0950950950951	3.12282950228049e-26\\
4.0990990990991	2.0122578962626e-26\\
4.1031031031031	1.2945622427456e-26\\
4.10710710710711	8.31507475059893e-27\\
4.11111111111111	5.33228435919208e-27\\
4.11511511511512	3.41400665481694e-27\\
4.11911911911912	2.18232455980371e-27\\
4.12312312312312	1.39276623161878e-27\\
4.12712712712713	8.87444137551042e-28\\
4.13113113113113	5.64556907741702e-28\\
4.13513513513514	3.58573613455334e-28\\
4.13913913913914	2.27380296125979e-28\\
4.14314314314314	1.43956481187491e-28\\
4.14714714714715	9.09941575385221e-29\\
4.15115115115115	5.74248253196429e-29\\
4.15515515515516	3.61817644701881e-29\\
4.15915915915916	2.27605988032124e-29\\
4.16316316316316	1.42949125861308e-29\\
4.16716716716717	8.96361511446163e-30\\
4.17117117117117	5.61162665509886e-30\\
4.17517517517518	3.50750427182205e-30\\
4.17917917917918	2.18882739132751e-30\\
4.18318318318318	1.36373091364884e-30\\
4.18718718718719	8.48300488827355e-31\\
4.19119119119119	5.26834985875361e-31\\
4.1951951951952	3.26665563262946e-31\\
4.1991991991992	2.02225516559868e-31\\
4.2032032032032	1.24989175521162e-31\\
4.20720720720721	7.71281120733074e-32\\
4.21121121121121	4.75178567239493e-32\\
4.21521521521522	2.92283856602953e-32\\
4.21921921921922	1.79496776849556e-32\\
4.22322322322322	1.10055643505837e-32\\
4.22722722722723	6.73708172297935e-33\\
4.23123123123123	4.1175145845079e-33\\
4.23523523523524	2.51247822755154e-33\\
4.23923923923924	1.5306408356747e-33\\
4.24324324324324	9.30996631672123e-34\\
4.24724724724725	5.65362172256992e-34\\
4.25125125125125	3.42775051764123e-34\\
4.25525525525526	2.07489193125901e-34\\
4.25925925925926	1.25396546220046e-34\\
4.26326326326326	7.56622901238006e-35\\
4.26726726726727	4.55803023398588e-35\\
4.27127127127127	2.74143990899073e-35\\
4.27527527527528	1.64620557377921e-35\\
4.27927927927928	9.86945589310071e-36\\
4.28328328328328	5.90753300742193e-36\\
4.28728728728729	3.53039179301755e-36\\
4.29129129129129	2.10641264138508e-36\\
4.2952952952953	1.25478060651591e-36\\
4.2992992992993	7.46269923386395e-37\\
4.3033033033033	4.43126654046141e-37\\
4.30730730730731	2.62702133801827e-37\\
4.31131131131131	1.55490234015844e-37\\
4.31531531531532	9.18853813735027e-38\\
4.31931931931932	5.4211760834369e-38\\
4.32332332332332	3.19333432904017e-38\\
4.32732732732733	1.87801496545261e-38\\
4.33133133133133	1.10270029007988e-38\\
4.33533533533534	6.46427328271343e-39\\
4.33933933933934	3.78343002076408e-39\\
4.34334334334334	2.21083066037159e-39\\
4.34734734734735	1.28981976162385e-39\\
4.35135135135135	7.51287818359011e-40\\
4.35535535535536	4.36905426502707e-40\\
4.35935935935936	2.5367186401232e-40\\
4.36336336336336	1.47048612020188e-40\\
4.36736736736737	8.51046569726288e-41\\
4.37137137137137	4.9175579509964e-41\\
4.37537537537538	2.83693516845178e-41\\
4.37937937937938	1.63400393382646e-41\\
4.38338338338338	9.39637942028324e-42\\
4.38738738738739	5.39475531400561e-42\\
4.39139139139139	3.09233628030923e-42\\
4.3953953953954	1.76972357562429e-42\\
4.3993993993994	1.01117869848359e-42\\
4.4034034034034	5.76838370847684e-43\\
4.40740740740741	3.28536876167313e-43\\
4.41141141141141	1.86817629156517e-43\\
4.41541541541542	1.06060907531537e-43\\
4.41941941941942	6.01168989608524e-44\\
4.42342342342342	3.40205670785241e-44\\
4.42742742742743	1.92216326407283e-44\\
4.43143143143143	1.08428323304466e-44\\
4.43543543543544	6.10659270886776e-45\\
4.43943943943944	3.43367281451515e-45\\
4.44344344344344	1.92762524990558e-45\\
4.44744744744745	1.08041360163833e-45\\
4.45145145145145	6.04590360402801e-46\\
4.45545545545546	3.37781711460728e-46\\
4.45945945945946	1.88414699454594e-46\\
4.46346346346346	1.0492939471633e-46\\
4.46746746746747	5.83422653235825e-47\\
4.47147147147147	3.23871805106842e-47\\
4.47547547547548	1.79500940409149e-47\\
4.47947947947948	9.93262515479779e-48\\
4.48348348348348	5.48738073655336e-48\\
4.48748748748749	3.02670345232317e-48\\
4.49149149149149	1.66678037518954e-48\\
4.4954954954955	9.16411672425858e-49\\
4.4994994994995	5.03044694084165e-49\\
4.5035035035035	2.75693329696584e-49\\
4.50750750750751	1.50851512652706e-49\\
4.51151151151151	8.24094098273004e-50\\
4.51551551551552	4.49477194490034e-50\\
4.51951951951952	2.44761012453502e-50\\
4.52352352352352	1.33070117906237e-50\\
4.52752752752753	7.2230822907296e-51\\
4.53153153153153	3.91442794493015e-51\\
4.53553553553554	2.11796004177595e-51\\
4.53953953953954	1.14411831867173e-51\\
4.54354354354354	6.17060622800634e-52\\
4.54754754754755	3.32267943369902e-52\\
4.55155155155155	1.7862932985156e-52\\
4.55555555555556	9.58784068846767e-53\\
4.55955955955956	5.13798182125096e-53\\
4.56356356356356	2.74895753257587e-53\\
4.56756756756757	1.46840958408183e-53\\
4.57157157157157	7.83123221153045e-54\\
4.57557557557558	4.16981420287613e-54\\
4.57957957957958	2.2167004883319e-54\\
4.58358358358358	1.17652467296777e-54\\
4.58758758758759	6.2344589595379e-55\\
4.59159159159159	3.2983763648692e-55\\
4.5955955955956	1.74222947457683e-55\\
4.5995995995996	9.18785664894933e-56\\
4.6036036036036	4.83756452416594e-56\\
4.60760760760761	2.54298053851888e-56\\
4.61161161161161	1.33463652897761e-56\\
4.61561561561562	6.99337283920853e-57\\
4.61961961961962	3.65859268297264e-57\\
4.62362362362362	1.91093160651152e-57\\
4.62762762762763	9.96506024079235e-58\\
4.63163163163163	5.18822037931012e-58\\
4.63563563563564	2.69687367216454e-58\\
4.63963963963964	1.39960824677823e-58\\
4.64364364364364	7.25197139427569e-59\\
4.64764764764765	3.75153818879719e-59\\
4.65165165165165	1.93760996954183e-59\\
4.65565565565566	9.99141614414631e-60\\
4.65965965965966	5.14388716489625e-60\\
4.66366366366366	2.64398822580501e-60\\
4.66766766766767	1.35684830803715e-60\\
4.67167167167167	6.95195215877376e-61\\
4.67567567567568	3.55619812264679e-61\\
4.67967967967968	1.81622152200525e-61\\
4.68368368368368	9.26094665021926e-62\\
4.68768768768769	4.71460857578841e-62\\
4.69169169169169	2.39629121554511e-62\\
4.6956956956957	1.21601027174896e-62\\
4.6996996996997	6.16082091494403e-63\\
4.7037037037037	3.11633128105237e-63\\
4.70770770770771	1.57381016485101e-63\\
4.71171171171171	7.93532585228144e-64\\
4.71571571571572	3.99466970553272e-64\\
4.71971971971972	2.00770859539185e-64\\
4.72372372372372	1.00745155501556e-64\\
4.72772772772773	5.04720977337084e-65\\
4.73173173173173	2.52453984370929e-65\\
4.73573573573574	1.26071463151389e-65\\
4.73973973973974	6.2857202186354e-66\\
4.74374374374374	3.1289382232753e-66\\
4.74774774774775	1.55504377697358e-66\\
4.75175175175175	7.71599502404732e-67\\
4.75575575575576	3.82247757408745e-67\\
4.75975975975976	1.89060865197573e-67\\
4.76376376376376	9.33602536922173e-68\\
4.76776776776777	4.60284226460287e-68\\
4.77177177177177	2.26565536037656e-68\\
4.77577577577578	1.1134360793936e-68\\
4.77977977977978	5.4631159464214e-69\\
4.78378378378378	2.67620409829442e-69\\
4.78778778778779	1.30888572235899e-69\\
4.79179179179179	6.39128103019056e-70\\
4.7957957957958	3.11585890197864e-70\\
4.7997997997998	1.51660099300755e-70\\
4.8038038038038	7.37001841869733e-71\\
4.80780780780781	3.57576932376675e-71\\
4.81181181181181	1.73210453555405e-71\\
4.81581581581582	8.37688178730899e-72\\
4.81981981981982	4.04477490463526e-72\\
4.82382382382382	1.94988954832844e-72\\
4.82782782782783	9.38489347738019e-73\\
4.83183183183183	4.50974894789413e-73\\
4.83583583583584	2.16361042944681e-73\\
4.83983983983984	1.03635715062125e-73\\
4.84384384384384	4.95613938423083e-74\\
4.84784784784785	2.36636240245869e-74\\
4.85185185185185	1.12803527870548e-74\\
4.85585585585586	5.36868333770758e-75\\
4.85985985985986	2.55103593065715e-75\\
4.86386386386386	1.21023314471513e-75\\
4.86786786786787	5.7322508858395e-76\\
4.87187187187187	2.71072222923242e-76\\
4.87587587587588	1.27981891337935e-76\\
4.87987987987988	6.03275530334174e-77\\
4.88388388388388	2.83913847698198e-77\\
4.88788788788789	1.33401624974293e-77\\
4.89189189189189	6.25805452740404e-78\\
4.8958958958959	2.93103662077008e-78\\
4.8998998998999	1.37058768317794e-78\\
4.9039039039039	6.39876385601523e-79\\
4.90790790790791	2.98255883371421e-79\\
4.91191191191191	1.38798775072527e-79\\
4.91591591591592	6.44890419568388e-80\\
4.91991991991992	2.9915061110081e-80\\
4.92392392392392	1.3854713223812e-80\\
4.92792792792793	6.40632348476734e-81\\
4.93193193193193	2.95749395022729e-81\\
4.93593593593594	1.36314652942965e-81\\
4.93993993993994	6.2728499060153e-82\\
4.94394394394394	2.8819797592737e-82\\
4.94794794794795	1.32196705734494e-82\\
4.95195195195195	6.05416169013016e-83\\
4.95595595595596	2.76815926983121e-83\\
4.95995995995996	1.26366456866919e-83\\
4.96396396396396	5.75938642959432e-84\\
4.96796796796797	2.62074221488027e-84\\
4.97197197197197	1.19062792786225e-84\\
4.97597597597598	5.40046929896556e-85\\
4.97997997997998	2.44562920949223e-85\\
4.98398398398398	1.10574096379471e-85\\
4.98798798798799	4.99137109821324e-86\\
4.99199199199199	2.24952070915605e-86\\
4.995995995996	1.01219411014876e-86\\
5	4.54717104180649e-87\\
};
\addlegendentry{$\Pr(\theta \given x, \sigma_{\text{spike}}^2, \sigma_{\text{slab}}^2, \omega^2)$};

\end{axis}
\end{tikzpicture}%
  \caption{The posterior density for the parameter $\theta$ given
    the example observation $x = 3$, $\omega^2 = 0.1^2$.}
\end{figure}

\end{document}
