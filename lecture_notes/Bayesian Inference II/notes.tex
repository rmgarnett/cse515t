\documentclass{article}

\usepackage[T1]{fontenc}
\usepackage[osf]{libertine}
\usepackage[scaled=0.8]{beramono}
\usepackage[margin=1.5in]{geometry}
\usepackage{url}
\usepackage{booktabs}
\usepackage{amsmath}
\usepackage{amssymb}
\usepackage{nicefrac}
\usepackage{microtype}

\usepackage{sectsty}
\sectionfont{\large}
\subsectionfont{\normalsize}

\usepackage{titlesec}
\titlespacing{\section}{0pt}{10pt plus 2pt minus 2pt}{0pt plus 2pt minus 0pt}
\titlespacing{\subsection}{0pt}{5pt plus 2pt minus 2pt}{0pt plus 2pt minus 0pt}
\titlespacing{\subsubsection}{0pt}{5pt plus 2pt minus 2pt}{0pt plus 2pt minus 0pt}

\usepackage{pgfplots}
\pgfplotsset{
  compat=newest,
  plot coordinates/math parser=false,
  tick label style={font=\footnotesize, /pgf/number format/fixed},
  label style={font=\small},
  legend style={font=\small},
  every axis/.append style={
    tick align=outside,
    clip mode=individual,
    scaled ticks=false,
    thick,
    tick style={semithick, black}
  }
}

\pgfkeys{/pgf/number format/.cd, set thousands separator={\,}}

\usepgfplotslibrary{external}
\tikzexternalize[prefix=tikz/]

\newlength\figurewidth
\newlength\figureheight

\setlength{\figurewidth}{8cm}
\setlength{\figureheight}{6cm}

\setlength{\parindent}{0pt}
\setlength{\parskip}{1ex}

\newcommand{\acro}[1]{\textsc{\MakeLowercase{#1}}}
\newcommand{\given}{\mid}
\newcommand{\mc}[1]{\mathcal{#1}}
\newcommand{\data}{\mc{D}}
\newcommand{\intd}[1]{\,\mathrm{d}{#1}}
\newcommand{\inv}{^{-1}}
\newcommand{\E}{\mathbb{E}}
\newcommand{\R}{\mathbb{R}}
\newcommand{\ci}{\text{CI}}

\DeclareMathOperator{\var}{var}
\DeclareMathOperator*{\argmin}{arg\,min}
\DeclareMathOperator*{\argmax}{arg\,max}

\begin{document}

\section*{Hypothesis testing}

Let's return to the issue of hypothesis testing. Suppose we are reasoning about
a parameter $\theta$ in light of data $\data$, and wish to consider a hypothesis
$\theta \in \mc{H}$, where $\mc{H} \subseteq \Theta$ is some set of possible
values for this parameter.

We have seen that the Bayesian approach to hypothesis testing is
straightforward. We first derive the posterior distribution $p(\theta \given
\data)$ and then may compute the probability of the hypothesis directly:
\[
  \Pr(\theta \in \mc{H} \given \data)
  =
  \int_{\mc{H}} p(\theta \given \data) \intd \theta.
\]

Let's consider an explicit example. Suppose we are interested in the unknown
bias of a coin $\theta \in (0, 1)$, and begin with the uniform prior on the
interval $(0, 1)$:
\[
  p(\theta) = \mc{U}\bigl(\theta; 0, 1\bigr)
            = \mc{B}(\theta; \alpha = 1, \beta = 1).
\]
Let's collect some data to further inform our belief about $\theta$. Suppose we
flip the coin independently $n = 50$ times and observe $x = 30$ heads. After
gathering this data, we wish to consider the natural question of whether the
coin is fair: that is, whether $\theta = \nicefrac{1}{2}$.

From the developments in the last lecture, we can compute the posterior
distribution easily. It is an updated beta distribution:
\[
  p(\theta \given \data) = \mc{B}(\theta; 31, 21).
\]
We may now compute the posterior probability of the hypothesis that the coin is
fair:
\[
  \Pr(\theta = \nicefrac{1}{2} \given \data)
  =
  \int_{\nicefrac{1}{2}}^{\nicefrac{1}{2}}
  p(\theta \given \data)
  \intd\theta
  =
  0.
\]
The posterior probability of the coin being \emph{exactly} fair is zero! This
should not be surprising, as suggesting that we could possibly know the bias of
the coin with infinite precision is unfathomable.

We may however relax the question a bit to get some more insight. One option
would be to consider a parameterized family of hypotheses of the form
\[
  \mc{H}(\varepsilon) = (\nicefrac{1}{2} - \varepsilon, \nicefrac{1}{2} + \varepsilon).
\]
Thus a high probability of the hypothesis $\mc{H}(\varepsilon)$ corresponds to
the notion that the coin is ``near fair'' with an allowed error of
$\varepsilon$. We may then compute the posterior probability of these hypotheses
and consider how they vary as a function of $\varepsilon$. Figure
\ref{near_fair_probabilities} shows the results for the coin-flipping example
above. We can see that there's approximately a 50\% posterior probability that
the bias of the coin is in the interval $(0.4, 0.6)$, corresponding to
$\varepsilon = 0.1$. We also have evidence to conclude $\theta \in (0.25, 0.75)$
with near certainty. These probabilities help constrain exactly how ``fair'' or
``not fair'' we believe the coin to be in light of our evidence.

\begin{figure}
  \centering
  % This file was created by matlab2tikz.
%
\tikzsetnextfilename{near_fair_probabilities}
\definecolor{mycolor1}{rgb}{0.12157,0.47059,0.70588}%
%
\begin{tikzpicture}

\begin{axis}[%
width=\figurewidth,
height=\figureheight,
at={(0\figurewidth,0\figureheight)},
scale only axis,
xmin=0,
xmax=0.5,
tick align=outside,
xlabel={$\varepsilon$},
ymin=0,
ymax=1.01,
axis background/.style={fill=white},
axis x line*=bottom,
axis y line*=left,
legend style={at={(0.97,0.03)},anchor=south east,legend cell align=left,align=left,fill=none,draw=none}
]
\addplot [color=mycolor1,solid]
  table[row sep=crcr]{%
0	0\\
0.0005005005005005	0.00213697140884687\\
0.001001001001001	0.00427404987295148\\
0.0015015015015015	0.00641134242182861\\
0.002002002002002	0.00854895603350765\\
0.0025025025025025	0.0106869976088049\\
0.003003003003003	0.0128255739456147\\
0.0035035035035035	0.0149647917132127\\
0.004004004004004	0.0171047574265903\\
0.0045045045045045	0.0192455774208287\\
0.005005005005005	0.0213873578254952\\
0.00550550550550551	0.0235302045390896\\
0.00600600600600601	0.0256742232035411\\
0.00650650650650651	0.0278195191787526\\
0.00700700700700701	0.0299661975172035\\
0.00750750750750751	0.0321143629386215\\
0.00800800800800801	0.0342641198047118\\
0.00850850850850851	0.0364155720939722\\
0.00900900900900901	0.0385688233765821\\
0.00950950950950951	0.0407239767893682\\
0.01001001001001	0.0428811350108746\\
0.0105105105105105	0.0450404002365156\\
0.011011011011011	0.0472018741538255\\
0.0115115115115115	0.0493656579178317\\
0.012012012012012	0.051531852126511\\
0.0125125125125125	0.0537005567963851\\
0.013013013013013	0.0558718713382197\\
0.0135135135135135	0.0580458945328542\\
0.014014014014014	0.0602227245071642\\
0.0145145145145145	0.0624024587101527\\
0.015015015015015	0.0645851938891866\\
0.0155155155155155	0.0667710260663737\\
0.016016016016016	0.0689600505150936\\
0.0165165165165165	0.0711523617366828\\
0.017017017017017	0.0733480534372694\\
0.0175175175175175	0.0755472185047922\\
0.018018018018018	0.0777499489861687\\
0.0185185185185185	0.0799563360646448\\
0.019019019019019	0.0821664700373327\\
0.0195195195195195	0.0843804402929101\\
0.02002002002002	0.0865983352895353\\
0.0205205205205205	0.088820242532924\\
0.021021021021021	0.0910462485546552\\
0.0215215215215215	0.09327643889065\\
0.022022022022022	0.0955108980598801\\
0.0225225225225225	0.0977497095432691\\
0.023023023023023	0.0999929557628252\\
0.0235235235235235	0.102240718060976\\
0.024024024024024	0.104493076680145\\
0.0245245245245245	0.106750110742545\\
0.025025025025025	0.109011898230209\\
0.0255255255255255	0.111278515965255\\
0.026026026026026	0.113550039590394\\
0.0265265265265265	0.115826543549685\\
0.027027027027027	0.118108101069537\\
0.0275275275275275	0.120394784139967\\
0.028028028028028	0.122686663496106\\
0.0285285285285285	0.124983808599985\\
0.029029029029029	0.127286287622566\\
0.0295295295295295	0.129594167426063\\
0.03003003003003	0.131907513546513\\
0.0305305305305305	0.134226390176647\\
0.031031031031031	0.13655086014902\\
0.0315315315315315	0.138880984919447\\
0.032032032032032	0.141216824550708\\
0.0325325325325325	0.143558437696558\\
0.033033033033033	0.145905881586022\\
0.0335335335335335	0.148259212007994\\
0.034034034034034	0.150618483296137\\
0.0345345345345345	0.152983748314089\\
0.035035035035035	0.155355058440978\\
0.0355355355355355	0.157732463557236\\
0.036036036036036	0.160116012030753\\
0.0365365365365365	0.162505750703321\\
0.037037037037037	0.164901724877424\\
0.0375375375375375	0.167303978303331\\
0.038038038038038	0.169712553166527\\
0.0385385385385385	0.172127490075471\\
0.039039039039039	0.174548828049689\\
0.0395395395395395	0.176976604508193\\
0.04004004004004	0.179410855258252\\
0.0405405405405405	0.181851614484483\\
0.041041041041041	0.184298914738313\\
0.0415415415415415	0.186752786927756\\
0.042042042042042	0.189213260307552\\
0.0425425425425425	0.191680362469664\\
0.043043043043043	0.194154119334111\\
0.0435435435435435	0.196634555140157\\
0.044044044044044	0.199121692437874\\
0.0445445445445445	0.201615552080035\\
0.045045045045045	0.204116153214398\\
0.0455455455455455	0.206623513276318\\
0.046046046046046	0.209137647981755\\
0.0465465465465465	0.211658571320629\\
0.047047047047047	0.214186295550536\\
0.0475475475475475	0.216720831190862\\
0.048048048048048	0.219262187017227\\
0.0485485485485486	0.221810370056331\\
0.049049049049049	0.224365385581166\\
0.0495495495495495	0.226927237106587\\
0.0500500500500501	0.229495926385279\\
0.0505505505505505	0.232071453404085\\
0.0510510510510511	0.234653816380721\\
0.0515515515515515	0.237243011760859\\
0.0520520520520521	0.2398390342156\\
0.0525525525525526	0.242441876639315\\
0.0530530530530531	0.245051530147876\\
0.0535535535535536	0.247667984077266\\
0.0540540540540541	0.250291225982565\\
0.0545545545545546	0.252921241637322\\
0.0550550550550551	0.255558015033307\\
0.0555555555555556	0.25820152838065\\
0.0560560560560561	0.260851762108359\\
0.0565565565565566	0.26350869486521\\
0.0570570570570571	0.266172303521049\\
0.0575575575575576	0.268842563168442\\
0.0580580580580581	0.271519447124731\\
0.0585585585585586	0.274202926934459\\
0.0590590590590591	0.276892972372187\\
0.0595595595595596	0.279589551445681\\
0.0600600600600601	0.282292630399491\\
0.0605605605605606	0.2850021737189\\
0.0610610610610611	0.287718144134262\\
0.0615615615615616	0.290440502625715\\
0.0620620620620621	0.293169208428256\\
0.0625625625625626	0.295904219037234\\
0.0630630630630631	0.298645490214157\\
0.0635635635635636	0.301392975992946\\
0.0640640640640641	0.3041466286865\\
0.0645645645645646	0.30690639889367\\
0.0650650650650651	0.30967223550658\\
0.0655655655655656	0.312444085718343\\
0.0660660660660661	0.315221895031129\\
0.0665665665665666	0.318005607264596\\
0.0670670670670671	0.320795164564682\\
0.0675675675675676	0.323590507412793\\
0.0680680680680681	0.326391574635303\\
0.0685685685685686	0.329198303413448\\
0.0690690690690691	0.332010629293563\\
0.0695695695695696	0.334828486197662\\
0.0700700700700701	0.337651806434417\\
0.0705705705705706	0.340480520710405\\
0.0710710710710711	0.343314558141794\\
0.0715715715715716	0.346153846266303\\
0.0720720720720721	0.348998311055543\\
0.0725725725725726	0.351847876927678\\
0.0730730730730731	0.354702466760444\\
0.0735735735735736	0.357562001904473\\
0.0740740740740741	0.360426402196974\\
0.0745745745745746	0.363295585975728\\
0.0750750750750751	0.366169470093409\\
0.0755755755755756	0.36904796993224\\
0.0760760760760761	0.371930999418943\\
0.0765765765765766	0.37481847104003\\
0.0770770770770771	0.377710295857395\\
0.0775775775775776	0.380606383524193\\
0.0780780780780781	0.383506642301072\\
0.0785785785785786	0.386410979072672\\
0.0790790790790791	0.389319299364405\\
0.0795795795795796	0.39223150735959\\
0.0800800800800801	0.395147505916803\\
0.0805805805805806	0.398067196587575\\
0.0810810810810811	0.400990479634337\\
0.0815815815815816	0.403917254048643\\
0.0820820820820821	0.406847417569702\\
0.0825825825825826	0.409780866703122\\
0.0830830830830831	0.412717496739971\\
0.0835835835835836	0.415657201776072\\
0.0840840840840841	0.418599874731558\\
0.0845845845845846	0.421545407370691\\
0.0850850850850851	0.424493690321908\\
0.0855855855855856	0.427444613098116\\
0.0860860860860861	0.430398064117256\\
0.0865865865865866	0.433353930723028\\
0.0870870870870871	0.436312099205946\\
0.0875875875875876	0.439272454824508\\
0.0880880880880881	0.442234881826674\\
0.0885885885885886	0.4451992634715\\
0.0890890890890891	0.448165482051006\\
0.0895895895895896	0.451133418912248\\
0.0900900900900901	0.454102954479575\\
0.0905905905905906	0.457073968277102\\
0.0910910910910911	0.460046338951326\\
0.0915915915915916	0.463019944293987\\
0.0920920920920921	0.465994661265051\\
0.0925925925925926	0.468970366015904\\
0.0930930930930931	0.471946933912672\\
0.0935935935935936	0.474924239559747\\
0.0940940940940941	0.477902156823432\\
0.0945945945945946	0.480880558855774\\
0.0950950950950951	0.483859318118466\\
0.0955955955955956	0.486838306406999\\
0.0960960960960961	0.489817394874842\\
0.0965965965965966	0.492796454057804\\
0.0970970970970971	0.495775353898519\\
0.0975975975975976	0.49875396377101\\
0.0980980980980981	0.501732152505399\\
0.0985985985985986	0.504709788412711\\
0.0990990990990991	0.507686739309746\\
0.0995995995995996	0.510662872544096\\
0.1001001001001	0.513638055019206\\
0.100600600600601	0.516612153219531\\
0.101101101101101	0.519585033235758\\
0.101601601601602	0.522556560790098\\
0.102102102102102	0.525526601261661\\
0.102602602602603	0.528495019711848\\
0.103103103103103	0.531461680909824\\
0.103603603603604	0.534426449358016\\
0.104104104104104	0.537389189317661\\
0.104604604604605	0.540349764834373\\
0.105105105105105	0.54330803976375\\
0.105605605605606	0.546263877796979\\
0.106106106106106	0.549217142486473\\
0.106606606606607	0.5521676972715\\
0.107107107107107	0.555115405503836\\
0.107607607607608	0.558060130473375\\
0.108108108108108	0.561001735433752\\
0.108608608608609	0.563940083627958\\
0.109109109109109	0.566875038313899\\
0.10960960960961	0.569806462789946\\
0.11011011011011	0.572734220420447\\
0.110610610610611	0.575658174661201\\
0.111111111111111	0.57857818908485\\
0.111611611611612	0.581494127406267\\
0.112112112112112	0.584405853507853\\
0.112612612612613	0.58731323146476\\
0.113113113113113	0.590216125570061\\
0.113613613613614	0.593114400359838\\
0.114114114114114	0.596007920638162\\
0.114614614614615	0.598896551502018\\
0.115115115115115	0.601780158366105\\
0.115615615615616	0.604658606987542\\
0.116116116116116	0.607531763490463\\
0.116616616616617	0.61039949439051\\
0.117117117117117	0.613261666619184\\
0.117617617617618	0.616118147548081\\
0.118118118118118	0.618968805013008\\
0.118618618618619	0.621813507337938\\
0.119119119119119	0.624652123358824\\
0.11961961961962	0.627484522447294\\
0.12012012012012	0.630310574534148\\
0.120620620620621	0.63313015013273\\
0.121121121121121	0.635943120362112\\
0.121621621621622	0.638749356970115\\
0.122122122122122	0.64154873235616\\
0.122622622622623	0.644341119593918\\
0.123123123123123	0.647126392453797\\
0.123623623623624	0.649904425425211\\
0.124124124124124	0.652675093738674\\
0.124624624624625	0.655438273387678\\
0.125125125125125	0.658193841150359\\
0.125625625625626	0.660941674610975\\
0.126126126126126	0.663681652181121\\
0.126626626626627	0.666413653120767\\
0.127127127127127	0.669137557559053\\
0.127627627627628	0.671853246514829\\
0.128128128128128	0.674560601916985\\
0.128628628628629	0.677259506624541\\
0.129129129129129	0.679949844446465\\
0.12962962962963	0.682631500161266\\
0.13013013013013	0.685304359536309\\
0.130630630630631	0.687968309346899\\
0.131131131131131	0.69062323739507\\
0.131631631631632	0.693269032528139\\
0.132132132132132	0.695905584656945\\
0.132632632632633	0.698532784773856\\
0.133133133133133	0.701150524970477\\
0.133633633633634	0.703758698455062\\
0.134134134134134	0.706357199569655\\
0.134634634634635	0.708945923806945\\
0.135135135135135	0.711524767826812\\
0.135635635635636	0.714093629472558\\
0.136136136136136	0.716652407786899\\
0.136636636636637	0.719201003027571\\
0.137137137137137	0.721739316682689\\
0.137637637637638	0.724267251485769\\
0.138138138138138	0.72678471143043\\
0.138638638638639	0.729291601784808\\
0.139139139139139	0.731787829105601\\
0.13963963963964	0.734273301251846\\
0.14014014014014	0.736747927398329\\
0.140640640640641	0.739211618048686\\
0.141141141141141	0.741664285048168\\
0.141641641641642	0.744105841596076\\
0.142142142142142	0.746536202257853\\
0.142642642642643	0.748955282976848\\
0.143143143143143	0.75136300108574\\
0.143643643643644	0.753759275317605\\
0.144144144144144	0.756144025816662\\
0.144644644644645	0.75851717414866\\
0.145145145145145	0.760878643310916\\
0.145645645645646	0.763228357742016\\
0.146146146146146	0.765566243331158\\
0.146646646646647	0.767892227427142\\
0.147147147147147	0.770206238847016\\
0.147647647647648	0.772508207884357\\
0.148148148148148	0.774798066317209\\
0.148648648648649	0.777075747415659\\
0.149149149149149	0.77934118594905\\
0.14964964964965	0.781594318192858\\
0.15015015015015	0.783835081935182\\
0.150650650650651	0.786063416482904\\
0.151151151151151	0.788279262667465\\
0.151651651651652	0.790482562850305\\
0.152152152152152	0.792673260927927\\
0.152652652652653	0.794851302336604\\
0.153153153153153	0.797016634056741\\
0.153653653653654	0.799169204616855\\
0.154154154154154	0.801308964097211\\
0.154654654654655	0.803435864133102\\
0.155155155155155	0.805549857917759\\
0.155655655655656	0.807650900204906\\
0.156156156156156	0.809738947310972\\
0.156656656656657	0.811813957116919\\
0.157157157157157	0.813875889069744\\
0.157657657657658	0.815924704183612\\
0.158158158158158	0.817960365040628\\
0.158658658658659	0.819982835791271\\
0.159159159159159	0.821992082154475\\
0.15965965965966	0.823988071417343\\
0.16016016016016	0.82597077243454\\
0.160660660660661	0.827940155627318\\
0.161161161161161	0.829896192982201\\
0.161661661661662	0.831838858049329\\
0.162162162162162	0.833768125940463\\
0.162662662662663	0.835683973326641\\
0.163163163163163	0.837586378435506\\
0.163663663663664	0.839475321048291\\
0.164164164164164	0.841350782496476\\
0.164664664664665	0.843212745658111\\
0.165165165165165	0.845061194953812\\
0.165665665665666	0.846896116342426\\
0.166166166166166	0.848717497316382\\
0.166666666666667	0.850525326896706\\
0.167167167167167	0.852319595627739\\
0.167667667667668	0.854100295571518\\
0.168168168168168	0.855867420301854\\
0.168668668668669	0.857620964898107\\
0.169169169169169	0.859360925938639\\
0.16966966966967	0.861087301493984\\
0.17017017017017	0.862800091119691\\
0.170670670670671	0.864499295848899\\
0.171171171171171	0.866184918184603\\
0.171671671671672	0.867856962091626\\
0.172172172172172	0.869515432988317\\
0.172672672672673	0.871160337737959\\
0.173173173173173	0.872791684639899\\
0.173673673673674	0.874409483420406\\
0.174174174174174	0.876013745223247\\
0.174674674674675	0.877604482600016\\
0.175175175175175	0.87918170950017\\
0.175675675675676	0.880745441260838\\
0.176176176176176	0.882295694596348\\
0.176676676676677	0.883832487587516\\
0.177177177177177	0.885355839670689\\
0.177677677677678	0.886865771626529\\
0.178178178178178	0.88836230556859\\
0.178678678678679	0.889845464931618\\
0.179179179179179	0.891315274459664\\
0.17967967967968	0.892771760193938\\
0.18018018018018	0.894214949460467\\
0.180680680680681	0.895644870857516\\
0.181181181181181	0.897061554242814\\
0.181681681681682	0.898465030720561\\
0.182182182182182	0.899855332628243\\
0.182682682682683	0.90123249352324\\
0.183183183183183	0.902596548169247\\
0.183683683683684	0.903947532522503\\
0.184184184184184	0.905285483717844\\
0.184684684684685	0.906610440054565\\
0.185185185185185	0.907922440982126\\
0.185685685685686	0.909221527085669\\
0.186186186186186	0.910507740071392\\
0.186686686686687	0.911781122751748\\
0.187187187187187	0.913041719030499\\
0.187687687687688	0.914289573887622\\
0.188188188188188	0.915524733364067\\
0.188688688688689	0.916747244546378\\
0.189189189189189	0.917957155551182\\
0.18968968968969	0.919154515509551\\
0.19019019019019	0.920339374551238\\
0.190690690690691	0.921511783788796\\
0.191191191191191	0.922671795301581\\
0.191691691691692	0.923819462119658\\
0.192192192192192	0.924954838207589\\
0.192692692692693	0.926077978448136\\
0.193193193193193	0.927188938625866\\
0.193693693693694	0.92828777541067\\
0.194194194194194	0.9293745463412\\
0.194694694694695	0.930449309808226\\
0.195195195195195	0.931512125037929\\
0.195695695695696	0.93256305207512\\
0.196196196196196	0.933602151766398\\
0.196696696696697	0.934629485743259\\
0.197197197197197	0.93564511640514\\
0.197697697697698	0.936649106902432\\
0.198198198198198	0.937641521119441\\
0.198698698698699	0.938622423657316\\
0.199199199199199	0.93959187981695\\
0.1996996996997	0.940549955581844\\
0.2002002002002	0.94149671760096\\
0.200700700700701	0.942432233171559\\
0.201201201201201	0.943356570222013\\
0.201701701701702	0.944269797294629\\
0.202202202202202	0.945171983528457\\
0.202702702702703	0.946063198642111\\
0.203203203203203	0.946943512916589\\
0.203703703703704	0.947812997178116\\
0.204204204204204	0.948671722780995\\
0.204704704704705	0.949519761590482\\
0.205205205205205	0.950357185965699\\
0.205705705705706	0.951184068742555\\
0.206206206206206	0.952000483216732\\
0.206706706706707	0.952806503126685\\
0.207207207207207	0.953602202636701\\
0.207707707707708	0.954387656320011\\
0.208208208208208	0.95516293914194\\
0.208708708708709	0.955928126443128\\
0.209209209209209	0.956683293922813\\
0.20970970970971	0.957428517622172\\
0.21021021021021	0.95816387390774\\
0.210710710710711	0.9588894394549\\
0.211211211211211	0.959605291231457\\
0.211711711711712	0.960311506481287\\
0.212212212212212	0.961008162708077\\
0.212712712712713	0.961695337659161\\
0.213213213213213	0.962373109309437\\
0.213713713713714	0.963041555845399\\
0.214214214214214	0.963700755649254\\
0.214714714714715	0.96435078728316\\
0.215215215215215	0.964991729473563\\
0.215715715715716	0.965623661095649\\
0.216216216216216	0.966246661157915\\
0.216716716716717	0.966860808786857\\
0.217217217217217	0.96746618321178\\
0.217717717717718	0.968062863749739\\
0.218218218218218	0.968650929790602\\
0.218718718718719	0.969230460782251\\
0.219219219219219	0.969801536215916\\
0.21971971971972	0.970364235611648\\
0.22022022022022	0.970918638503935\\
0.220720720720721	0.971464824427459\\
0.221221221221221	0.972002872902997\\
0.221721721721722	0.972532863423484\\
0.222222222222222	0.973054875440213\\
0.222722722722723	0.973568988349202\\
0.223223223223223	0.974075281477705\\
0.223723723723724	0.974573834070898\\
0.224224224224224	0.975064725278715\\
0.224724724724725	0.975548034142852\\
0.225225225225225	0.976023839583939\\
0.225725725725726	0.976492220388875\\
0.226226226226226	0.97695325519834\\
0.226726726726727	0.977407022494476\\
0.227227227227227	0.977853600588739\\
0.227727727727728	0.978293067609932\\
0.228228228228228	0.978725501492412\\
0.228728728728729	0.979150979964482\\
0.229229229229229	0.97956958053695\\
0.22972972972973	0.979981380491889\\
0.23023023023023	0.980386456871566\\
0.230730730730731	0.980784886467562\\
0.231231231231231	0.981176745810075\\
0.231731731731732	0.981562111157415\\
0.232232232232232	0.981941058485684\\
0.232732732732733	0.982313663478644\\
0.233233233233233	0.982680001517781\\
0.233733733733734	0.983040147672552\\
0.234234234234234	0.983394176690832\\
0.234734734734735	0.983742162989548\\
0.235235235235235	0.984084180645511\\
0.235735735735736	0.984420303386438\\
0.236236236236236	0.984750604582167\\
0.236736736736737	0.985075157236075\\
0.237237237237237	0.985394033976683\\
0.237737737737738	0.985707307049457\\
0.238238238238238	0.986015048308813\\
0.238738738738739	0.986317329210305\\
0.239239239239239	0.986614220803018\\
0.23973973973974	0.986905793722158\\
0.24024024024024	0.987192118181827\\
0.240740740740741	0.987473263968007\\
0.241241241241241	0.987749300431731\\
0.241741741741742	0.988020296482448\\
0.242242242242242	0.988286320581593\\
0.242742742742743	0.988547440736336\\
0.243243243243243	0.988803724493543\\
0.243743743743744	0.989055238933912\\
0.244244244244244	0.989302050666316\\
0.244744744744745	0.989544225822331\\
0.245245245245245	0.989781830050956\\
0.245745745745746	0.990014928513528\\
0.246246246246246	0.990243585878821\\
0.246746746746747	0.990467866318336\\
0.247247247247247	0.990687833501783\\
0.247747747747748	0.990903550592741\\
0.248248248248248	0.991115080244515\\
0.248748748748749	0.991322484596169\\
0.249249249249249	0.991525825268744\\
0.24974974974975	0.991725163361666\\
0.25025025025025	0.991920559449322\\
0.250750750750751	0.992112073577827\\
0.251251251251251	0.992299765261965\\
0.251751751751752	0.992483693482306\\
0.252252252252252	0.992663916682501\\
0.252752752752753	0.992840492766748\\
0.253253253253253	0.993013479097435\\
0.253753753753754	0.993182932492949\\
0.254254254254254	0.993348909225656\\
0.254754754754755	0.993511465020049\\
0.255255255255255	0.993670655051062\\
0.255755755755756	0.993826533942548\\
0.256256256256256	0.993979155765916\\
0.256756756756757	0.994128574038936\\
0.257257257257257	0.994274841724691\\
0.257757757757758	0.994418011230698\\
0.258258258258258	0.994558134408175\\
0.258758758758759	0.994695262551463\\
0.259259259259259	0.9948294463976\\
0.25975975975976	0.994960736126044\\
0.26026026026026	0.995089181358539\\
0.260760760760761	0.995214831159129\\
0.261261261261261	0.995337734034313\\
0.261761761761762	0.995457937933342\\
0.262262262262262	0.995575490248649\\
0.262762762762763	0.995690437816422\\
0.263263263263263	0.995802826917305\\
0.263763763763764	0.995912703277234\\
0.264264264264264	0.996020112068398\\
0.264764764764765	0.996125097910335\\
0.265265265265265	0.996227704871142\\
0.265765765765766	0.996327976468815\\
0.266266266266266	0.996425955672709\\
0.266766766766767	0.99652168490511\\
0.267267267267267	0.996615206042929\\
0.267767767767768	0.996706560419505\\
0.268268268268268	0.996795788826519\\
0.268768768768769	0.996882931516022\\
0.269269269269269	0.996968028202558\\
0.26976976976977	0.997051118065401\\
0.27027027027027	0.997132239750891\\
0.270770770770771	0.997211431374863\\
0.271271271271271	0.997288730525178\\
0.271771771771772	0.997364174264351\\
0.272272272272272	0.997437799132261\\
0.272772772772773	0.997509641148961\\
0.273273273273273	0.997579735817565\\
0.273773773773774	0.997648118127231\\
0.274274274274274	0.997714822556215\\
0.274774774774775	0.997779883075012\\
0.275275275275275	0.997843333149572\\
0.275775775775776	0.997905205744591\\
0.276276276276276	0.997965533326876\\
0.276776776776777	0.99802434786878\\
0.277277277277277	0.998081680851703\\
0.277777777777778	0.998137563269665\\
0.278278278278278	0.998192025632931\\
0.278778778778779	0.998245097971713\\
0.279279279279279	0.998296809839914\\
0.27977977977978	0.998347190318943\\
0.28028028028028	0.998396268021574\\
0.280780780780781	0.998444071095864\\
0.281281281281281	0.998490627229117\\
0.281781781781782	0.998535963651896\\
0.282282282282282	0.998580107142086\\
0.282782782782783	0.998623084028987\\
0.283283283283283	0.998664920197464\\
0.283783783783784	0.998705641092128\\
0.284284284284284	0.99874527172155\\
0.284784784784785	0.998783836662519\\
0.285285285285285	0.998821360064328\\
0.285785785785786	0.998857865653086\\
0.286286286286286	0.998893376736071\\
0.286786786786787	0.998927916206095\\
0.287287287287287	0.998961506545904\\
0.287787787787788	0.998994169832597\\
0.288288288288288	0.999025927742067\\
0.288788788788789	0.999056801553459\\
0.289289289289289	0.999086812153645\\
0.28978978978979	0.999115980041716\\
0.29029029029029	0.999144325333486\\
0.290790790790791	0.999171867766008\\
0.291291291291291	0.999198626702101\\
0.291791791791792	0.99922462113488\\
0.292292292292292	0.999249869692297\\
0.292792792792793	0.999274390641687\\
0.293293293293293	0.999298201894312\\
0.293793793793794	0.999321321009912\\
0.294294294294294	0.999343765201251\\
0.294794794794795	0.999365551338664\\
0.295295295295295	0.999386695954598\\
0.295795795795796	0.999407215248152\\
0.296296296296296	0.999427125089607\\
0.296796796796797	0.999446441024948\\
0.297297297297297	0.999465178280378\\
0.297797797797798	0.999483351766823\\
0.298298298298298	0.999500976084421\\
0.298798798798799	0.999518065527002\\
0.299299299299299	0.999534634086546\\
0.2997997997998	0.999550695457636\\
0.3003003003003	0.999566263041883\\
0.300800800800801	0.999581349952339\\
0.301301301301301	0.999595969017891\\
0.301801801801802	0.999610132787626\\
0.302302302302302	0.999623853535186\\
0.302802802802803	0.999637143263092\\
0.303303303303303	0.999650013707046\\
0.303803803803804	0.999662476340209\\
0.304304304304304	0.99967454237745\\
0.304804804804805	0.999686222779574\\
0.305305305305305	0.999697528257517\\
0.305805805805806	0.999708469276514\\
0.306306306306306	0.999719056060236\\
0.306806806806807	0.9997292985949\\
0.307307307307307	0.999739206633344\\
0.307807807807808	0.999748789699075\\
0.308308308308308	0.999758057090274\\
0.308808808808809	0.999767017883783\\
0.309309309309309	0.999775680939045\\
0.30980980980981	0.999784054902012\\
0.31031031031031	0.999792148209024\\
0.310810810810811	0.999799969090644\\
0.311311311311311	0.99980752557546\\
0.311811811811812	0.999814825493852\\
0.312312312312312	0.999821876481716\\
0.312812812812813	0.999828685984157\\
0.313313313313313	0.999835261259135\\
0.313813813813814	0.99984160938108\\
0.314314314314314	0.999847737244463\\
0.314814814814815	0.999853651567327\\
0.315315315315315	0.999859358894784\\
0.315815815815816	0.999864865602462\\
0.316316316316316	0.99987017789992\\
0.316816816816817	0.999875301834015\\
0.317317317317317	0.999880243292233\\
0.317817817817818	0.99988500800598\\
0.318318318318318	0.999889601553822\\
0.318818818818819	0.999894029364692\\
0.319319319319319	0.999898296721054\\
0.31981981981982	0.999902408762021\\
0.32032032032032	0.999906370486432\\
0.320820820820821	0.999910186755892\\
0.321321321321321	0.999913862297761\\
0.321821821821822	0.999917401708106\\
0.322322322322322	0.999920809454611\\
0.322822822822823	0.999924089879441\\
0.323323323323323	0.999927247202066\\
0.323823823823824	0.999930285522045\\
0.324324324324324	0.999933208821758\\
0.324824824824825	0.99993602096911\\
0.325325325325325	0.999938725720181\\
0.325825825825826	0.99994132672184\\
0.326326326326326	0.999943827514313\\
0.326826826826827	0.999946231533713\\
0.327327327327327	0.999948542114527\\
0.327827827827828	0.999950762492062\\
0.328328328328328	0.999952895804845\\
0.328828828828829	0.99995494509699\\
0.329329329329329	0.999956913320517\\
0.32982982982983	0.999958803337634\\
0.33033033033033	0.999960617922979\\
0.330830830830831	0.999962359765818\\
0.331331331331331	0.999964031472206\\
0.331831831831832	0.999965635567111\\
0.332332332332332	0.99996717449649\\
0.332832832832833	0.999968650629335\\
0.333333333333333	0.999970066259677\\
0.333833833833834	0.999971423608548\\
0.334334334334334	0.999972724825912\\
0.334834834834835	0.999973971992551\\
0.335335335335335	0.999975167121917\\
0.335835835835836	0.999976312161949\\
0.336336336336336	0.999977408996851\\
0.336836836836837	0.999978459448829\\
0.337337337337337	0.999979465279801\\
0.337837837837838	0.999980428193069\\
0.338338338338338	0.999981349834948\\
0.338838838838839	0.999982231796369\\
0.339339339339339	0.999983075614446\\
0.33983983983984	0.999983882774008\\
0.34034034034034	0.999984654709093\\
0.340840840840841	0.999985392804419\\
0.341341341341341	0.999986098396812\\
0.341841841841842	0.999986772776605\\
0.342342342342342	0.999987417189014\\
0.342842842842843	0.999988032835465\\
0.343343343343343	0.999988620874906\\
0.343843843843844	0.99998918242508\\
0.344344344344344	0.999989718563772\\
0.344844844844845	0.999990230330022\\
0.345345345345345	0.999990718725312\\
0.345845845845846	0.999991184714726\\
0.346346346346346	0.999991629228075\\
0.346846846846847	0.999992053161003\\
0.347347347347347	0.999992457376056\\
0.347847847847848	0.999992842703732\\
0.348348348348348	0.999993209943501\\
0.348848848848849	0.999993559864801\\
0.349349349349349	0.999993893208003\\
0.34984984984985	0.999994210685354\\
0.35035035035035	0.999994512981904\\
0.350850850850851	0.999994800756389\\
0.351351351351351	0.999995074642111\\
0.351851851851852	0.99999533524778\\
0.352352352352352	0.999995583158341\\
0.352852852852853	0.999995818935772\\
0.353353353353353	0.99999604311987\\
0.353853853853854	0.999996256229004\\
0.354354354354354	0.999996458760851\\
0.354854854854855	0.999996651193118\\
0.355355355355355	0.999996833984231\\
0.355855855855856	0.999997007574016\\
0.356356356356356	0.999997172384349\\
0.356856856856857	0.9999973288198\\
0.357357357357357	0.999997477268246\\
0.357857857857858	0.999997618101474\\
0.358358358358358	0.999997751675763\\
0.358858858858859	0.999997878332443\\
0.359359359359359	0.999997998398454\\
0.35985985985986	0.999998112186863\\
0.36036036036036	0.999998219997389\\
0.360860860860861	0.999998322116896\\
0.361361361361361	0.99999841881988\\
0.361861861861862	0.99999851036893\\
0.362362362362362	0.999998597015187\\
0.362862862862863	0.99999867899878\\
0.363363363363363	0.99999875654925\\
0.363863863863864	0.999998829885961\\
0.364364364364364	0.999998899218497\\
0.364864864864865	0.999998964747043\\
0.365365365365365	0.999999026662762\\
0.365865865865866	0.999999085148147\\
0.366366366366366	0.999999140377373\\
0.366866866866867	0.99999919251663\\
0.367367367367367	0.999999241724444\\
0.367867867867868	0.999999288151993\\
0.368368368368368	0.999999331943405\\
0.368868868868869	0.999999373236053\\
0.369369369369369	0.999999412160829\\
0.36986986986987	0.999999448842423\\
0.37037037037037	0.999999483399577\\
0.370870870870871	0.999999515945338\\
0.371371371371371	0.999999546587304\\
0.371871871871872	0.999999575427853\\
0.372372372372372	0.999999602564371\\
0.372872872872873	0.999999628089466\\
0.373373373373373	0.999999652091179\\
0.373873873873874	0.999999674653183\\
0.374374374374374	0.999999695854977\\
0.374874874874875	0.999999715772069\\
0.375375375375375	0.999999734476161\\
0.375875875875876	0.999999752035311\\
0.376376376376376	0.999999768514106\\
0.376876876876877	0.999999783973819\\
0.377377377377377	0.999999798472555\\
0.377877877877878	0.999999812065406\\
0.378378378378378	0.999999824804584\\
0.378878878878879	0.999999836739561\\
0.379379379379379	0.999999847917193\\
0.37987987987988	0.999999858381848\\
0.38038038038038	0.999999868175522\\
0.380880880880881	0.999999877337952\\
0.381381381381381	0.99999988590673\\
0.381881881881882	0.999999893917401\\
0.382382382382382	0.999999901403567\\
0.382882882882883	0.999999908396979\\
0.383383383383383	0.999999914927634\\
0.383883883883884	0.999999921023856\\
0.384384384384384	0.999999926712387\\
0.384884884884885	0.999999932018458\\
0.385385385385385	0.999999936965875\\
0.385885885885886	0.999999941577084\\
0.386386386386386	0.999999945873245\\
0.386886886886887	0.9999999498743\\
0.387387387387387	0.99999995359903\\
0.387887887887888	0.999999957065123\\
0.388388388388388	0.999999960289229\\
0.388888888888889	0.999999963287012\\
0.389389389389389	0.999999966073208\\
0.38988988988989	0.99999996866167\\
0.39039039039039	0.99999997106542\\
0.390890890890891	0.999999973296691\\
0.391391391391391	0.999999975366973\\
0.391891891891892	0.999999977287051\\
0.392392392392392	0.999999979067046\\
0.392892892892893	0.999999980716453\\
0.393393393393393	0.999999982244172\\
0.393893893893894	0.999999983658549\\
0.394394394394394	0.999999984967399\\
0.394894894894895	0.999999986178043\\
0.395395395395395	0.999999987297335\\
0.395895895895896	0.999999988331685\\
0.396396396396396	0.999999989287093\\
0.396896896896897	0.999999990169164\\
0.397397397397397	0.999999990983139\\
0.397897897897898	0.999999991733911\\
0.398398398398398	0.999999992426052\\
0.398898898898899	0.999999993063825\\
0.399399399399399	0.99999999365121\\
0.3998998998999	0.999999994191915\\
0.4004004004004	0.999999994689399\\
0.400900900900901	0.999999995146882\\
0.401401401401401	0.999999995567363\\
0.401901901901902	0.999999995953634\\
0.402402402402402	0.99999999630829\\
0.402902902902903	0.999999996633745\\
0.403403403403403	0.999999996932245\\
0.403903903903904	0.999999997205871\\
0.404404404404404	0.99999999745656\\
0.404904904904905	0.999999997686106\\
0.405405405405405	0.999999997896176\\
0.405905905905906	0.999999998088312\\
0.406406406406406	0.999999998263946\\
0.406906906906907	0.999999998424402\\
0.407407407407407	0.999999998570906\\
0.407907907907908	0.999999998704592\\
0.408408408408408	0.99999999882651\\
0.408908908908909	0.999999998937627\\
0.409409409409409	0.999999999038839\\
0.40990990990991	0.999999999130972\\
0.41041041041041	0.999999999214789\\
0.410910910910911	0.999999999290992\\
0.411411411411411	0.999999999360228\\
0.411911911911912	0.999999999423095\\
0.412412412412412	0.99999999948014\\
0.412912912912913	0.999999999531869\\
0.413413413413413	0.999999999578746\\
0.413913913913914	0.999999999621198\\
0.414414414414414	0.999999999659615\\
0.414914914914915	0.999999999694359\\
0.415415415415415	0.999999999725757\\
0.415915915915916	0.999999999754113\\
0.416416416416416	0.999999999779702\\
0.416916916916917	0.999999999802778\\
0.417417417417417	0.999999999823573\\
0.417917917917918	0.999999999842298\\
0.418418418418418	0.999999999859147\\
0.418918918918919	0.999999999874296\\
0.419419419419419	0.999999999887906\\
0.41991991991992	0.999999999900124\\
0.42042042042042	0.999999999911084\\
0.420920920920921	0.999999999920908\\
0.421421421421421	0.999999999929706\\
0.421921921921922	0.999999999937579\\
0.422422422422422	0.999999999944618\\
0.422922922922923	0.999999999950907\\
0.423423423423423	0.999999999956521\\
0.423923923923924	0.999999999961527\\
0.424424424424424	0.999999999965988\\
0.424924924924925	0.99999999996996\\
0.425425425425425	0.999999999973493\\
0.425925925925926	0.999999999976633\\
0.426426426426426	0.99999999997942\\
0.426926926926927	0.999999999981893\\
0.427427427427427	0.999999999984085\\
0.427927927927928	0.999999999986025\\
0.428428428428428	0.999999999987741\\
0.428928928928929	0.999999999989258\\
0.429429429429429	0.999999999990597\\
0.42992992992993	0.999999999991778\\
0.43043043043043	0.999999999992818\\
0.430930930930931	0.999999999993733\\
0.431431431431431	0.999999999994538\\
0.431931931931932	0.999999999995245\\
0.432432432432432	0.999999999995865\\
0.432932932932933	0.999999999996408\\
0.433433433433433	0.999999999996884\\
0.433933933933934	0.9999999999973\\
0.434434434434434	0.999999999997663\\
0.434934934934935	0.99999999999798\\
0.435435435435435	0.999999999998256\\
0.435935935935936	0.999999999998496\\
0.436436436436436	0.999999999998705\\
0.436936936936937	0.999999999998886\\
0.437437437437437	0.999999999999044\\
0.437937937937938	0.99999999999918\\
0.438438438438438	0.999999999999297\\
0.438938938938939	0.999999999999399\\
0.439439439439439	0.999999999999487\\
0.43993993993994	0.999999999999562\\
0.44044044044044	0.999999999999627\\
0.440940940940941	0.999999999999683\\
0.441441441441441	0.999999999999731\\
0.441941941941942	0.999999999999772\\
0.442442442442442	0.999999999999807\\
0.442942942942943	0.999999999999837\\
0.443443443443443	0.999999999999862\\
0.443943943943944	0.999999999999884\\
0.444444444444444	0.999999999999902\\
0.444944944944945	0.999999999999918\\
0.445445445445445	0.999999999999931\\
0.445945945945946	0.999999999999943\\
0.446446446446446	0.999999999999952\\
0.446946946946947	0.99999999999996\\
0.447447447447447	0.999999999999967\\
0.447947947947948	0.999999999999972\\
0.448448448448448	0.999999999999977\\
0.448948948948949	0.999999999999981\\
0.449449449449449	0.999999999999984\\
0.44994994994995	0.999999999999987\\
0.45045045045045	0.999999999999989\\
0.450950950950951	0.999999999999991\\
0.451451451451451	0.999999999999993\\
0.451951951951952	0.999999999999994\\
0.452452452452452	0.999999999999995\\
0.452952952952953	0.999999999999996\\
0.453453453453453	0.999999999999997\\
0.453953953953954	0.999999999999997\\
0.454454454454454	0.999999999999998\\
0.454954954954955	0.999999999999998\\
0.455455455455455	0.999999999999999\\
0.455955955955956	0.999999999999999\\
0.456456456456456	0.999999999999999\\
0.456956956956957	0.999999999999999\\
0.457457457457457	0.999999999999999\\
0.457957957957958	1\\
0.458458458458458	1\\
0.458958958958959	1\\
0.459459459459459	1\\
0.45995995995996	1\\
0.46046046046046	1\\
0.460960960960961	1\\
0.461461461461461	1\\
0.461961961961962	1\\
0.462462462462462	1\\
0.462962962962963	1\\
0.463463463463463	1\\
0.463963963963964	1\\
0.464464464464464	1\\
0.464964964964965	1\\
0.465465465465465	1\\
0.465965965965966	1\\
0.466466466466466	1\\
0.466966966966967	1\\
0.467467467467467	1\\
0.467967967967968	1\\
0.468468468468468	1\\
0.468968968968969	1\\
0.469469469469469	1\\
0.46996996996997	1\\
0.47047047047047	1\\
0.470970970970971	1\\
0.471471471471471	1\\
0.471971971971972	1\\
0.472472472472472	1\\
0.472972972972973	1\\
0.473473473473473	1\\
0.473973973973974	1\\
0.474474474474474	1\\
0.474974974974975	1\\
0.475475475475475	1\\
0.475975975975976	1\\
0.476476476476476	1\\
0.476976976976977	1\\
0.477477477477477	1\\
0.477977977977978	1\\
0.478478478478478	1\\
0.478978978978979	1\\
0.479479479479479	1\\
0.47997997997998	1\\
0.48048048048048	1\\
0.480980980980981	1\\
0.481481481481481	1\\
0.481981981981982	1\\
0.482482482482482	1\\
0.482982982982983	1\\
0.483483483483483	1\\
0.483983983983984	1\\
0.484484484484485	1\\
0.484984984984985	1\\
0.485485485485485	1\\
0.485985985985986	1\\
0.486486486486487	1\\
0.486986986986987	1\\
0.487487487487487	1\\
0.487987987987988	1\\
0.488488488488488	1\\
0.488988988988989	1\\
0.48948948948949	1\\
0.48998998998999	1\\
0.49049049049049	1\\
0.490990990990991	1\\
0.491491491491492	1\\
0.491991991991992	1\\
0.492492492492492	1\\
0.492992992992993	1\\
0.493493493493493	1\\
0.493993993993994	1\\
0.494494494494495	1\\
0.494994994994995	1\\
0.495495495495495	1\\
0.495995995995996	1\\
0.496496496496497	1\\
0.496996996996997	1\\
0.497497497497497	1\\
0.497997997997998	1\\
0.498498498498498	1\\
0.498998998998999	1\\
0.4994994994995	1\\
0.5	1\\
};
\addlegendentry{$\Pr\bigl(\theta \in \mc{H}(\varepsilon) \given \data\bigr)$};

\end{axis}
\end{tikzpicture}%
  \caption{The posterior probability of the hypotheses $\mc{H}(\varepsilon)$ for
    $0 < \varepsilon < \nicefrac{1}{2}$.}
  \label{near_fair_probabilities}
\end{figure}

We briefly discussed the classical approach to hypothesis testing in the last
lecture, and will expand upon that procedure here. The idea is to create a
so-called ``null hypothesis'' $\mc{H}_0$ that serves to define what ``typical''
data may look like assuming that hypothesis. For example, for reasoning about
the fairness of a coin, we may choose the natural null hypothesis
$\mc{H}_0\colon \theta = \nicefrac{1}{2}$. Now we can use the likelihood
\[
  \Pr(x \given n, \theta = \nicefrac{1}{2})
\]
to reason about what observed data would look like if this hypothesis were
true. This is a critical point: the null hypothesis exists to define what sort
of data we would expect to see under an assumed value of $\theta$.

The classical procedure is then to define a statistic summarizing a given
dataset $s(\data)$ in some way. An example for coin flipping would be the sample
mean $s(\data) = \hat{\theta} = \nicefrac{x}{n}$. This happens to be a common
estimator for $\theta$ as well, but this is a coincidence. We now compute a
so-called \emph{critical set} $C(\alpha)$ with the property
\[
  \Pr\bigl(s(\data) \in C(\alpha) \given \mc{H}_0\bigr) = 1 - \alpha,
\]
where $\alpha$ is called the \emph{significance level} of the test. The
interpretation of the critical set is that the statistic computed from datasets
generated assuming the null hypothesis ``usually'' have values in this range.

Finally, we compute the statistic for a particular set of observed data and
determine whether it lay inside the critical set $C(\alpha)$ we have defined. If
so, the dataset appears, according to the statistic, typical for datasets
generated from the null hypothesis. If not, the dataset appears unusual, in the
sense that data generated assuming the null hypothesis would have such extreme
values of the statistic only a small portion of the time ($100\alpha$\%). In
this case, you ``reject'' the null hypothesis with significance $1 - \alpha$.

What is a $p$-value? It must be the probability that the null hypothesis is
true, right? No, it can't be: the null hypothesis cannot be associated with a
probability in the classical interpretation of probability. A $p$-value is
actually the minimum $\alpha$ for which you would reject the null hypothesis
using this procedure. That is, a $p$-value is not the probability that the null
hypothesis is true, but rather the probability that we would observe results as
extreme as those in our dataset, as measured by the chosen statistic, \emph{if
  the null hypothesis were true!} The $p$-value is thus only a probability that
is well-defined when already assuming the null hypothesis to be true.  A
$p$-value does \emph{not} say how extreme our results would appear under
alternative hypotheses.

Bayesian model selection will eventually allow us to explicitly quantify the
plausibility of a collection of models having generated the observed data.

To interpret the above procedure in the frequency interpretation of probability,
the critical sets are constructed by reasoning about the following experiment:
\begin{itemize}
\item
  generate $\data$ assuming $\mc{H}_0$;
\item
  compute $s(\data)$;
\item
  state $s(\data) \in C(\alpha)$.
\end{itemize}
In the limit of infinitely many repetitions of this experiment, the final claim
will be true exactly $100(1 - \alpha)\%$ of the time. Recall this is the
definition of probability in this context: the frequency of occurrence in the
limit of infinitely many trials. Note that the experiment we repeat here
\emph{includes generating data from the null hypothesis} as its first step! This
is not the experiment we are conducting, since we have a dataset in front of us
that we want to analyze, which may have been generated in any number of ways.

\section*{Summarizing Distributions}

In the Bayesian method, the posterior distribution $p(\theta \given \data)$ is
the main object of interest and contains all relevant information about $\theta$
in light of the observations $\data$.  A natural task is to provide a summary of
the posterior distribution, for example to efficiently convey its relevant
properties.

In the next lecture we will consider point estimation, which is one common
summarization method. Another commonly considered problem is \emph{interval
  summarization,} where we provide an interval $(\ell, u)$ indicating plausible
values of the parameter $\theta$ in light of the observed data. Classical
interval estimates are known as \emph{confidence intervals,} and we will discuss
them in more detail shortly.

The Bayesian approach to interval estimation is straightforward. Again we use
the posterior distribution $p(\theta \given \data)$ to guide the construction of
an interval summary. If we can find an interval $(\ell, u)$ such that the
posterior probability that $\theta \in (\ell, u)$ is ``large'' (say, has
probability $\alpha$):
\[
  \Pr\bigl(\theta \in (\ell, u) \given \data\bigr)
  =
  \int_\ell^u p(\theta \given \data) \intd\theta
  =
  \alpha,
\]
then we call $(\ell, u)$ an $\alpha$\emph{-credible interval} for $\theta$.
Note the parallel in this definition to our treatment of hypothesis testing
above! Effectively, an $\alpha$-credible interval is simply a hypothesis that
has posterior probability equal to $\alpha$ and happens to take the form of an
interval.

Examining our coin flipping example from before, we can construct some credible
intervals immediately from the data in Figure \ref{near_fair_probabilities}.  We
have that $\mathcal{H}(\varepsilon = 0.1) = (0.4, 0.5)$ is a 50\%-credible
interval for the bias of the coin, and $\mathcal{H}(\varepsilon = 0.2) = (0.3,
0.7)$ is a 95\%-credible interval. The slightly wider interval
$\mathcal{H}(\varepsilon = 0.25) = (0.25, 0.75)$ represents a very high
probability credible interval, corresponding to $\alpha > 99\%$.

It is clear from the definition that multiple intervals (in fact, often
uncountably many) can serve as a credible interval for a particular value of
$\alpha$. Exactly which interval should we construct to summarize a given
distribution? This is a question for which we will need to develop Bayesian
decision theory before we can continue, which we will discuss in the next
lecture. In short, we will first need to quantify how ``desirable'' a given
credible interval is in some way, then select the one maximizing this measure.
For example, we may want to construct the narrowest possible interval, or we may
wish it to be centered on a particular point (such as the posterior mean,
median, or mode), or we may wish the interval to have some other property.

The classical approach to interval summarization is to construct a so-called
confidence interval for the parameter of interest $\theta$. Again a confidence
interval is described in terms of repeating a particular experiment infinitely
many times. The experiment we consider will proceed as follows. First we are
going to define a function $\ci(\data)$ that will map a given dataset $\data$ to
an interval $(\ell, u) = \ci(\data)$. Now we consider repeating the following
experiment:
\begin{itemize}
\item
  collect data $\data$
\item
  compute the interval $(\ell, u) = \ci(\data)$
\item
  state $\theta \in (\ell, u)$.
\end{itemize}
In the limit of infinitely many repetitions of this experiment, if the final
statement is true with probability $\alpha$, then the procedure $\ci(\data)$ is
called an $\alpha$\emph{-confidence interval procedure,} and we will write
$\ci(\data; \alpha)$ to indicate the confidence level $\alpha$ when required.

This might sound like exactly the same definition as a Bayesian credible
interval. For example, if we have an $\alpha$-confidence interval procedure
available, then when we plug in a given dataset $\data$, we must have
\begin{equation}
  \Pr\bigl(\theta \in \ci(\data; \alpha) \given \data) = \alpha, \tag{$\star$}
  \label{wrong}
\end{equation}
right? \emph{No!} This interpretation is widespread, but it is wrong.  The
conclusion in \eqref{wrong} is sometimes known as the \emph{fundamental
  confidence fallacy,}%
%
\footnote{See the following reference for some excellent extended discussion on
  confidence intervals:  Richard D.\ Morey, et al.\ (2015). The fallacy of placing
  confidence in confidence intervals. \emph{Psychonomic Bulletin \& Review}
  23(1): 103--123}
%
and confuses the nature of prior information with that of posterior
information. Namely, note that the experiment we consider when defining the
confidence interval procedure \emph{includes gathering a random dataset} as its
first step. All we know is that if we repeat the confidence interval procedure
on \emph{infinitely many datasets,} that it will succeed with probability
$\alpha$. However, we usually have only one particular dataset in front of us to
analyze that we care about, and we cannot say anything about the interval
produced for this dataset in isolation.

Here is a simple example that shows how \eqref{wrong} can fail. Suppose we are
going to observe two values $x_1, x_2 \in \R$ generated independently from some
unknown distribution $p(x)$ and wish to construct a confidence interval for the
mean of the distribution generating the data, $\theta = \E[x]$. Consider the
following procedure:
\[
  \ci(\data) =
  \begin{cases}
    (-\infty, \infty) & x_1  <   x_2 \\
    \emptyset         & x_2 \geq x_1.
  \end{cases}
\]
Obviously this trivial map is a 50\%-confidence interval procedure! Because the
values are generated independently, $x_1$ will be the lesser value exactly 50\%
of the time.  In this case, the absurdly large interval produced will
\emph{definitely} contain $\theta$.  The other 50\% of the time, the interval
will be empty, and \emph{definitely will not} contain $\theta$. Therefore the
procedure succeeds exactly 50\% of the time. However, in half the cases, the
posterior probability that $\theta$ is inside the interval produced is 100\%,
and otherwise this probability is 0\%. In no case is this probability equal to
the confidence level.

Another fallacy in the interpretation of confidence intervals is the so called
\emph{precision fallacy,} that shorter confidence intervals indicate the data
provide more precise information about $\theta$. A striking illustration of this
fallacy is provided by the ``lost submarine'' example provided by Morey, et
al.\ in the reference given below. I encourage you to read this paper and
reflect!

\end{document}
