\documentclass{article}

\usepackage[scaled=0.8]{beramono}
\usepackage[margin=1.25in]{geometry}
\usepackage{url}
\usepackage{booktabs}
\usepackage{amsmath}
\usepackage{nicefrac}
\usepackage{microtype}

% font (currently libertine, will revisit!)
\usepackage[semibold,lining]{libertine}
\usepackage[T1]{fontenc}
\usepackage{textcomp}
\usepackage[varqu,varl,scaled=0.96,var0]{zi4}
\usepackage[libertine,vvarbb,upint]{newtxmath}
\usepackage{bm}
\useosf

\usepackage{pgfplots}

\pgfplotsset{
  compat=newest,
  plot coordinates/math parser=false,
  tick label style={font=\footnotesize, /pgf/number format/fixed},
  label style={font=\small},
  legend style={font=\small},
  /pgfplots/area legend/.style={%
    /pgfplots/legend image code/.code={%
      \fill[##1] (0cm,-0.1cm) rectangle (0.6cm,0.1cm);
    }%
  },
  every non boxed x axis/.append style={axis line style={line cap=rect}},
  every axis/.append style={
    axis on top=true,
    tick align=outside,
    clip mode=global,
    clip=false,
    semithick,
    scaled ticks=false,
    tick style={semithick, black}
  }
}
\pgfkeys{/pgf/number format/.cd, set thousands separator={\,}}

\pgfplotsset{colormap legend/.style={%
    legend image code/.code={%
      \path[draw=none,shading=tempshading,shade] (0cm,-0.1cm) rectangle (0.6cm,0.1cm);
    }%
  }
}

\newcommand{\definelength}[2]{
    \expandafter\newlength\csname #1\endcsname%
    \expandafter\setlength\csname #1\endcsname{#2}%
}

\definelength{fullwidth}{\textwidth}
\definelength{base}{3cm}                                              % [-3,   4]
\usetikzlibrary{fit}

\usepackage{sectsty}
\sectionfont{\large}
\subsectionfont{\normalsize}

\usepackage{titlesec}
\titlespacing{\section}{0pt}{10pt plus 2pt minus 2pt}{0pt plus 2pt minus 0pt}
\titlespacing{\subsection}{0pt}{5pt plus 2pt minus 2pt}{0pt plus 2pt minus 0pt}

\usepackage{enumitem}
\setlist[enumerate]{%
  leftmargin=\itemindent,%
  topsep=0\baselineskip,%
  itemsep=0\baselineskip}
\setlist[itemize]{%
  leftmargin=\itemindent,%
  topsep=0.2\baselineskip,%
  itemsep=0.1\baselineskip,%
  parsep=0pt}

\setlength{\parindent}{0pt}
\setlength{\parskip}{1ex}

\newcommand{\acro}[1]{\textsc{\MakeLowercase{#1}}}
\newcommand{\given}{\mid}
\newcommand{\mc}[1]{\mathcal{#1}}
\newcommand{\data}{\mc{D}}
\newcommand{\intd}[1]{\,\mathrm{d}{#1}}
\newcommand{\tikzsetnextfilename}[1]{}

\frenchspacing

\begin{document}

\section*{Introduction to Bayesian Inference}

In this course, we will focus on \emph{inference,} the process of inferring
unknown properties of a system given observations. In particular, we will study
\emph{probabilistic inference,} where we draw upon the mechanics of probability
theory for this task.

To provide a motivating example, suppose we want to understand some aspect of
the population of the United States, such as the portion of adults who prefer
Coca Cola to Pepsi (or Uber to Lyft?). Let this unknown value be called
$\theta$.  How can we gain insight into this value? (This is just a simple A/B
situation that is extremely common at Uber!  At Uber, we might be interested in,
say, the proportion of users who interact with a particular feature of the app
when it's configured in a certain way.)

One thing we could do is ask stakeholders what they \emph{believe} about
$\theta$, which they might compactly communicate via a probability distribution
over its plausible values, perhaps even by simply drawing a curve on a piece of
paper.  Note that different people might give different answers to this
question!  (What do \emph{you} think are plausible values of $\theta$?)

Of course, the Coca Cola corporation (or your boss at Uber) probably wouldn't be
too happy if you were to simply draw your personal beliefs and submit a report.
Instead, we might proceed by conducting a survey or an A/B experiment to gain
some more information about $\theta$.  So let's say we contact some people, ask
them if they prefer Coca Cola to Pepsi, and record the results.  Let's call the
results of this experiment $\data$ (for ``data'').

What do we do with this data once we've measured it?  The goal of inference is
to draw some conclusions about $\theta$ given these observations: is $\theta >
\text{50\%?}$ how much does $\theta$ change in response to advertising? etc. At
Uber we might be interested in questions like ``what is the average treatment
effect of a given intervention?''

\section*{Quick Probability Refresher}

Let's pause for a quick probability refresher.  There are just two laws of
probability you need to know.  The first is the \emph{sum rule,} also called the
\emph{rule of total probability.}  Suppose $X$ is some event (an event is simply
a collection of things that could be true, for example ``the total value shown
on two dice is less than six''), and suppose $\{Y_i\}$ are some mutually
exclusive and exhaustive events on the same space (for example, $Y_i$ could be
the event ``the first rolled die has value $i$,'' for $i \in \{1, 2, \dotsc,
6\}$).  Then:
\[
  \Pr(X) = \sum_i \Pr(X, Y_i),
\]
so the total probability of $X$ is the sum of the probabilities of $X$ occurring
alongside each of the scenarios described by the $\{Y_i\}$ -- the \emph{joint
probabilities} $\{ \Pr(X, Y_i) \}$.

If there is a sum rule, there must be a \emph{product rule:}
\[
  \Pr(X, Y) = \Pr(Y \given X)\Pr(X) = \Pr(X \given Y)\Pr(Y).
\]
Here the \emph{conditional probability} $\Pr(X \given Y)$, read ``the
probability of $X$ given $Y$,'' is the probability of event $X$ occurring when
we restrict to cases where event $Y$ also holds.  (It is easy to draw this
conclusion from drawing a Venn diagram and reasoning about what should happen
when you restrict the possible worlds to only those where $Y$ occurs.)

By manipulating the product rule (and applying the sum rule), we arrive at
\emph{Bayes' theorem:}
\[
  \Pr(Y \given X) = \frac{\Pr(X \given Y)\Pr(Y)}{\Pr(X)}
                  = \frac{\Pr(X \given Y)\Pr(Y)}{\sum \Pr(X \given Y)\Pr(Y)}.
\]

\section*{Inductive reasoning}

One way to interpret Bayes' theorem is as follows.  Suppose we are interested in
whether event $Y$ holds.  We begin with a prior \emph{belief} about $Y$,
$\Pr(Y)$.  We then learn that $X$ is true.  We use the formula above to update
our belief about $Y$ by conditioning on this new information, giving $\Pr(Y
\given X)$.  Bayes' theorem therefore provides a probabilistically consistent
way to update one's beliefs given new information!  In this context, the value
$\Pr(Y)$ is called a \emph{prior probability,} as it represents our belief prior
to observing $X$.  The result $\Pr(Y \given X)$ is called the \emph{posterior
probability,} as it has been updated in light of the new information.

(What is the prior probability that the outcome of rolling two dice is less than
six?  What if I told you that the value of the first die was a five, does that
change anything?)

Despite its name, inference does not necessarily end with the posterior
distribution!  What if we now learn that some other event $Z$ also holds? We
simply apply the same update, with what was the posterior after observing $Y$
now assuming the role of the prior before observing $Z$ -- we simply ``turn the
crank'':
\[
  \Pr(Y \given X, Z) =
  \frac{\Pr(Z \given X, Y)\Pr(Y \given X)}{\Pr(Z \given X)}.
\]
We can continue this process ad infinitum, at every point in time maintaining a
probabilistic belief in light of the currently available data and applying
Bayes' theorem to update this belief whenever we observe relevant additional
information. This inductive process is the backbone of Bayesian inference.  We
now know why we need a prior distribution: every induction requires a base case!

\subsection*{Continuous random variables}

The above results are only valid for discrete random variables $X$.  However, in
practice, we will often be interested in continuous variables as well, such as
the Coke-vs-Pepsi $\theta$ above.  Thankfully, the above formulas are perfectly
valid for continuous random variables $x$ and $\theta$, where we replace
probabilities (which I usually notate with $\Pr$) with probability density
functions (which I usually notate with $p$) and replace sums with integrals:
\[
  p(x) = \int p(x, y) \intd{y} = \int p(x \given y) \, p(y) \intd{y};
  \qquad
  p(y \given x) = \frac{p(x \given y) \, p(y)}{\int p(x \given y) \, p(y) \intd{y}}.
\]

\subsection*{Return to survey example}

Let us return to our survey example. When we left off, we had established a
prior belief about $\theta$, represented by a prior probability distribution
$p(\theta)$, and had conducted a survey, yielding results $\data$.

To couple the experimental outcomes to the value of interest, we construct a
data-generation model $p(\data \given \theta)$, which describes how likely we
would see a particular survey outcome $\data$ given a particular value of
$\theta$.  Note that this model could have any form and we are free to make it
as complicated as we'd like: was there bias in the sampling mechanism? do people
always tell the truth? etc. In the context of Bayesian inference, this model
mapping from the parameter of interest onto observed data is called the
\emph{likelihood.}

Finally, we compute the posterior distribution of $\theta$ given the survey
results by appealing to Bayes' theorem:%
%
\footnote{The constant proportionality is $p(\data) = \int p(\data \given \theta)\, p(\theta) \intd\theta$. Note that this doesn't depend on $\theta$ and simply ensures normalization.}%
\[
  p(\theta \given \data) \propto p(\data \given \theta)\, p(\theta); \qquad
  \text{posterior} \propto \text{likelihood} \times \text{prior.}
\]
The posterior distribution encapsulates our belief about $\theta$ in light of the
observed data and combines information from both our prior knowledge and
experience (reflected by the prior distribution) and from the observed evidence
(reflected by the likelihood of the observed data).

Figure \ref{inference_example} illustrates this process. In the left-hand panel,
we begin with a prior subjective belief indicating a range of plausible values
for $\theta$. One possible likelihood for a given set of survey results is shown
in the middle panel (as a function of $\theta$), with the prior shown for
reference. The likelihood heavily favors Pepsi!  However, these results were
also rather surprising given our prior belief. This tension is resolved in the
right-hand panel, where the posterior distribution provides a compromise between
these two sources of information.

\begin{figure}
  {\hspace*{-2mm}\input{figures/inference_1}}\hfill\nobreak%
  \raisebox{0.00cm}{\hspace*{-2mm}\input{figures/inference_2}}\hfill\nobreak%
  \hspace*{-2mm}\input{figures/inference_3}%
  \caption{A cartoon of Bayesian inference.}
  \label{inference_example}
\end{figure}

\section*{The Bayesian method: A recipe}

There are four main steps to the Bayesian approach to probabilistic
inference:\footnote{These are summarized from Tony O'Hagan and Jonathan
Forster's eloquent introduction in \emph{Kendall's Advanced Theory of Statistics
Volume 2B.}}
\begin{itemize}
\item \textbf{Likelihood.} First, we construct the likelihood (or
  data-generation model), $p(\data \given \theta)$.  This serves to describe the
  mechanism giving rise our observations $\data$ given a particular value of the
  parameter of interest $\theta$.
\item \textbf{Prior.} Next, we summarize our prior beliefs about the parameters
  $\theta$, which we encode via a probability distribution $p(\theta)$.
\item \textbf{Posterior.} Given some observations $\data$, we obtain the
  posterior distribution $p(\theta \given \data)$ using Bayes' theorem.
\item \textbf{Inference.} We now use the posterior distribution to draw further
  conclusions as required.
\end{itemize}
The last step is purposely open-ended. For example, we might wish to:
\begin{itemize}
\item summarize our beliefs in various ways (for example, point estimates or
interval summaries),
\item make predictions about new data,
\item support a follow-on decision, such as whether to ship a new product
  (giving rise to Bayesian decision theory),
\item compare alternative models for the data (Bayesian model
comparison),
\item determine which data to obtain next (Bayesian experimental design),
\item or any one of countless other things.
\end{itemize}
We will consider several of these goals in this course.

%% The posterior distribution encapsulates our entire belief about $\theta$! We can
%% use it to answer various questions we might be about $\theta$.  For example,
%% what is the probability that $\theta > \text{50\%}$ in light of the survey
%% results? This question is easy to answer given the posterior probability
%% density:
%% \[
%% \Pr(\theta > \text{50\%} \given \data) =
%% \int_{0.5}^1 p(\theta \given \data) \intd \theta.
%% \]

\section*{The meaning of probability}

Some readers might balk at the treatment of ``probability'' used in the sketch
of Bayesian inference above. What exactly is the meaning of ``probability?''
There is a great deal of fascinating philosophical writing on the subject, which
we will largely avoid. However, we can provide a quick rundown of the situation.

The dominant statistical practice for many years (known as the \emph{classical}
or \emph{frequentist} theory) defines the probability of an event as its
frequency (hence ``frequentist'') of its occurrence in the limit of infinitely
many repeated experiments. For example, the frequentist probability of a fair
coin landing heads is 50\%, because if we were to flip the coin infinitely many
times and measure the empirical frequency of heads vs.\ tails, the resulting
value would converge to 50\%. This is a perfectly workable definition, but
adopting it can at times limit its practical application.

For example, in the context of A/B testing, it is impossible to consider the
``probability'' that over 50\% of adults prefer Coca Cola to Pepsi. If we could
conduct an experiment revealing this information (which would itself entail
infinitely many trials to determine $\theta$), then every realization of this
experiment would yield the same result!  Thus we can only conclude that the
probability of this proposition is either zero or one, but we can’t determine
which of these worlds we live in from a finite experiment. The same is true for
assessing the probability of any logical proposition: ``there is life on Mars,''
``it will rain tomorrow,'' ``the average treatment effect of some treatment is
positive,'' etc. We face a similar issue when trying to reason about continuous
unknown parameters, which we can only interpret this parameter as having an
unknown (but fixed) value whose only workable probability ``distribution'' is a
Dirac delta distribution on the true value. Although this conclusion is
logically consistent with the chosen definition, it is not especially useful for
inference.

In the Bayesian approach to inference, all unknown quantities (parameters,
hypotheses, even the best choice of model) are treated as random variables,
whose associated probability distributions represent \emph{subjective degrees of
belief} in their plausible values rather than any asymptotic property regarding
frequency of occurrence. Inference then takes the form of an inductive process
where these beliefs are iteratively refined in light of observed data by
appealing to Bayes' theorem. With this relaxed interpretation, ``there is a 90\%
probability that $\theta > \text{50}\%$'' is a perfectly valid claim.

Note that these interpretations of probability are not completely incompatible;
any reasonable Bayesian would agree that the probability of a fair coin landing
on heads is 50\%. (But a Bayesian might also entertain the probability of
alternative hypotheses such as ``the nominally fair coin has been secretly
swapped with a two-headed coin.'') Further, although the Bayesian interpretation
of probability is as a subjective degree of belief (and thus any claim can be
rationally disputed), we are bound by the laws of probability when manipulating
these beliefs, and we are compelled to ``face the facts'' when presented with
overwhelming evidence. In particular, in the large-sample limit of infinitely
many observations, subjectivity is no longer a feasible position, and
frequentist and Bayesian probabilities typically coincide. For example, both
groups would agree in the limit of an infinitely long A/B experiment that the
``distribution'' of the parameter of interest had collapsed to a Dirac delta
(and agree on its location). However, a Bayesian analysis of the data is allowed
to relax these inevitable conclusions along the way, whereas a frequentist
analysis cannot.

\section*{Issues}

Bayesian inference is a completely consistent system for probabilistic
reasoning.  Unfortunately, it is not without its issues, two of which we list
below.

\subsection*{Origin of priors}

In contrast to the data-generation model $p(\data \given \theta)$, it is not
usually clear where the prior $p(\theta)$ should come from.  There is an entire
branch of study concerning prior elicitation, but for now we will simply treat
it as given.  We will continue to discuss this throughout the course. Indeed, we
will see that several ``tricks'' often encountered in alternative approaches
(such as \emph{regularization}) can be interpreted as implicitly placing
particular prior beliefs on $\theta$.

The need for a prior distribution does not need to be a crux. Instead, we can
view it as a mechanism to inject our knowledge about and experience with the
system of interest into the inferential process, saving us from having to begin
``from scratch'' or entertain patently absurd possibilities. At Uber, in some
situations we might very well have a great deal of prior information coming
from, e.g., the outcomes of a reliable backtesting procedure or the results of
previous experiments.

\subsection*{Intractable integrals}

Unfortunately, the integral in the denominator of Bayes' theorem:
\[
  p(x) = \int p(x \given y)\, p(y) \intd{y}
\]
is not in general tractable for arbitrary combinations of priors and
likelihoods.  For this reason, much of everyday Bayesian inference entails first
very briefly writing down a model then spending a great deal of time fretting
over how to deal with it computationally.  Sometimes this can be more of an art
than a science, but nowadays there are excellent computational tools available,
and automating this process is the goal of the rapidly evolving field of
\emph{probabilistic programming}.

\end{document}
