\documentclass{article}

\usepackage[T1]{fontenc}
\usepackage[osf]{libertine}
\usepackage[scaled=0.8]{beramono}
\usepackage[margin=1.5in]{geometry}
\usepackage{url}
\usepackage{booktabs}
\usepackage{amsmath}
\usepackage{amsfonts}
\usepackage{nicefrac}
\usepackage{microtype}
\usepackage{bm}

\usepackage{sectsty}
\sectionfont{\large}
\subsectionfont{\normalsize}

\usepackage{titlesec}
\titlespacing{\section}{0pt}{10pt plus 2pt minus 2pt}{0pt plus 2pt minus 0pt}
\titlespacing{\subsection}{0pt}{5pt plus 2pt minus 2pt}{0pt plus 2pt minus 0pt}

\setlength{\parindent}{0pt}
\setlength{\parskip}{1ex}

\newcommand{\acro}[1]{\textsc{\MakeLowercase{#1}}}
\newcommand{\given}{\mid}
\newcommand{\mc}[1]{\mathcal{#1}}
\newcommand{\data}{\mc{D}}
\newcommand{\intd}[1]{\,\mathrm{d}{#1}}
\newcommand{\R}{\mathbb{R}}

\begin{document}

{\large \textbf{CSE 515T (Fall 2024) Assignment 3}} \\
Due Wednesday, 13 November 2024 \\

\begin{enumerate}

\item

  Replicate figure 2.1 in the Bayesian optimization book
  (\url{bayesoptbook.com}) from scratch in an environment such as Python, R, or
  Julia.  I will post hints to slack.  This should require less than 100 lines
  of code.

\item

  Replicate figure 2.2 in the Bayesian optimization book
  (\url{bayesoptbook.com}) from scratch in an environment such as Python, R, or
  Julia.  I will post hints to slack.  This should require less than 100 lines
  of code.

\item

  Replicate figure 2.3 in the Bayesian optimization book
  (\url{bayesoptbook.com}) from scratch in an environment such as Python, R, or
  Julia.  I will post hints to slack.  This should require less than 100 lines
  of code.

\item

  Consider a Gaussian process on $f\colon \R^d \to \R$, $p(f) = \mc{GP}(f; \mu,
  K).$ Suppose $\mu$ and $K$ are differentiable with respect to their
  inputs. Consider the $i$th partial derivative of $f$ at some point $x$:
  \[
  f'_i = \begin{frac}{\partial f}{\partial x_i}(x)\end{frac}.
  \]
  Show that $f'_i$ has a Gaussian distribution. What is its mean and variance?
  Hint: work from the definition of the partial derivative and use the fact that
  a limiting sequence of Gaussian distributions is Gaussian.

\end{enumerate}

\end{document}
