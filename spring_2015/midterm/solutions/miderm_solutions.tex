\documentclass{article}

\usepackage[T1]{fontenc}
\usepackage[osf]{libertine}
\usepackage[scaled=0.8]{beramono}
\usepackage[margin=1.5in]{geometry}
\usepackage{url}
\usepackage{booktabs}
\usepackage{amsmath}
\usepackage{amssymb}
\usepackage{nicefrac}
\usepackage{microtype}
\usepackage{bm}

\usepackage{sectsty}
\sectionfont{\large}
\subsectionfont{\normalsize}

\usepackage{titlesec}
\titlespacing{\section}{0pt}{10pt plus 2pt minus 2pt}{0pt plus 2pt minus 0pt}
\titlespacing{\subsection}{0pt}{5pt plus 2pt minus 2pt}{0pt plus 2pt minus 0pt}

\usepackage{pgfplots}
\pgfplotsset{
  compat=newest,
  plot coordinates/math parser=false,
  tick label style={font=\footnotesize, /pgf/number format/fixed},
  label style={font=\small},
  legend style={font=\small},
  every axis/.append style={
    tick align=outside,
    clip mode=individual,
    scaled ticks=false,
    thick,
    tick style={semithick, black}
  }
}

\pgfkeys{/pgf/number format/.cd, set thousands separator={\,}}

\usepgfplotslibrary{external}
\tikzexternalize[prefix=tikz/]

\newlength\figurewidth
\newlength\figureheight

\setlength{\figurewidth}{8cm}
\setlength{\figureheight}{6cm}

\newlength\squarefigurewidth
\newlength\squarefigureheight

\setlength{\squarefigurewidth}{7cm}
\setlength{\squarefigureheight}{6cm}

\setlength{\parindent}{0pt}
\setlength{\parskip}{1ex}

\newcommand{\acro}[1]{\textsc{\MakeLowercase{#1}}}
\newcommand{\given}{\mid}
\newcommand{\mc}[1]{\mathcal{#1}}
\newcommand{\data}{\mc{D}}
\newcommand{\trans}{^\top}
\newcommand{\inv}{^{-1}}
\newcommand{\intd}[1]{\,\mathrm{d}{#1}}
\newcommand{\mat}[1]{\bm{\mathrm{#1}}}
\renewcommand{\vec}[1]{\bm{\mathrm{#1}}}

\DeclareMathOperator{\var}{var}

\begin{document}

{\large \textbf{CSE 515T (Spring 2015) Midterm solutions}}
\begin{enumerate}
\item
  Consider two coins with unknown bias $\theta_1$ and $\theta_2$,
  respectively.  We place independent, identical beta priors on these
  quantities:
  \begin{equation*}
    p(\theta_1) = \mc{B}(\theta_1; 2, 2);
    \qquad
    p(\theta_2) = \mc{B}(\theta_2; 2, 2).
  \end{equation*}
  Imagine someone flips both coins and tells you that \emph{exactly
    one} of the outcomes (but not which) was a ``head.''  Thus the
  observation was either HT or TH, but you are not told which.  The
  below expressions are conditioned on ``H'' to indicate this
  observation.
  \begin{itemize}
  \item
    Give an expression for the posterior of the first coin's bias
    given this observation, $p(\theta_1 \given \text{H})$.  Simplify
    the result as much as you can.  Plot the prior and the posterior
    for $\theta_1$ over the interval $\theta_1 \in (0, 1)$.
  \item
    Give an expression for the joint posterior $p(\theta_1, \theta_2
    \given \text{H})$. Plot the joint prior, the likelihood, and the
    joint posterior as three separate heat maps over the unit square
    $(\theta_1, \theta_2) \in (0, 1)^2$.  Use a grid with at least 100
    values along each of the two $\theta$ axes.
  \item
    Summarize what the observation taught us about the bias of the
    coins.
  \end{itemize}
\end{enumerate}

\subsection*{Solution}

The outcome of the unseen experiment was either HT or TH.  These are
mutually exhaustive and independent events, so we may use the sum
rule to derive the desired posterior:
\begin{equation*}
  p(\theta_1 \given \text{H})
  =
  \Pr(\text{HT})p(\theta_1 \given \text{HT})
  +
  \Pr(\text{TH})p(\theta_1 \given \text{TH})
  .
\end{equation*}
Notice that both $p(\theta_1 \given \text{HT})$ and $p(\theta_1 \given
\text{TH})$ can be computed explicitly as updated beta distributions,
now that the coins in the outcomes have been identified:
\begin{align*}
  p(\theta_1 \given \text{HT})
  &=
  \mc{B}(\theta_1; 3, 2)
  \\
  p(\theta_1 \given \text{TH})
  &=
  \mc{B}(\theta_1; 2, 3).
\end{align*}

What is $\Pr(\text{HT})$?  It can be calculated explicitly, but a
simpler approach is to appeal to symmetry between the coins to
conclude $\Pr(\text{HT}) = \Pr(\text{TH}) = \nicefrac{1}{2}$.  Thus
\begin{equation*}
  p(\theta_1 \given \text{H})
  =
  \tfrac{1}{2}
  \bigl(
  p(\theta_1 \given \text{HT}) +
  p(\theta_1 \given \text{TH})
  \bigr)
  =
  \tfrac{1}{2}
  \bigl(
  \mc{B}(\theta_1; 3, 2) +
  \mc{B}(\theta_1; 2, 3)
  \bigr)
  .
\end{equation*}
We can simply this expression further.  The posterior is proportional
to
\begin{equation*}
  p(\theta_1 \given \text{H})
  \propto
  \theta_1^2 (1 - \theta_1)
  +
  \theta_1 (1 - \theta_1)^2
  =
  \theta_1(1 - \theta_1)
  \propto
  \mc{B}(\theta_1; 2, 2).
\end{equation*}
Therefore the distribution of $\theta_1$ has not changed given our
observation!  The prior (and posterior!) for $\theta_1$ is plotted
in Figure \ref{problem_1_prior_1}.

\begin{figure}
  \centering
  % This file was created by matlab2tikz.
% Minimal pgfplots version: 1.3
%
\tikzsetnextfilename{problem_1_prior_1}
\definecolor{mycolor1}{rgb}{0.12157,0.47059,0.70588}%
%
\begin{tikzpicture}

\begin{axis}[%
width=0.95092\figurewidth,
height=\figureheight,
at={(0\figurewidth,0\figureheight)},
scale only axis,
xmin=0,
xmax=1,
xlabel={$\theta_1$},
ymin=0,
ymax=1.75,
axis x line*=bottom,
axis y line*=left,
legend style={legend cell align=left,align=left,fill=none,draw=none}
]
\addplot [color=mycolor1,solid]
  table[row sep=crcr]{%
0	0\\
0.001001001001001	0.00599999398798197\\
0.002002002002002	0.0119879639399159\\
0.003003003003003	0.0179639098558017\\
0.004004004004004	0.0239278317356395\\
0.005005005005005	0.0298797295794293\\
0.00600600600600601	0.0358196033871709\\
0.00700700700700701	0.0417474531588646\\
0.00800800800800801	0.0476632788945101\\
0.00900900900900901	0.0535670805941076\\
0.01001001001001	0.059458858257657\\
0.011011011011011	0.0653386118851584\\
0.012012012012012	0.0712063414766117\\
0.013013013013013	0.077062047032017\\
0.014014014014014	0.0829057285513742\\
0.015015015015015	0.0887373860346834\\
0.016016016016016	0.0945570194819444\\
0.017017017017017	0.100364628893157\\
0.018018018018018	0.106160214268322\\
0.019019019019019	0.111943775607439\\
0.02002002002002	0.117715312910508\\
0.021021021021021	0.123474826177529\\
0.022022022022022	0.129222315408502\\
0.023023023023023	0.134957780603426\\
0.024024024024024	0.140681221762303\\
0.025025025025025	0.146392638885131\\
0.026026026026026	0.152092031971912\\
0.027027027027027	0.157779401022644\\
0.028028028028028	0.163454746037329\\
0.029029029029029	0.169118067015965\\
0.03003003003003	0.174769363958553\\
0.031031031031031	0.180408636865093\\
0.032032032032032	0.186035885735585\\
0.033033033033033	0.19165111057003\\
0.034034034034034	0.197254311368425\\
0.035035035035035	0.202845488130773\\
0.036036036036036	0.208424640857073\\
0.037037037037037	0.213991769547325\\
0.038038038038038	0.219546874201529\\
0.039039039039039	0.225089954819685\\
0.04004004004004	0.230621011401792\\
0.041041041041041	0.236140043947852\\
0.042042042042042	0.241647052457863\\
0.043043043043043	0.247142036931827\\
0.044044044044044	0.252624997369742\\
0.045045045045045	0.258095933771609\\
0.046046046046046	0.263554846137429\\
0.047047047047047	0.2690017344672\\
0.048048048048048	0.274436598760923\\
0.049049049049049	0.279859439018598\\
0.0500500500500501	0.285270255240225\\
0.0510510510510511	0.290669047425804\\
0.0520520520520521	0.296055815575335\\
0.0530530530530531	0.301430559688818\\
0.0540540540540541	0.306793279766253\\
0.0550550550550551	0.312143975807639\\
0.0560560560560561	0.317482647812978\\
0.0570570570570571	0.322809295782269\\
0.0580580580580581	0.328123919715511\\
0.0590590590590591	0.333426519612706\\
0.0600600600600601	0.338717095473852\\
0.0610610610610611	0.343995647298951\\
0.0620620620620621	0.349262175088001\\
0.0630630630630631	0.354516678841003\\
0.0640640640640641	0.359759158557957\\
0.0650650650650651	0.364989614238864\\
0.0660660660660661	0.370208045883722\\
0.0670670670670671	0.375414453492532\\
0.0680680680680681	0.380608837065294\\
0.0690690690690691	0.385791196602007\\
0.0700700700700701	0.390961532102673\\
0.0710710710710711	0.396119843567291\\
0.0720720720720721	0.401266130995861\\
0.0730730730730731	0.406400394388382\\
0.0740740740740741	0.411522633744856\\
0.0750750750750751	0.416632849065282\\
0.0760760760760761	0.421731040349659\\
0.0770770770770771	0.426817207597988\\
0.0780780780780781	0.43189135081027\\
0.0790790790790791	0.436953469986503\\
0.0800800800800801	0.442003565126688\\
0.0810810810810811	0.447041636230825\\
0.0820820820820821	0.452067683298914\\
0.0830830830830831	0.457081706330956\\
0.0840840840840841	0.462083705326949\\
0.0850850850850851	0.467073680286893\\
0.0860860860860861	0.47205163121079\\
0.0870870870870871	0.477017558098639\\
0.0880880880880881	0.48197146095044\\
0.0890890890890891	0.486913339766193\\
0.0900900900900901	0.491843194545897\\
0.0910910910910911	0.496761025289554\\
0.0920920920920921	0.501666831997162\\
0.0930930930930931	0.506560614668723\\
0.0940940940940941	0.511442373304235\\
0.0950950950950951	0.5163121079037\\
0.0960960960960961	0.521169818467116\\
0.0970970970970971	0.526015504994484\\
0.0980980980980981	0.530849167485804\\
0.0990990990990991	0.535670805941076\\
0.1001001001001	0.5404804203603\\
0.101101101101101	0.545278010743476\\
0.102102102102102	0.550063577090604\\
0.103103103103103	0.554837119401684\\
0.104104104104104	0.559598637676716\\
0.105105105105105	0.5643481319157\\
0.106106106106106	0.569085602118635\\
0.107107107107107	0.573811048285523\\
0.108108108108108	0.578524470416362\\
0.109109109109109	0.583225868511154\\
0.11011011011011	0.587915242569897\\
0.111111111111111	0.592592592592593\\
0.112112112112112	0.59725791857924\\
0.113113113113113	0.601911220529839\\
0.114114114114114	0.60655249844439\\
0.115115115115115	0.611181752322894\\
0.116116116116116	0.615798982165348\\
0.117117117117117	0.620404187971755\\
0.118118118118118	0.624997369742114\\
0.119119119119119	0.629578527476425\\
0.12012012012012	0.634147661174688\\
0.121121121121121	0.638704770836903\\
0.122122122122122	0.64324985646307\\
0.123123123123123	0.647782918053188\\
0.124124124124124	0.652303955607259\\
0.125125125125125	0.656812969125281\\
0.126126126126126	0.661309958607256\\
0.127127127127127	0.665794924053182\\
0.128128128128128	0.670267865463061\\
0.129129129129129	0.674728782836891\\
0.13013013013013	0.679177676174673\\
0.131131131131131	0.683614545476407\\
0.132132132132132	0.688039390742093\\
0.133133133133133	0.692452211971731\\
0.134134134134134	0.696853009165322\\
0.135135135135135	0.701241782322864\\
0.136136136136136	0.705618531444357\\
0.137137137137137	0.709983256529803\\
0.138138138138138	0.714335957579201\\
0.139139139139139	0.718676634592551\\
0.14014014014014	0.723005287569852\\
0.141141141141141	0.727321916511106\\
0.142142142142142	0.731626521416311\\
0.143143143143143	0.735919102285469\\
0.144144144144144	0.740199659118578\\
0.145145145145145	0.74446819191564\\
0.146146146146146	0.748724700676652\\
0.147147147147147	0.752969185401618\\
0.148148148148148	0.757201646090535\\
0.149149149149149	0.761422082743404\\
0.15015015015015	0.765630495360225\\
0.151151151151151	0.769826883940998\\
0.152152152152152	0.774011248485723\\
0.153153153153153	0.7781835889944\\
0.154154154154154	0.782343905467029\\
0.155155155155155	0.786492197903609\\
0.156156156156156	0.790628466304142\\
0.157157157157157	0.794752710668627\\
0.158158158158158	0.798864930997063\\
0.159159159159159	0.802965127289452\\
0.16016016016016	0.807053299545792\\
0.161161161161161	0.811129447766084\\
0.162162162162162	0.815193571950329\\
0.163163163163163	0.819245672098525\\
0.164164164164164	0.823285748210673\\
0.165165165165165	0.827313800286773\\
0.166166166166166	0.831329828326825\\
0.167167167167167	0.835333832330829\\
0.168168168168168	0.839325812298785\\
0.169169169169169	0.843305768230693\\
0.17017017017017	0.847273700126553\\
0.171171171171171	0.851229607986365\\
0.172172172172172	0.855173491810128\\
0.173173173173173	0.859105351597844\\
0.174174174174174	0.863025187349512\\
0.175175175175175	0.866932999065131\\
0.176176176176176	0.870828786744703\\
0.177177177177177	0.874712550388226\\
0.178178178178178	0.878584289995701\\
0.179179179179179	0.882444005567129\\
0.18018018018018	0.886291697102508\\
0.181181181181181	0.890127364601839\\
0.182182182182182	0.893951008065122\\
0.183183183183183	0.897762627492357\\
0.184184184184184	0.901562222883544\\
0.185185185185185	0.905349794238683\\
0.186186186186186	0.909125341557774\\
0.187187187187187	0.912888864840817\\
0.188188188188188	0.916640364087812\\
0.189189189189189	0.920379839298758\\
0.19019019019019	0.924107290473657\\
0.191191191191191	0.927822717612508\\
0.192192192192192	0.93152612071531\\
0.193193193193193	0.935217499782064\\
0.194194194194194	0.938896854812771\\
0.195195195195195	0.942564185807429\\
0.196196196196196	0.946219492766039\\
0.197197197197197	0.949862775688602\\
0.198198198198198	0.953494034575116\\
0.199199199199199	0.957113269425582\\
0.2002002002002	0.96072048024\\
0.201201201201201	0.96431566701837\\
0.202202202202202	0.967898829760692\\
0.203203203203203	0.971469968466966\\
0.204204204204204	0.975029083137191\\
0.205205205205205	0.978576173771369\\
0.206206206206206	0.982111240369498\\
0.207207207207207	0.98563428293158\\
0.208208208208208	0.989145301457614\\
0.209209209209209	0.992644295947599\\
0.21021021021021	0.996131266401537\\
0.211211211211211	0.999606212819426\\
0.212212212212212	1.00306913520127\\
0.213213213213213	1.00652003354706\\
0.214214214214214	1.00995890785681\\
0.215215215215215	1.0133857581305\\
0.216216216216216	1.01680058436815\\
0.217217217217217	1.02020338656975\\
0.218218218218218	1.02359416473531\\
0.219219219219219	1.02697291886481\\
0.22022022022022	1.03033964895827\\
0.221221221221221	1.03369435501568\\
0.222222222222222	1.03703703703704\\
0.223223223223223	1.04036769502235\\
0.224224224224224	1.04368632897161\\
0.225225225225225	1.04699293888483\\
0.226226226226226	1.050287524762\\
0.227227227227227	1.05357008660312\\
0.228228228228228	1.05684062440819\\
0.229229229229229	1.06009913817722\\
0.23023023023023	1.06334562791019\\
0.231231231231231	1.06658009360712\\
0.232232232232232	1.069802535268\\
0.233233233233233	1.07301295289283\\
0.234234234234234	1.07621134648162\\
0.235235235235235	1.07939771603435\\
0.236236236236236	1.08257206155104\\
0.237237237237237	1.08573438303168\\
0.238238238238238	1.08888468047627\\
0.239239239239239	1.09202295388482\\
0.24024024024024	1.09514920325731\\
0.241241241241241	1.09826342859376\\
0.242242242242242	1.10136562989416\\
0.243243243243243	1.10445580715851\\
0.244244244244244	1.10753396038681\\
0.245245245245245	1.11060008957907\\
0.246246246246246	1.11365419473528\\
0.247247247247247	1.11669627585544\\
0.248248248248248	1.11972633293955\\
0.249249249249249	1.12274436598761\\
0.25025025025025	1.12575037499962\\
0.251251251251251	1.12874435997559\\
0.252252252252252	1.13172632091551\\
0.253253253253253	1.13469625781938\\
0.254254254254254	1.1376541706872\\
0.255255255255255	1.14060005951898\\
0.256256256256256	1.14353392431471\\
0.257257257257257	1.14645576507438\\
0.258258258258258	1.14936558179801\\
0.259259259259259	1.1522633744856\\
0.26026026026026	1.15514914313713\\
0.261261261261261	1.15802288775262\\
0.262262262262262	1.16088460833206\\
0.263263263263263	1.16373430487545\\
0.264264264264264	1.16657197738279\\
0.265265265265265	1.16939762585408\\
0.266266266266266	1.17221125028933\\
0.267267267267267	1.17501285068853\\
0.268268268268268	1.17780242705168\\
0.269269269269269	1.18057997937878\\
0.27027027027027	1.18334550766983\\
0.271271271271271	1.18609901192484\\
0.272272272272272	1.1888404921438\\
0.273273273273273	1.19156994832671\\
0.274274274274274	1.19428738047357\\
0.275275275275275	1.19699278858438\\
0.276276276276276	1.19968617265915\\
0.277277277277277	1.20236753269786\\
0.278278278278278	1.20503686870053\\
0.279279279279279	1.20769418066715\\
0.28028028028028	1.21033946859773\\
0.281281281281281	1.21297273249225\\
0.282282282282282	1.21559397235073\\
0.283283283283283	1.21820318817316\\
0.284284284284284	1.22080037995954\\
0.285285285285285	1.22338554770987\\
0.286286286286286	1.22595869142416\\
0.287287287287287	1.22851981110239\\
0.288288288288288	1.23106890674458\\
0.289289289289289	1.23360597835072\\
0.29029029029029	1.23613102592082\\
0.291291291291291	1.23864404945486\\
0.292292292292292	1.24114504895286\\
0.293293293293293	1.24363402441481\\
0.294294294294294	1.24611097584071\\
0.295295295295295	1.24857590323056\\
0.296296296296296	1.25102880658436\\
0.297297297297297	1.25346968590212\\
0.298298298298298	1.25589854118383\\
0.299299299299299	1.25831537242949\\
0.3003003003003	1.2607201796391\\
0.301301301301301	1.26311296281266\\
0.302302302302302	1.26549372195018\\
0.303303303303303	1.26786245705165\\
0.304304304304304	1.27021916811707\\
0.305305305305305	1.27256385514644\\
0.306306306306306	1.27489651813976\\
0.307307307307307	1.27721715709704\\
0.308308308308308	1.27952577201826\\
0.309309309309309	1.28182236290344\\
0.31031031031031	1.28410692975258\\
0.311311311311311	1.28637947256566\\
0.312312312312312	1.28863999134269\\
0.313313313313313	1.29088848608368\\
0.314314314314314	1.29312495678862\\
0.315315315315315	1.29534940345751\\
0.316316316316316	1.29756182609035\\
0.317317317317317	1.29976222468715\\
0.318318318318318	1.3019505992479\\
0.319319319319319	1.3041269497726\\
0.32032032032032	1.30629127626125\\
0.321321321321321	1.30844357871385\\
0.322322322322322	1.3105838571304\\
0.323323323323323	1.31271211151091\\
0.324324324324324	1.31482834185537\\
0.325325325325325	1.31693254816378\\
0.326326326326326	1.31902473043614\\
0.327327327327327	1.32110488867246\\
0.328328328328328	1.32317302287272\\
0.329329329329329	1.32522913303694\\
0.33033033033033	1.32727321916511\\
0.331331331331331	1.32930528125723\\
0.332332332332332	1.33132531931331\\
0.333333333333333	1.33333333333333\\
0.334334334334334	1.33532932331731\\
0.335335335335335	1.33731328926524\\
0.336336336336336	1.33928523117712\\
0.337337337337337	1.34124514905296\\
0.338338338338338	1.34319304289274\\
0.339339339339339	1.34512891269648\\
0.34034034034034	1.34705275846417\\
0.341341341341341	1.34896458019581\\
0.342342342342342	1.35086437789141\\
0.343343343343343	1.35275215155095\\
0.344344344344344	1.35462790117445\\
0.345345345345345	1.3564916267619\\
0.346346346346346	1.3583433283133\\
0.347347347347347	1.36018300582865\\
0.348348348348348	1.36201065930796\\
0.349349349349349	1.36382628875121\\
0.35035035035035	1.36562989415842\\
0.351351351351351	1.36742147552958\\
0.352352352352352	1.3692010328647\\
0.353353353353353	1.37096856616376\\
0.354354354354354	1.37272407542678\\
0.355355355355355	1.37446756065375\\
0.356356356356356	1.37619902184467\\
0.357357357357357	1.37791845899954\\
0.358358358358358	1.37962587211836\\
0.359359359359359	1.38132126120114\\
0.36036036036036	1.38300462624787\\
0.361361361361361	1.38467596725855\\
0.362362362362362	1.38633528423318\\
0.363363363363363	1.38798257717177\\
0.364364364364364	1.3896178460743\\
0.365365365365365	1.39124109094079\\
0.366366366366366	1.39285231177123\\
0.367367367367367	1.39445150856562\\
0.368368368368368	1.39603868132397\\
0.369369369369369	1.39761383004626\\
0.37037037037037	1.39917695473251\\
0.371371371371371	1.40072805538271\\
0.372372372372372	1.40226713199686\\
0.373373373373373	1.40379418457497\\
0.374374374374374	1.40530921311702\\
0.375375375375375	1.40681221762303\\
0.376376376376376	1.40830319809299\\
0.377377377377377	1.4097821545269\\
0.378378378378378	1.41124908692476\\
0.379379379379379	1.41270399528658\\
0.38038038038038	1.41414687961234\\
0.381381381381381	1.41557773990206\\
0.382382382382382	1.41699657615574\\
0.383383383383383	1.41840338837336\\
0.384384384384384	1.41979817655493\\
0.385385385385385	1.42118094070046\\
0.386386386386386	1.42255168080994\\
0.387387387387387	1.42391039688337\\
0.388388388388388	1.42525708892075\\
0.389389389389389	1.42659175692209\\
0.39039039039039	1.42791440088737\\
0.391391391391391	1.42922502081661\\
0.392392392392392	1.4305236167098\\
0.393393393393393	1.43181018856695\\
0.394394394394394	1.43308473638804\\
0.395395395395395	1.43434726017309\\
0.396396396396396	1.43559775992208\\
0.397397397397397	1.43683623563503\\
0.398398398398398	1.43806268731194\\
0.399399399399399	1.43927711495279\\
0.4004004004004	1.4404795185576\\
0.401401401401401	1.44166989812635\\
0.402402402402402	1.44284825365906\\
0.403403403403403	1.44401458515573\\
0.404404404404404	1.44516889261634\\
0.405405405405405	1.44631117604091\\
0.406406406406406	1.44744143542942\\
0.407407407407407	1.44855967078189\\
0.408408408408408	1.44966588209831\\
0.409409409409409	1.45076006937869\\
0.41041041041041	1.45184223262301\\
0.411411411411411	1.45291237183129\\
0.412412412412412	1.45397048700352\\
0.413413413413413	1.4550165781397\\
0.414414414414414	1.45605064523983\\
0.415415415415415	1.45707268830392\\
0.416416416416416	1.45808270733196\\
0.417417417417417	1.45908070232395\\
0.418418418418418	1.46006667327989\\
0.419419419419419	1.46104062019978\\
0.42042042042042	1.46200254308362\\
0.421421421421421	1.46295244193142\\
0.422422422422422	1.46389031674317\\
0.423423423423423	1.46481616751887\\
0.424424424424424	1.46572999425852\\
0.425425425425425	1.46663179696213\\
0.426426426426426	1.46752157562968\\
0.427427427427427	1.46839933026119\\
0.428428428428428	1.46926506085665\\
0.429429429429429	1.47011876741606\\
0.43043043043043	1.47096044993943\\
0.431431431431431	1.47179010842675\\
0.432432432432432	1.47260774287801\\
0.433433433433433	1.47341335329323\\
0.434434434434434	1.4742069396724\\
0.435435435435435	1.47498850201553\\
0.436436436436436	1.47575804032261\\
0.437437437437437	1.47651555459363\\
0.438438438438438	1.47726104482861\\
0.439439439439439	1.47799451102754\\
0.44044044044044	1.47871595319043\\
0.441441441441441	1.47942537131726\\
0.442442442442442	1.48012276540805\\
0.443443443443443	1.48080813546279\\
0.444444444444444	1.48148148148148\\
0.445445445445445	1.48214280346412\\
0.446446446446446	1.48279210141072\\
0.447447447447447	1.48342937532127\\
0.448448448448448	1.48405462519577\\
0.449449449449449	1.48466785103422\\
0.45045045045045	1.48526905283662\\
0.451451451451451	1.48585823060298\\
0.452452452452452	1.48643538433328\\
0.453453453453453	1.48700051402754\\
0.454454454454454	1.48755361968575\\
0.455455455455455	1.48809470130791\\
0.456456456456456	1.48862375889403\\
0.457457457457457	1.4891407924441\\
0.458458458458458	1.48964580195811\\
0.459459459459459	1.49013878743608\\
0.46046046046046	1.49061974887801\\
0.461461461461461	1.49108868628388\\
0.462462462462462	1.49154559965371\\
0.463463463463463	1.49199048898749\\
0.464464464464464	1.49242335428522\\
0.465465465465465	1.4928441955469\\
0.466466466466466	1.49325301277253\\
0.467467467467467	1.49364980596212\\
0.468468468468468	1.49403457511566\\
0.469469469469469	1.49440732023315\\
0.47047047047047	1.49476804131459\\
0.471471471471471	1.49511673835998\\
0.472472472472472	1.49545341136933\\
0.473473473473473	1.49577806034262\\
0.474474474474474	1.49609068527987\\
0.475475475475475	1.49639128618108\\
0.476476476476476	1.49667986304623\\
0.477477477477477	1.49695641587533\\
0.478478478478478	1.49722094466839\\
0.479479479479479	1.4974734494254\\
0.48048048048048	1.49771393014636\\
0.481481481481481	1.49794238683128\\
0.482482482482482	1.49815881948014\\
0.483483483483483	1.49836322809296\\
0.484484484484485	1.49855561266973\\
0.485485485485485	1.49873597321045\\
0.486486486486487	1.49890430971512\\
0.487487487487487	1.49906062218375\\
0.488488488488488	1.49920491061632\\
0.48948948948949	1.49933717501285\\
0.49049049049049	1.49945741537333\\
0.491491491491492	1.49956563169776\\
0.492492492492492	1.49966182398615\\
0.493493493493493	1.49974599223848\\
0.494494494494495	1.49981813645477\\
0.495495495495495	1.49987825663501\\
0.496496496496497	1.49992635277921\\
0.497497497497497	1.49996242488735\\
0.498498498498498	1.49998647295945\\
0.4994994994995	1.49999849699549\\
0.500500500500501	1.49999849699549\\
0.501501501501502	1.49998647295945\\
0.502502502502503	1.49996242488735\\
0.503503503503503	1.49992635277921\\
0.504504504504504	1.49987825663501\\
0.505505505505506	1.49981813645477\\
0.506506506506507	1.49974599223848\\
0.507507507507508	1.49966182398615\\
0.508508508508508	1.49956563169776\\
0.509509509509509	1.49945741537333\\
0.510510510510511	1.49933717501285\\
0.511511511511512	1.49920491061632\\
0.512512512512513	1.49906062218375\\
0.513513513513513	1.49890430971512\\
0.514514514514514	1.49873597321045\\
0.515515515515516	1.49855561266973\\
0.516516516516517	1.49836322809296\\
0.517517517517518	1.49815881948014\\
0.518518518518518	1.49794238683128\\
0.519519519519519	1.49771393014636\\
0.520520520520521	1.4974734494254\\
0.521521521521522	1.49722094466839\\
0.522522522522523	1.49695641587533\\
0.523523523523523	1.49667986304623\\
0.524524524524524	1.49639128618108\\
0.525525525525526	1.49609068527987\\
0.526526526526527	1.49577806034262\\
0.527527527527528	1.49545341136933\\
0.528528528528528	1.49511673835998\\
0.529529529529529	1.49476804131459\\
0.530530530530531	1.49440732023315\\
0.531531531531532	1.49403457511566\\
0.532532532532533	1.49364980596212\\
0.533533533533533	1.49325301277253\\
0.534534534534535	1.4928441955469\\
0.535535535535536	1.49242335428522\\
0.536536536536537	1.49199048898749\\
0.537537537537538	1.49154559965371\\
0.538538538538539	1.49108868628388\\
0.53953953953954	1.49061974887801\\
0.540540540540541	1.49013878743608\\
0.541541541541542	1.48964580195811\\
0.542542542542543	1.4891407924441\\
0.543543543543544	1.48862375889403\\
0.544544544544545	1.48809470130791\\
0.545545545545546	1.48755361968575\\
0.546546546546547	1.48700051402754\\
0.547547547547548	1.48643538433328\\
0.548548548548549	1.48585823060298\\
0.54954954954955	1.48526905283662\\
0.550550550550551	1.48466785103422\\
0.551551551551552	1.48405462519577\\
0.552552552552553	1.48342937532127\\
0.553553553553554	1.48279210141072\\
0.554554554554555	1.48214280346412\\
0.555555555555556	1.48148148148148\\
0.556556556556557	1.48080813546279\\
0.557557557557558	1.48012276540805\\
0.558558558558559	1.47942537131726\\
0.55955955955956	1.47871595319043\\
0.560560560560561	1.47799451102754\\
0.561561561561562	1.47726104482861\\
0.562562562562563	1.47651555459363\\
0.563563563563564	1.4757580403226\\
0.564564564564565	1.47498850201553\\
0.565565565565566	1.4742069396724\\
0.566566566566567	1.47341335329323\\
0.567567567567568	1.47260774287801\\
0.568568568568569	1.47179010842675\\
0.56956956956957	1.47096044993943\\
0.570570570570571	1.47011876741606\\
0.571571571571572	1.46926506085665\\
0.572572572572573	1.46839933026119\\
0.573573573573574	1.46752157562968\\
0.574574574574575	1.46663179696213\\
0.575575575575576	1.46572999425852\\
0.576576576576577	1.46481616751887\\
0.577577577577578	1.46389031674317\\
0.578578578578579	1.46295244193142\\
0.57957957957958	1.46200254308362\\
0.580580580580581	1.46104062019978\\
0.581581581581582	1.46006667327989\\
0.582582582582583	1.45908070232395\\
0.583583583583584	1.45808270733196\\
0.584584584584585	1.45707268830392\\
0.585585585585586	1.45605064523983\\
0.586586586586587	1.4550165781397\\
0.587587587587588	1.45397048700352\\
0.588588588588589	1.45291237183129\\
0.58958958958959	1.45184223262301\\
0.590590590590591	1.45076006937869\\
0.591591591591592	1.44966588209831\\
0.592592592592593	1.44855967078189\\
0.593593593593594	1.44744143542942\\
0.594594594594595	1.44631117604091\\
0.595595595595596	1.44516889261634\\
0.596596596596597	1.44401458515573\\
0.597597597597598	1.44284825365906\\
0.598598598598599	1.44166989812635\\
0.5995995995996	1.4404795185576\\
0.600600600600601	1.43927711495279\\
0.601601601601602	1.43806268731194\\
0.602602602602603	1.43683623563503\\
0.603603603603604	1.43559775992208\\
0.604604604604605	1.43434726017309\\
0.605605605605606	1.43308473638804\\
0.606606606606607	1.43181018856695\\
0.607607607607608	1.4305236167098\\
0.608608608608609	1.42922502081661\\
0.60960960960961	1.42791440088737\\
0.610610610610611	1.42659175692209\\
0.611611611611612	1.42525708892075\\
0.612612612612613	1.42391039688337\\
0.613613613613614	1.42255168080994\\
0.614614614614615	1.42118094070046\\
0.615615615615616	1.41979817655493\\
0.616616616616617	1.41840338837336\\
0.617617617617618	1.41699657615574\\
0.618618618618619	1.41557773990206\\
0.61961961961962	1.41414687961234\\
0.620620620620621	1.41270399528658\\
0.621621621621622	1.41124908692476\\
0.622622622622623	1.4097821545269\\
0.623623623623624	1.40830319809299\\
0.624624624624625	1.40681221762303\\
0.625625625625626	1.40530921311702\\
0.626626626626627	1.40379418457497\\
0.627627627627628	1.40226713199686\\
0.628628628628629	1.40072805538271\\
0.62962962962963	1.39917695473251\\
0.630630630630631	1.39761383004626\\
0.631631631631632	1.39603868132397\\
0.632632632632633	1.39445150856562\\
0.633633633633634	1.39285231177123\\
0.634634634634635	1.39124109094079\\
0.635635635635636	1.3896178460743\\
0.636636636636637	1.38798257717177\\
0.637637637637638	1.38633528423318\\
0.638638638638639	1.38467596725855\\
0.63963963963964	1.38300462624787\\
0.640640640640641	1.38132126120114\\
0.641641641641642	1.37962587211836\\
0.642642642642643	1.37791845899954\\
0.643643643643644	1.37619902184467\\
0.644644644644645	1.37446756065375\\
0.645645645645646	1.37272407542678\\
0.646646646646647	1.37096856616376\\
0.647647647647648	1.3692010328647\\
0.648648648648649	1.36742147552958\\
0.64964964964965	1.36562989415842\\
0.650650650650651	1.36382628875121\\
0.651651651651652	1.36201065930796\\
0.652652652652653	1.36018300582865\\
0.653653653653654	1.3583433283133\\
0.654654654654655	1.3564916267619\\
0.655655655655656	1.35462790117445\\
0.656656656656657	1.35275215155095\\
0.657657657657658	1.3508643778914\\
0.658658658658659	1.34896458019581\\
0.65965965965966	1.34705275846417\\
0.660660660660661	1.34512891269648\\
0.661661661661662	1.34319304289274\\
0.662662662662663	1.34124514905296\\
0.663663663663664	1.33928523117712\\
0.664664664664665	1.33731328926524\\
0.665665665665666	1.33532932331731\\
0.666666666666667	1.33333333333333\\
0.667667667667668	1.33132531931331\\
0.668668668668669	1.32930528125723\\
0.66966966966967	1.32727321916511\\
0.670670670670671	1.32522913303694\\
0.671671671671672	1.32317302287272\\
0.672672672672673	1.32110488867246\\
0.673673673673674	1.31902473043614\\
0.674674674674675	1.31693254816378\\
0.675675675675676	1.31482834185537\\
0.676676676676677	1.31271211151091\\
0.677677677677678	1.3105838571304\\
0.678678678678679	1.30844357871385\\
0.67967967967968	1.30629127626125\\
0.680680680680681	1.3041269497726\\
0.681681681681682	1.3019505992479\\
0.682682682682683	1.29976222468715\\
0.683683683683684	1.29756182609035\\
0.684684684684685	1.29534940345751\\
0.685685685685686	1.29312495678862\\
0.686686686686687	1.29088848608368\\
0.687687687687688	1.28863999134269\\
0.688688688688689	1.28637947256566\\
0.68968968968969	1.28410692975258\\
0.690690690690691	1.28182236290344\\
0.691691691691692	1.27952577201826\\
0.692692692692693	1.27721715709704\\
0.693693693693694	1.27489651813976\\
0.694694694694695	1.27256385514644\\
0.695695695695696	1.27021916811707\\
0.696696696696697	1.26786245705165\\
0.697697697697698	1.26549372195018\\
0.698698698698699	1.26311296281266\\
0.6996996996997	1.2607201796391\\
0.700700700700701	1.25831537242949\\
0.701701701701702	1.25589854118383\\
0.702702702702703	1.25346968590212\\
0.703703703703704	1.25102880658436\\
0.704704704704705	1.24857590323056\\
0.705705705705706	1.24611097584071\\
0.706706706706707	1.24363402441481\\
0.707707707707708	1.24114504895286\\
0.708708708708709	1.23864404945486\\
0.70970970970971	1.23613102592082\\
0.710710710710711	1.23360597835072\\
0.711711711711712	1.23106890674458\\
0.712712712712713	1.22851981110239\\
0.713713713713714	1.22595869142416\\
0.714714714714715	1.22338554770987\\
0.715715715715716	1.22080037995954\\
0.716716716716717	1.21820318817316\\
0.717717717717718	1.21559397235073\\
0.718718718718719	1.21297273249225\\
0.71971971971972	1.21033946859773\\
0.720720720720721	1.20769418066715\\
0.721721721721722	1.20503686870053\\
0.722722722722723	1.20236753269786\\
0.723723723723724	1.19968617265915\\
0.724724724724725	1.19699278858438\\
0.725725725725726	1.19428738047357\\
0.726726726726727	1.1915699483267\\
0.727727727727728	1.1888404921438\\
0.728728728728729	1.18609901192484\\
0.72972972972973	1.18334550766983\\
0.730730730730731	1.18057997937878\\
0.731731731731732	1.17780242705168\\
0.732732732732733	1.17501285068853\\
0.733733733733734	1.17221125028933\\
0.734734734734735	1.16939762585408\\
0.735735735735736	1.16657197738279\\
0.736736736736737	1.16373430487545\\
0.737737737737738	1.16088460833206\\
0.738738738738739	1.15802288775262\\
0.73973973973974	1.15514914313713\\
0.740740740740741	1.1522633744856\\
0.741741741741742	1.14936558179801\\
0.742742742742743	1.14645576507438\\
0.743743743743744	1.14353392431471\\
0.744744744744745	1.14060005951898\\
0.745745745745746	1.1376541706872\\
0.746746746746747	1.13469625781938\\
0.747747747747748	1.13172632091551\\
0.748748748748749	1.12874435997559\\
0.74974974974975	1.12575037499962\\
0.750750750750751	1.12274436598761\\
0.751751751751752	1.11972633293955\\
0.752752752752753	1.11669627585543\\
0.753753753753754	1.11365419473528\\
0.754754754754755	1.11060008957907\\
0.755755755755756	1.10753396038681\\
0.756756756756757	1.10445580715851\\
0.757757757757758	1.10136562989416\\
0.758758758758759	1.09826342859376\\
0.75975975975976	1.09514920325731\\
0.760760760760761	1.09202295388482\\
0.761761761761762	1.08888468047627\\
0.762762762762763	1.08573438303168\\
0.763763763763764	1.08257206155104\\
0.764764764764765	1.07939771603435\\
0.765765765765766	1.07621134648162\\
0.766766766766767	1.07301295289283\\
0.767767767767768	1.069802535268\\
0.768768768768769	1.06658009360712\\
0.76976976976977	1.06334562791019\\
0.770770770770771	1.06009913817722\\
0.771771771771772	1.05684062440819\\
0.772772772772773	1.05357008660312\\
0.773773773773774	1.050287524762\\
0.774774774774775	1.04699293888483\\
0.775775775775776	1.04368632897161\\
0.776776776776777	1.04036769502235\\
0.777777777777778	1.03703703703704\\
0.778778778778779	1.03369435501568\\
0.77977977977978	1.03033964895827\\
0.780780780780781	1.02697291886481\\
0.781781781781782	1.02359416473531\\
0.782782782782783	1.02020338656975\\
0.783783783783784	1.01680058436815\\
0.784784784784785	1.0133857581305\\
0.785785785785786	1.00995890785681\\
0.786786786786787	1.00652003354706\\
0.787787787787788	1.00306913520127\\
0.788788788788789	0.999606212819426\\
0.78978978978979	0.996131266401537\\
0.790790790790791	0.992644295947599\\
0.791791791791792	0.989145301457614\\
0.792792792792793	0.98563428293158\\
0.793793793793794	0.982111240369498\\
0.794794794794795	0.978576173771369\\
0.795795795795796	0.975029083137191\\
0.796796796796797	0.971469968466965\\
0.797797797797798	0.967898829760692\\
0.798798798798799	0.96431566701837\\
0.7997997997998	0.96072048024\\
0.800800800800801	0.957113269425582\\
0.801801801801802	0.953494034575115\\
0.802802802802803	0.949862775688602\\
0.803803803803804	0.946219492766039\\
0.804804804804805	0.942564185807429\\
0.805805805805806	0.938896854812771\\
0.806806806806807	0.935217499782064\\
0.807807807807808	0.93152612071531\\
0.808808808808809	0.927822717612508\\
0.80980980980981	0.924107290473657\\
0.810810810810811	0.920379839298758\\
0.811811811811812	0.916640364087811\\
0.812812812812813	0.912888864840817\\
0.813813813813814	0.909125341557774\\
0.814814814814815	0.905349794238683\\
0.815815815815816	0.901562222883544\\
0.816816816816817	0.897762627492357\\
0.817817817817818	0.893951008065122\\
0.818818818818819	0.890127364601839\\
0.81981981981982	0.886291697102508\\
0.820820820820821	0.882444005567129\\
0.821821821821822	0.878584289995701\\
0.822822822822823	0.874712550388226\\
0.823823823823824	0.870828786744703\\
0.824824824824825	0.866932999065131\\
0.825825825825826	0.863025187349511\\
0.826826826826827	0.859105351597844\\
0.827827827827828	0.855173491810128\\
0.828828828828829	0.851229607986365\\
0.82982982982983	0.847273700126553\\
0.830830830830831	0.843305768230693\\
0.831831831831832	0.839325812298785\\
0.832832832832833	0.835333832330829\\
0.833833833833834	0.831329828326825\\
0.834834834834835	0.827313800286773\\
0.835835835835836	0.823285748210673\\
0.836836836836837	0.819245672098525\\
0.837837837837838	0.815193571950329\\
0.838838838838839	0.811129447766084\\
0.83983983983984	0.807053299545792\\
0.840840840840841	0.802965127289452\\
0.841841841841842	0.798864930997063\\
0.842842842842843	0.794752710668627\\
0.843843843843844	0.790628466304142\\
0.844844844844845	0.786492197903609\\
0.845845845845846	0.782343905467028\\
0.846846846846847	0.7781835889944\\
0.847847847847848	0.774011248485723\\
0.848848848848849	0.769826883940998\\
0.84984984984985	0.765630495360225\\
0.850850850850851	0.761422082743404\\
0.851851851851852	0.757201646090535\\
0.852852852852853	0.752969185401618\\
0.853853853853854	0.748724700676653\\
0.854854854854855	0.744468191915639\\
0.855855855855856	0.740199659118578\\
0.856856856856857	0.735919102285469\\
0.857857857857858	0.731626521416311\\
0.858858858858859	0.727321916511106\\
0.85985985985986	0.723005287569852\\
0.860860860860861	0.71867663459255\\
0.861861861861862	0.714335957579201\\
0.862862862862863	0.709983256529803\\
0.863863863863864	0.705618531444357\\
0.864864864864865	0.701241782322863\\
0.865865865865866	0.696853009165321\\
0.866866866866867	0.692452211971731\\
0.867867867867868	0.688039390742094\\
0.868868868868869	0.683614545476407\\
0.86986986986987	0.679177676174673\\
0.870870870870871	0.674728782836891\\
0.871871871871872	0.670267865463061\\
0.872872872872873	0.665794924053182\\
0.873873873873874	0.661309958607256\\
0.874874874874875	0.656812969125281\\
0.875875875875876	0.652303955607259\\
0.876876876876877	0.647782918053188\\
0.877877877877878	0.64324985646307\\
0.878878878878879	0.638704770836903\\
0.87987987987988	0.634147661174688\\
0.880880880880881	0.629578527476425\\
0.881881881881882	0.624997369742114\\
0.882882882882883	0.620404187971756\\
0.883883883883884	0.615798982165349\\
0.884884884884885	0.611181752322893\\
0.885885885885886	0.60655249844439\\
0.886886886886887	0.601911220529839\\
0.887887887887888	0.59725791857924\\
0.888888888888889	0.592592592592593\\
0.88988988988989	0.587915242569897\\
0.890890890890891	0.583225868511154\\
0.891891891891892	0.578524470416362\\
0.892892892892893	0.573811048285523\\
0.893893893893894	0.569085602118635\\
0.894894894894895	0.564348131915699\\
0.895895895895896	0.559598637676716\\
0.896896896896897	0.554837119401684\\
0.897897897897898	0.550063577090604\\
0.898898898898899	0.545278010743477\\
0.8998998998999	0.5404804203603\\
0.900900900900901	0.535670805941076\\
0.901901901901902	0.530849167485804\\
0.902902902902903	0.526015504994484\\
0.903903903903904	0.521169818467116\\
0.904904904904905	0.516312107903699\\
0.905905905905906	0.511442373304235\\
0.906906906906907	0.506560614668723\\
0.907907907907908	0.501666831997162\\
0.908908908908909	0.496761025289553\\
0.90990990990991	0.491843194545897\\
0.910910910910911	0.486913339766192\\
0.911911911911912	0.48197146095044\\
0.912912912912913	0.477017558098639\\
0.913913913913914	0.47205163121079\\
0.914914914914915	0.467073680286893\\
0.915915915915916	0.462083705326949\\
0.916916916916917	0.457081706330956\\
0.917917917917918	0.452067683298915\\
0.918918918918919	0.447041636230825\\
0.91991991991992	0.442003565126688\\
0.920920920920921	0.436953469986503\\
0.921921921921922	0.43189135081027\\
0.922922922922923	0.426817207597988\\
0.923923923923924	0.421731040349659\\
0.924924924924925	0.416632849065281\\
0.925925925925926	0.411522633744856\\
0.926926926926927	0.406400394388383\\
0.927927927927928	0.401266130995861\\
0.928928928928929	0.396119843567291\\
0.92992992992993	0.390961532102673\\
0.930930930930931	0.385791196602007\\
0.931931931931932	0.380608837065294\\
0.932932932932933	0.375414453492532\\
0.933933933933934	0.370208045883721\\
0.934934934934935	0.364989614238863\\
0.935935935935936	0.359759158557957\\
0.936936936936937	0.354516678841003\\
0.937937937937938	0.349262175088001\\
0.938938938938939	0.34399564729895\\
0.93993993993994	0.338717095473852\\
0.940940940940941	0.333426519612706\\
0.941941941941942	0.328123919715511\\
0.942942942942943	0.322809295782269\\
0.943943943943944	0.317482647812978\\
0.944944944944945	0.312143975807639\\
0.945945945945946	0.306793279766253\\
0.946946946946947	0.301430559688818\\
0.947947947947948	0.296055815575335\\
0.948948948948949	0.290669047425804\\
0.94994994994995	0.285270255240225\\
0.950950950950951	0.279859439018598\\
0.951951951951952	0.274436598760923\\
0.952952952952953	0.2690017344672\\
0.953953953953954	0.263554846137429\\
0.954954954954955	0.258095933771609\\
0.955955955955956	0.252624997369742\\
0.956956956956957	0.247142036931827\\
0.957957957957958	0.241647052457864\\
0.958958958958959	0.236140043947851\\
0.95995995995996	0.230621011401792\\
0.960960960960961	0.225089954819685\\
0.961961961961962	0.219546874201529\\
0.962962962962963	0.213991769547325\\
0.963963963963964	0.208424640857073\\
0.964964964964965	0.202845488130773\\
0.965965965965966	0.197254311368425\\
0.966966966966967	0.19165111057003\\
0.967967967967968	0.186035885735586\\
0.968968968968969	0.180408636865093\\
0.96996996996997	0.174769363958553\\
0.970970970970971	0.169118067015965\\
0.971971971971972	0.163454746037329\\
0.972972972972973	0.157779401022644\\
0.973973973973974	0.152092031971912\\
0.974974974974975	0.146392638885131\\
0.975975975975976	0.140681221762303\\
0.976976976976977	0.134957780603426\\
0.977977977977978	0.129222315408501\\
0.978978978978979	0.123474826177529\\
0.97997997997998	0.117715312910508\\
0.980980980980981	0.111943775607439\\
0.981981981981982	0.106160214268323\\
0.982982982982983	0.100364628893157\\
0.983983983983984	0.0945570194819443\\
0.984984984984985	0.0887373860346834\\
0.985985985985986	0.0829057285513743\\
0.986986986986987	0.0770620470320172\\
0.987987987987988	0.0712063414766114\\
0.988988988988989	0.0653386118851583\\
0.98998998998999	0.059458858257657\\
0.990990990990991	0.0535670805941077\\
0.991991991991992	0.0476632788945104\\
0.992992992992993	0.0417474531588643\\
0.993993993993994	0.0358196033871708\\
0.994994994994995	0.0298797295794293\\
0.995995995995996	0.0239278317356397\\
0.996996996996997	0.017963909855802\\
0.997997997997998	0.0119879639399156\\
0.998998998998999	0.00599999398798184\\
1	0\\
};
\addlegendentry{$p(\theta_1) = p(\theta_1 \given \text{H})$};

\end{axis}
\end{tikzpicture}%
  \caption{The prior and posterior (given the observation H) of
    $\theta_1$.}
  \label{problem_1_prior_1}
\end{figure}

There are two ways to compute the joint posterior over $(\theta_1,
\theta_2)$: the easy way and the hard way.  The easy way is to use the
sum rule again to write
\begin{align*}
  p(\theta_1, \theta_2 \given \text{H})
  &=
  \Pr(\text{HT})
  p(\theta_1, \theta_2 \given \text{HT})
  +
  \Pr(\text{TH})
  p(\theta_1, \theta_2 \given \text{TH})
  \\
  &=
  \tfrac{1}{2}
  \mc{B}(\theta_1; 3, 2)
  \mc{B}(\theta_2; 2, 3)
  +
  \tfrac{1}{2}
  \mc{B}(\theta_1; 2, 3)
  \mc{B}(\theta_2; 3, 2).
\end{align*}

If we go with the hard way, we begin by computing the joint prior.  By
independence, we have:
\begin{equation*}
  p(\theta_1, \theta_2)
  =
  \mc{B}(\theta_1; 2, 2)
  \mc{B}(\theta_2; 2, 2).
\end{equation*}
To derive the likelihood, we again note that our observation could
have been generated by two mutually independent events: HT or TH.
Given $\theta_1$ and $\theta_2$, the total probability of these
events is:
\begin{equation*}
  \Pr(\text{H} \given \theta_1, \theta_2)
  =
  \theta_1(1 - \theta_2)
  +
  (1 - \theta_1)\theta_2;
\end{equation*}
the first term accounts for HT and the second term for TH.  The
posterior is now:
\begin{align*}
  p(\theta_1, \theta_2 \given \text{H})
  &=
  \tfrac{1}{Z}
  \Pr(\text{H} \given \theta_1, \theta_2)
  p(\theta_1, \theta_2)
  \\
  &=
  \tfrac{1}{Z}
  \bigl(
  \theta_1(1 - \theta_2)
  +
  (1 - \theta_1)\theta_2
  \bigr)
  \mc{B}(\theta_1; 2, 2)
  \mc{B}(\theta_2; 2, 2).
\end{align*}
The normalization constant is
\begin{equation*}
  Z
  =
  \Pr(\text{H})
  =
  \int_0^1
  \int_0^1
  \bigl(
  \theta_1(1 - \theta_2)
  +
  (1 - \theta_1)\theta_2
  \bigr)
  \mc{B}(\theta_1; 2, 2)
  \mc{B}(\theta_2; 2, 2)
  \intd{\theta_1}
  \intd{\theta_2}.
\end{equation*}
This integral is tractable and equals
\nicefrac{1}{2}.\footnote{\url{http://goo.gl/wRofuX}} In fact, this is
always true for any arbitrary mean-\nicefrac{1}{2} beta priors on
$\theta_1, \theta_2$: if our best guess is that each coin is unbiased,
then the outcomes HT/TH always have equal combined probability as the
outcomes HH/TT.

The posterior is now
\begin{equation*}
  p(\theta_1, \theta_2 \given \text{H})
  =
  2
  \bigl(
  \theta_1(1 - \theta_2)
  +
  (1 - \theta_1)\theta_2
  \bigr)
  \mc{B}(\theta_1; 2, 2)
  \mc{B}(\theta_2; 2, 2).
\end{equation*}
This is equivalent to the expression we derived with ``the easy way.''

The prior, likelihood, and posterior are plotted below.  From the
posterior, we can see that joint probabilities corresponding to
jointly low or high values: these combinations would correspond to a
higher probability of seeing either HH or TT observations.  Despite
the marginals for $\theta_1$ and $\theta_2$ remaining unchanged, the H
observation has entangled the previously independent beliefs in the
anticorrelated posterior.

\begin{figure}
  \centering
  \input{figures/problem_1_joint_prior.tex}
  \caption{The joint prior $p(\theta_1, \theta_2)$.}
  \label{problem_joint_prior}
\end{figure}

\begin{figure}
  \centering
  \input{figures/problem_1_joint_likelihood.tex}
  \caption{The joint likelihood $p(\text{H} \given \theta_1, \theta_2)$.}
  \label{problem_joint_likelihood}
\end{figure}

\begin{figure}
  \centering
  % This file was created by matlab2tikz.
% Minimal pgfplots version: 1.3
%
\tikzsetnextfilename{problem_1_joint_posterior}
\begin{tikzpicture}

\begin{axis}[%
width=0.825873\squarefigurewidth,
height=\squarefigureheight,
at={(0\squarefigurewidth,0\squarefigureheight)},
scale only axis,
axis on top,
xmin=-0.0005005005005005,
xmax=1.0005005005005,
xlabel={$\theta_1$},
ymin=-0.0005005005005005,
ymax=1.0005005005005,
ytick={  0, 0.2, 0.4, 0.6, 0.8,   1},
ylabel={$\theta_2$},
axis x line*=bottom,
axis y line*=left,
colormap={mymap}{[1pt] rgb(0pt)=(0.0143,0.0143,0.0143); rgb(1pt)=(0.0244084,0.0168792,0.0197143); rgb(2pt)=(0.0343475,0.0194402,0.02529); rgb(3pt)=(0.0440851,0.0219827,0.031023); rgb(4pt)=(0.0535893,0.0245066,0.0369095); rgb(5pt)=(0.062828,0.0270114,0.0429454); rgb(6pt)=(0.071769,0.0294971,0.0491269); rgb(7pt)=(0.0803802,0.0319633,0.05545); rgb(8pt)=(0.0886297,0.0344097,0.0619107); rgb(9pt)=(0.0964852,0.0368363,0.0685051); rgb(10pt)=(0.103915,0.0392426,0.0752293); rgb(11pt)=(0.110886,0.0416284,0.0820793); rgb(12pt)=(0.117368,0.0439936,0.0890511); rgb(13pt)=(0.123327,0.0463378,0.0961409); rgb(14pt)=(0.128732,0.0486608,0.103345); rgb(15pt)=(0.13355,0.0509623,0.110658); rgb(16pt)=(0.13775,0.0532421,0.118078); rgb(17pt)=(0.1413,0.0555,0.1256); rgb(18pt)=(0.144494,0.0577222,0.133297); rgb(19pt)=(0.147635,0.0598991,0.141234); rgb(20pt)=(0.150707,0.0620365,0.149395); rgb(21pt)=(0.15369,0.0641403,0.157763); rgb(22pt)=(0.156566,0.0662164,0.166324); rgb(23pt)=(0.159318,0.0682706,0.175061); rgb(24pt)=(0.161926,0.0703088,0.183957); rgb(25pt)=(0.164372,0.0723369,0.192998); rgb(26pt)=(0.166638,0.0743606,0.202166); rgb(27pt)=(0.168707,0.076386,0.211447); rgb(28pt)=(0.170558,0.0784188,0.220823); rgb(29pt)=(0.172176,0.0804649,0.230279); rgb(30pt)=(0.17354,0.0825301,0.2398); rgb(31pt)=(0.174632,0.0846204,0.249368); rgb(32pt)=(0.175436,0.0867415,0.258968); rgb(33pt)=(0.175931,0.0888995,0.268584); rgb(34pt)=(0.1761,0.0911,0.2782); rgb(35pt)=(0.17608,0.0932893,0.288236); rgb(36pt)=(0.17602,0.0954203,0.299026); rgb(37pt)=(0.175922,0.0975092,0.310428); rgb(38pt)=(0.175786,0.0995723,0.322297); rgb(39pt)=(0.175613,0.101626,0.334492); rgb(40pt)=(0.175404,0.103685,0.346867); rgb(41pt)=(0.17516,0.105768,0.359281); rgb(42pt)=(0.174882,0.107889,0.37159); rgb(43pt)=(0.174571,0.110066,0.383651); rgb(44pt)=(0.174227,0.112313,0.39532); rgb(45pt)=(0.173853,0.114648,0.406454); rgb(46pt)=(0.173448,0.117087,0.41691); rgb(47pt)=(0.173013,0.119645,0.426544); rgb(48pt)=(0.172551,0.122339,0.435213); rgb(49pt)=(0.17206,0.125186,0.442775); rgb(50pt)=(0.171543,0.128201,0.449085); rgb(51pt)=(0.171,0.1314,0.454); rgb(52pt)=(0.170031,0.134957,0.458091); rgb(53pt)=(0.16829,0.139002,0.46202); rgb(54pt)=(0.165863,0.143483,0.465783); rgb(55pt)=(0.162835,0.148348,0.469374); rgb(56pt)=(0.15929,0.153547,0.47279); rgb(57pt)=(0.155314,0.159029,0.476026); rgb(58pt)=(0.150992,0.164743,0.479077); rgb(59pt)=(0.146408,0.170638,0.481938); rgb(60pt)=(0.141647,0.176662,0.484605); rgb(61pt)=(0.136795,0.182765,0.487073); rgb(62pt)=(0.131936,0.188896,0.489338); rgb(63pt)=(0.127156,0.195003,0.491395); rgb(64pt)=(0.122538,0.201036,0.49324); rgb(65pt)=(0.118169,0.206943,0.494867); rgb(66pt)=(0.114133,0.212673,0.496273); rgb(67pt)=(0.110515,0.218176,0.497452); rgb(68pt)=(0.1074,0.2234,0.4984); rgb(69pt)=(0.104662,0.228422,0.49922); rgb(70pt)=(0.102104,0.233358,0.500015); rgb(71pt)=(0.099699,0.238221,0.500783); rgb(72pt)=(0.0974245,0.24302,0.50152); rgb(73pt)=(0.0952557,0.247765,0.502223); rgb(74pt)=(0.0931681,0.252468,0.502888); rgb(75pt)=(0.0911374,0.257138,0.503513); rgb(76pt)=(0.0891393,0.261787,0.504093); rgb(77pt)=(0.0871493,0.266426,0.504626); rgb(78pt)=(0.0851432,0.271063,0.505109); rgb(79pt)=(0.0830965,0.275711,0.505538); rgb(80pt)=(0.0809849,0.280379,0.50591); rgb(81pt)=(0.0787839,0.285079,0.506222); rgb(82pt)=(0.0764694,0.28982,0.50647); rgb(83pt)=(0.0740168,0.294614,0.506651); rgb(84pt)=(0.0714018,0.29947,0.506762); rgb(85pt)=(0.0686,0.3044,0.5068); rgb(86pt)=(0.0653572,0.309404,0.506262); rgb(87pt)=(0.0615042,0.314469,0.504714); rgb(88pt)=(0.0571445,0.319589,0.502254); rgb(89pt)=(0.0523815,0.324755,0.498978); rgb(90pt)=(0.0473184,0.32996,0.494986); rgb(91pt)=(0.0420588,0.335197,0.490374); rgb(92pt)=(0.036706,0.340457,0.48524); rgb(93pt)=(0.0313634,0.345734,0.479682); rgb(94pt)=(0.0261344,0.351019,0.473798); rgb(95pt)=(0.0211225,0.356305,0.467685); rgb(96pt)=(0.0164309,0.361585,0.461442); rgb(97pt)=(0.0121631,0.366851,0.455165); rgb(98pt)=(0.00842253,0.372095,0.448952); rgb(99pt)=(0.00531253,0.377309,0.442902); rgb(100pt)=(0.00293651,0.382486,0.437111); rgb(101pt)=(0.00139787,0.387619,0.431678); rgb(102pt)=(0.0008,0.3927,0.4267); rgb(103pt)=(0.000709623,0.397742,0.42201); rgb(104pt)=(0.000624546,0.402766,0.417367); rgb(105pt)=(0.000544794,0.407771,0.412761); rgb(106pt)=(0.000470387,0.412759,0.408183); rgb(107pt)=(0.00040135,0.417729,0.403621); rgb(108pt)=(0.000337705,0.422684,0.399065); rgb(109pt)=(0.000279475,0.427622,0.394505); rgb(110pt)=(0.000226682,0.432546,0.38993); rgb(111pt)=(0.00017935,0.437456,0.385331); rgb(112pt)=(0.000137501,0.442352,0.380696); rgb(113pt)=(0.000101158,0.447235,0.376016); rgb(114pt)=(7.03435e-05,0.452106,0.37128); rgb(115pt)=(4.50806e-05,0.456965,0.366478); rgb(116pt)=(2.5392e-05,0.461813,0.361599); rgb(117pt)=(1.13005e-05,0.466652,0.356634); rgb(118pt)=(2.82893e-06,0.47148,0.351571); rgb(119pt)=(0,0.4763,0.3464); rgb(120pt)=(0,0.481102,0.341072); rgb(121pt)=(0,0.485879,0.335558); rgb(122pt)=(0,0.490634,0.329884); rgb(123pt)=(0,0.49537,0.324072); rgb(124pt)=(0,0.50009,0.318147); rgb(125pt)=(0,0.504796,0.312132); rgb(126pt)=(0,0.509492,0.306052); rgb(127pt)=(0,0.514179,0.29993); rgb(128pt)=(0,0.518862,0.293791); rgb(129pt)=(0,0.523544,0.287657); rgb(130pt)=(0,0.528226,0.281554); rgb(131pt)=(0,0.532912,0.275504); rgb(132pt)=(0,0.537604,0.269532); rgb(133pt)=(0,0.542307,0.263662); rgb(134pt)=(0,0.547021,0.257917); rgb(135pt)=(0,0.551752,0.252322); rgb(136pt)=(0,0.5565,0.2469); rgb(137pt)=(0,0.561307,0.241711); rgb(138pt)=(0,0.5662,0.236772); rgb(139pt)=(0,0.571161,0.232039); rgb(140pt)=(0,0.576172,0.22747); rgb(141pt)=(0,0.581218,0.223024); rgb(142pt)=(0,0.58628,0.218657); rgb(143pt)=(0,0.591341,0.214328); rgb(144pt)=(0,0.596384,0.209994); rgb(145pt)=(0,0.601392,0.205613); rgb(146pt)=(0,0.606347,0.201143); rgb(147pt)=(0,0.611231,0.196542); rgb(148pt)=(0,0.616029,0.191766); rgb(149pt)=(0,0.620721,0.186774); rgb(150pt)=(0,0.625292,0.181524); rgb(151pt)=(0,0.629724,0.175973); rgb(152pt)=(0,0.633999,0.170079); rgb(153pt)=(0,0.6381,0.1638); rgb(154pt)=(0.00153448,0.642042,0.156421); rgb(155pt)=(0.00594177,0.645859,0.147433); rgb(156pt)=(0.0129276,0.649563,0.137101); rgb(157pt)=(0.0221978,0.653167,0.125691); rgb(158pt)=(0.033458,0.656683,0.113468); rgb(159pt)=(0.046414,0.660125,0.100697); rgb(160pt)=(0.0607716,0.663503,0.0876444); rgb(161pt)=(0.0762365,0.666832,0.0745752); rgb(162pt)=(0.0925145,0.670123,0.0617548); rgb(163pt)=(0.109311,0.673389,0.0494487); rgb(164pt)=(0.126333,0.676643,0.0379222); rgb(165pt)=(0.143285,0.679896,0.0274408); rgb(166pt)=(0.159872,0.683162,0.0182699); rgb(167pt)=(0.175802,0.686453,0.010675); rgb(168pt)=(0.19078,0.689781,0.00492137); rgb(169pt)=(0.20451,0.693159,0.00127458); rgb(170pt)=(0.2167,0.6966,0); rgb(171pt)=(0.227739,0.700166,0); rgb(172pt)=(0.238267,0.703889,0); rgb(173pt)=(0.248364,0.707737,0); rgb(174pt)=(0.258112,0.711676,0); rgb(175pt)=(0.267591,0.715675,0); rgb(176pt)=(0.276883,0.7197,0); rgb(177pt)=(0.286067,0.723718,0); rgb(178pt)=(0.295225,0.727696,0); rgb(179pt)=(0.304438,0.731602,0); rgb(180pt)=(0.313786,0.735402,0); rgb(181pt)=(0.32335,0.739065,0); rgb(182pt)=(0.333211,0.742556,0); rgb(183pt)=(0.34345,0.745844,0); rgb(184pt)=(0.354148,0.748895,0); rgb(185pt)=(0.365385,0.751677,0); rgb(186pt)=(0.377242,0.754156,0); rgb(187pt)=(0.3898,0.7563,0); rgb(188pt)=(0.403574,0.758195,0); rgb(189pt)=(0.418885,0.759957,0); rgb(190pt)=(0.435525,0.761599,0); rgb(191pt)=(0.453286,0.763136,0); rgb(192pt)=(0.471958,0.764579,0); rgb(193pt)=(0.491333,0.765943,0); rgb(194pt)=(0.511203,0.767241,0); rgb(195pt)=(0.531358,0.768485,0); rgb(196pt)=(0.551591,0.76969,0); rgb(197pt)=(0.571692,0.770869,0); rgb(198pt)=(0.591453,0.772035,0); rgb(199pt)=(0.610666,0.773201,0); rgb(200pt)=(0.629122,0.77438,0); rgb(201pt)=(0.646612,0.775587,0); rgb(202pt)=(0.662927,0.776833,0); rgb(203pt)=(0.677859,0.778133,0); rgb(204pt)=(0.6912,0.7795,0); rgb(205pt)=(0.70344,0.780892,0.00343064); rgb(206pt)=(0.715224,0.782261,0.0132589); rgb(207pt)=(0.72658,0.783614,0.0287895); rgb(208pt)=(0.737533,0.784956,0.0493267); rgb(209pt)=(0.748107,0.786293,0.0741753); rgb(210pt)=(0.758328,0.787632,0.10264); rgb(211pt)=(0.768223,0.788977,0.134025); rgb(212pt)=(0.777816,0.790335,0.167635); rgb(213pt)=(0.787133,0.791712,0.202774); rgb(214pt)=(0.7962,0.793114,0.238748); rgb(215pt)=(0.805042,0.794545,0.27486); rgb(216pt)=(0.813684,0.796013,0.310415); rgb(217pt)=(0.822153,0.797523,0.344718); rgb(218pt)=(0.830473,0.799081,0.377074); rgb(219pt)=(0.838671,0.800692,0.406786); rgb(220pt)=(0.846771,0.802363,0.43316); rgb(221pt)=(0.8548,0.8041,0.4555); rgb(222pt)=(0.863049,0.805883,0.475346); rgb(223pt)=(0.871713,0.807693,0.494702); rgb(224pt)=(0.880675,0.809539,0.513574); rgb(225pt)=(0.889816,0.811426,0.531964); rgb(226pt)=(0.899019,0.813361,0.549875); rgb(227pt)=(0.908164,0.815352,0.567311); rgb(228pt)=(0.917134,0.817406,0.584275); rgb(229pt)=(0.925811,0.819529,0.600771); rgb(230pt)=(0.934077,0.821728,0.616801); rgb(231pt)=(0.941812,0.824011,0.63237); rgb(232pt)=(0.9489,0.826383,0.64748); rgb(233pt)=(0.955221,0.828853,0.662135); rgb(234pt)=(0.960658,0.831427,0.676338); rgb(235pt)=(0.965093,0.834111,0.690093); rgb(236pt)=(0.968407,0.836914,0.703402); rgb(237pt)=(0.970482,0.839841,0.71627); rgb(238pt)=(0.9712,0.8429,0.7287); rgb(239pt)=(0.9712,0.846163,0.740814); rgb(240pt)=(0.971197,0.84969,0.752722); rgb(241pt)=(0.971189,0.853471,0.764409); rgb(242pt)=(0.971174,0.857495,0.775859); rgb(243pt)=(0.971149,0.861753,0.787056); rgb(244pt)=(0.971112,0.866234,0.797986); rgb(245pt)=(0.97106,0.870928,0.808633); rgb(246pt)=(0.970992,0.875824,0.81898); rgb(247pt)=(0.970903,0.880913,0.829014); rgb(248pt)=(0.970793,0.886183,0.838717); rgb(249pt)=(0.970658,0.891625,0.848075); rgb(250pt)=(0.970497,0.897228,0.857072); rgb(251pt)=(0.970306,0.902981,0.865692); rgb(252pt)=(0.970083,0.908876,0.873921); rgb(253pt)=(0.969826,0.914901,0.881742); rgb(254pt)=(0.969533,0.921046,0.88914); rgb(255pt)=(0.9692,0.9273,0.8961)},
colorbar,
colorbar style={separate axis lines},
point meta min=0,
point meta max=2.24999774549098
]
\addplot [forget plot] graphics [xmin=-0.0005005005005005,xmax=1.0005005005005,ymin=-0.0005005005005005,ymax=1.0005005005005] {problem_1_joint_posterior-1.png};
\end{axis}
\end{tikzpicture}%
  \caption{The joint posterior $p(\theta_1, \theta_2 \given \text{H})$.}
  \label{problem_joint_posterior}
\end{figure}

\clearpage
\begin{enumerate}
\setcounter{enumi}{1}
\item
  Consider the three-dimensional parameter vector $\vec{\theta} =
  [\theta_1, \theta_2, \theta_3]\trans$, with the following joint
  multivariate Gaussian prior:
  \begin{equation*}
    p(\vec{\theta})
    =
    \mc{N}(\vec{\theta}; \vec{\mu}, \mat{\Sigma})
    =
    \mc{N}
    \left(
    \begin{bmatrix}
      \theta_1 \\
      \theta_2 \\
      \theta_3
    \end{bmatrix}
    ;
    \begin{bmatrix}
      0 \\
      1 \\
      2
    \end{bmatrix},
    \begin{bmatrix}
      1 & 2 & 0   \\
      2 & 9 & 0 \\
      0 & 0 & 16
    \end{bmatrix}
    \right).
  \end{equation*}
  We are going to consider a decision problem with action space
  $\mc{A} = \{1, 2, 3\}$.  The result of choosing an action $a \in
  \mc{A}$ will be to observe the exact value of $\theta_a$, the $a$th
  element of $\vec{\theta}$.

  Consider the following loss functions, $\ell_1$ and $\ell_2$:
  \begin{equation*}
    \ell_1(\vec{\theta}, a) =
    \begin{cases}
      1 & \theta_a   >  0 \\
      0 & \theta_a \leq 0
    \end{cases}
    \qquad
    \ell_2(\vec{\theta}, a) = \min(0, \theta_a).
  \end{equation*}
  For each:
  \begin{itemize}
  \item
    Write a generic expression for the expected loss of action $a$ in
    terms of $\vec{\mu}$ and $\mat{\Sigma}$.  Evaluate any integrals
    you encounter.
  \item
    Give a numerical value for the expected loss of each action, using
    the values of $(\vec{\mu}, \mat{\Sigma})$ provided above.
  \item
    State the Bayes action.
  \end{itemize}
\end{enumerate}

\subsection*{Solution}

First, we note that the loss functions only depend on $\vec{\theta}$
through $\theta_a$, so we must only consider the marginal belief about
$\theta_a$ when contemplating action $a$.  By applying the marginalization
formula for multivariate Gaussians, this belief is:
\begin{equation*}
  p(\theta_a) = \mc{N}(\theta_a; \mu_a, \Sigma_{aa}).
\end{equation*}
For loss $\ell_1$, we may calculate the expected loss of each action:
\begin{equation*}
  \mathbb{E}
  \bigl[
    \ell_1(\vec{\theta}, a)
  \bigr]
  =
  \int_{-\infty}^\infty
  \ell_1(\vec{\theta}, a)
  p(\theta_a)
  \intd{\theta_a}
  =
  \int_0^\infty
  \mc{N}(\theta_a; \mu_a, \Sigma_{aa})
  \intd{\theta_a}
  =
  1 - \Phi(0; \mu_a, \Sigma_{aa}).
\end{equation*}
Using this result, we may numerically calculate the expected loss for
each action:
\begin{equation*}
  \mathbb{E}\bigl[\ell_1(\vec{\theta}, 1)\bigr]
  =
  0.5 \qquad
  \mathbb{E}\bigl[\ell_1(\vec{\theta}, 2)\bigr]
  =
  0.631 \qquad
  \mathbb{E}\bigl[\ell_1(\vec{\theta}, 3)\bigr]
  =
  0.691.
\end{equation*}
The Bayes action is $a = 1$, with the lowest expected loss.

For loss $\ell_2$, we proceed in the same way:
\begin{equation*}
  \mathbb{E}
  \bigl[
    \ell_2(\vec{\theta}, a)
  \bigr]
  =
  \int_{-\infty}^\infty
  \ell_2(\vec{\theta}, a)
  p(\theta_a)
  \intd{\theta_a}
  =
  \int_{-\infty}^0
  \theta_a
  \mc{N}(\theta_a; \mu_a, \Sigma_{aa})
  \intd{\theta_a}.
\end{equation*}
We may compute this definite integral; I used a table of Gaussian
integrals\footnote{\url{http://en.wikipedia.org/wiki/List_of_integrals_of_Gaussian_functions}}
and the identity
\begin{equation*}
  \phi\biggl(
  \frac{a - \mu}{\sigma}
  \biggr)
  =
  \sigma \mc{N}(a; \mu, \sigma^2)
\end{equation*}
to derive
\begin{equation*}
  \mathbb{E}
  \bigl[
    \ell_2(\vec{\theta}, a)
  \bigr]
  =
  \mu_a \Phi(0; \mu_a, \Sigma_{aa})
  -
  \Sigma_{aa}
  \mc{N}(0; \mu_a, \Sigma_{aa}).
\end{equation*}
Using this result, we may numerically calculate the expected loss for
each action:
\begin{equation*}
  \mathbb{E}\bigl[\ell_2(\vec{\theta}, 1)\bigr]
  =
  -0.399 \qquad
  \mathbb{E}\bigl[\ell_2(\vec{\theta}, 2)\bigr]
  =
  -0.763 \qquad
  \mathbb{E}\bigl[\ell_2(\vec{\theta}, 3)\bigr]
  =
  -0.791.
\end{equation*}
The Bayes action is now $a = 3$, with the lowest expected loss.

\clearpage
\begin{enumerate}
\setcounter{enumi}{2}
\item
  Consider a $d$-dimensional vector $\vec{\theta}$ with an arbitrary
  multivariate Gaussian distribution:
  \begin{equation*}
    p(\vec{\theta})
    =
    \mc{N}(\vec{\theta}; \vec{\mu}, \mat{\Sigma}).
  \end{equation*}
  \begin{itemize}
  \item
    Give a general expression for the distribution of the following
    (scalar) value $\tau$.
    \begin{equation*}
      \tau = \theta_1 + 2\theta_2 + \dotsb d\theta_d
    \end{equation*}
  \item
    Consider again the specific distribution of the three-dimensional
    vector $\vec{\theta}$ from the last problem, as well as the action
    space $\mc{A}$ with the same observation mechanism: after choosing
    $a \in \mc{A}$, we will observe the corresponding value
    $\theta_a$.  Suppose we may select one action and then must
    predict $\tau$ under a squared loss function:
    \begin{equation*}
      \ell(\tau, \hat{\tau}) = (\tau - \hat{\tau})^2.
    \end{equation*}
    Using the distribution from the last problem. what is the expected
    loss of each of the three available actions?  Which is the Bayes
    action?
  \end{itemize}
\end{enumerate}

\subsection*{Solution}

Define the (row) vector $\vec{d}\trans = [1, 2, \dotsc, d]$.  We first
notice that $\tau$ is simply a linear transformation of
$\vec{\theta}$:
\begin{equation*}
  \tau = \vec{d}\trans \vec{\theta};
\end{equation*}
therefore $\tau$ has a multivariate Gaussian distribution:
\begin{equation*}
  p(\tau)
  =
  p(\vec{d}\trans \vec{\theta})
  =
  \mc{N}(\tau; \vec{d}\trans \vec{\mu}, \vec{d}\trans \mat{\Sigma} \vec{d}).
\end{equation*}

In the second part of the question, we must consider estimating $\tau$
under a squared loss function $\ell(\tau, \hat{\tau}) = (\tau -
\hat{\tau})^2$.  A general result from Bayesian decision theory is
that the Bayes action is to estimate $\hat{\tau}$ as the (posterior)
mean of $\tau$.  For example, given the initial belief from the last
problem, we would predict $\hat{\tau} = \vec{d}\trans \vec{\mu} = 8$.
What is the \emph{expected} loss when predicting the mean $\hat{\tau}
= \mathbb{E}[\tau]$?
\begin{equation*}
  \mathbb{E}
  \Bigl[
    \ell\bigl(\tau, \mathbb{E}[\tau]\bigr)
  \Bigr]
  =
  \mathbb{E}
  \Bigl[
    \bigl(\tau - \mathbb{E}[\tau]\bigr)^2
  \Bigr]
  =
  \var [ \tau ]
  =
  \vec{d}\trans \mat{\Sigma} \vec{d}.
\end{equation*}
The expected squared loss is simply the variance of $\tau$!  Conveniently,
we have a closed-form expression for this variance.

The problem asks us to consider how we would proceed with estimating
$\tau$ if we could observe one of the entries of the vector
$\vec{\theta}$ before making our prediction $\hat{\tau}$.  If we wish
to minimize our expected loss, we should minimize the variance of
$\tau$ with our observation.  Observing an entry of $\vec{\theta}$ is
a conditioning observation of a multivariate Gaussian.  We have a
closed-form expression for the posterior covariance of $\vec{\theta}$
after observing any entry $\theta_a$.  Remarkably, the posterior
covariance of $\vec{\theta}$ does not depend on the actual value we
observe, only the index of the entry we choose, $a$.  The posterior
covariance matrices $\mat{\Sigma}_{\vec{\theta} \given \theta_a}$ for
each available action $a \in \mc{A}$ are:
\begin{align*}
  \mat{\Sigma}_{\vec{\theta} \given \theta_1}
  &=
  \begin{bmatrix}
    0 & 0 & 0   \\
    0 & 5 & 0   \\
    0 & 0 & 16
  \end{bmatrix}
  \\
  \mat{\Sigma}_{\vec{\theta} \given \theta_2}
  &=
  \begin{bmatrix}
    \frac{5}{9} & 0 & 0   \\
    0           & 0 & 0   \\
    0           & 0 & 16
  \end{bmatrix}
  \\
  \mat{\Sigma}_{\vec{\theta} \given \theta_3}
  &=
  \begin{bmatrix}
    1 & 2 & 0 \\
    2 & 9 & 0 \\
    0 & 0 & 0
  \end{bmatrix}.
\end{align*}
The expected losses of our final prediction of $\hat{\tau}$
given $\theta_a$ is now given by
\begin{equation*}
  \mathbb{E}\bigl[\ell(\tau, \hat{\tau}) \given \theta_a \bigr]
  =
  \vec{d}\trans \mat{\Sigma}_{\vec{\theta} \given \theta_a} \vec{d}.
\end{equation*}
For our particular problem, the expected final loss after each potential
action is:
\begin{equation*}
  \mathbb{E}\bigl[\ell(\tau, \hat{\tau}) \given \theta_1\bigr]
  =
  164 \qquad
  \mathbb{E}\bigl[\ell(\tau, \hat{\tau}) \given \theta_2\bigr]
  =
  144\,\nicefrac{5}{9} \qquad
  \mathbb{E}\bigl[\ell(\tau, \hat{\tau}) \given \theta_3\bigr]
  =
  45.
\end{equation*}
The Bayes action is $a = 3$.  Despite the fact that $\theta_3$ is
uncorrelated with the other two entries, collapsing its large variance
from $16$ to zero has the effect of reducing the variance of $\tau$
(and therefore our expected loss) by $3^2 \cdot 16 = 144$.

\clearpage
\begin{enumerate}
\setcounter{enumi}{3}
\item
  Consider the following data:
  \begin{align*}
    \vec{x} &= [0.54, 1.84, -2.26, 0.86, 0.32]\trans; \\
    \vec{y} &= [-1.31, -0.43, 0.34, 3.58, 2.77]\trans.
  \end{align*}
  Consider the Bayesian linear regression model with $\phi(x) = [1,
    x]\trans$.  Use the prior $p(\vec{w}) = \mc{N}(\vec{w}; \vec{0},
  \mat{I})$.

  Plot the posterior probability that the slope of the regression line
  is positive as a function of the standard deviation of the
  observation noise $\sigma$ (the noise variance is then $\sigma^2$).
  Use a grid of at least 100 points in the range $\sigma \in (0.01,
  10)$.
\end{enumerate}

\subsection*{Solution}

Using the given linear regression model, we assume
\begin{equation*}
  y
  =
  \phi(x)\trans \vec{w} + \varepsilon
  =
  w_1 + w_2 x + \varepsilon.
\end{equation*}
The second entry of the weight vector $\vec{w}$, $w_2$, therefore
serves as the slope of the regression line.

The Bayesian linear regression model gives the following posterior for
$\vec{w}$ given observations $\data$ and a specified noise variance
$\sigma^2$:
\begin{equation*}
  p(\vec{w} \given \data, \sigma^2)
  =
  \mc{N}(\vec{w};
  \vec{\mu}_{\vec{w}\given\data},
  \mat{\Sigma}_{\vec{w}\given\data}
  ),
\end{equation*}
where
\begin{align*}
  \vec{\mu}_{\vec{w}\given\data}
  &=
  \mat{X}\trans
  (\mat{X}\mat{X}\trans + \sigma^2 \mat{I})\inv
  \vec{y};
  \\
  \mat{\Sigma}_{\vec{w}\given\data}
  &=
  \mat{I}
  -
  \mat{X}\trans
  (\mat{X}\mat{X}\trans + \sigma^2 \mat{I})\inv
  \mat{X},
\end{align*}
where we have plugged the given prior $p(\vec{w}) = \mc{N}(\vec{w};
\vec{0}, \mat{I})$ into the general result.

Given a value of $\sigma$, the formulas above give the posterior over
$\vec{w}$.  To determine the probability that $w_2$ is positive, we
simply take the marginal posterior distribution and evaluate the
normal \acro{CDF}:
\begin{equation*}
  \Pr(w_2 > 0 \given \data, \sigma^2)
  =
  1 -
  \Phi\bigl(
  0;
  (\vec{\mu}_{\vec{w}\given\data})_2
  ,
  (\mat{\Sigma}_{\vec{w}\given\data})_{22}
  \bigr).
\end{equation*}

This quantity is plotted as a function of $\sigma$ in Figure
\ref{problem_4}.  The larger the noise, the less confident we become
about the sign of the slope.

\begin{figure}
  \centering
  % This file was created by matlab2tikz.
% Minimal pgfplots version: 1.3
%
\tikzsetnextfilename{problem_4}
\definecolor{mycolor1}{rgb}{0.12157,0.47059,0.70588}%
%
\begin{tikzpicture}

\begin{axis}[%
width=0.95092\figurewidth,
height=\figureheight,
at={(0\figurewidth,0\figureheight)},
scale only axis,
xmin=0,
xmax=10,
xlabel={$\sigma$},
ymin=0.5,
ymax=1.05,
ylabel={$\Pr(w_2 > 0 \given \data, \sigma^2)$},
axis x line*=bottom,
axis y line*=left
]
\addplot [color=mycolor1,solid,forget plot]
  table[row sep=crcr]{%
0.01	1\\
0.02	0.999999999992479\\
0.03	0.999996595966551\\
0.04	0.999632318134581\\
0.05	0.996556025677947\\
0.06	0.987878259818996\\
0.07	0.973368167687833\\
0.08	0.954735487167244\\
0.09	0.933964947476636\\
0.1	0.912583952534793\\
0.11	0.891574753363322\\
0.12	0.871496490590082\\
0.13	0.852626389646931\\
0.14	0.835068176630924\\
0.15	0.818824723732505\\
0.16	0.80384371699183\\
0.17	0.790045318348458\\
0.18	0.777338451294903\\
0.19	0.765630100748129\\
0.2	0.754830397766391\\
0.21	0.744855196659661\\
0.22	0.735627183930615\\
0.23	0.727076147021121\\
0.24	0.719138779835368\\
0.25	0.711758249529762\\
0.26	0.704883656624097\\
0.27	0.698469464559199\\
0.28	0.692474941101449\\
0.29	0.686863633782152\\
0.3	0.68160288958312\\
0.31	0.676663422118278\\
0.32	0.672018925643197\\
0.33	0.667645733148085\\
0.34	0.663522514809105\\
0.35	0.659630012737433\\
0.36	0.65595080799485\\
0.37	0.652469116070091\\
0.38	0.64917060733047\\
0.39	0.646042249317254\\
0.4	0.643072168107797\\
0.41	0.640249526302906\\
0.42	0.637564415504967\\
0.43	0.635007761429093\\
0.44	0.632571240033259\\
0.45	0.630247203268365\\
0.46	0.628028613235669\\
0.47	0.625908983700946\\
0.48	0.623882328055273\\
0.49	0.621943112932756\\
0.5	0.620086216800299\\
0.51	0.618306892924071\\
0.52	0.616600736194957\\
0.53	0.61496365336201\\
0.54	0.613391836280597\\
0.55	0.611881737831659\\
0.56	0.610430050211411\\
0.57	0.609033685328275\\
0.58	0.607689757075817\\
0.59	0.606395565278646\\
0.6	0.605148581132367\\
0.61	0.603946433980035\\
0.62	0.602786899285812\\
0.63	0.601667887682682\\
0.64	0.600587434985228\\
0.65	0.599543693070632\\
0.66	0.598534921541953\\
0.67	0.597559480097205\\
0.68	0.596615821536055\\
0.69	0.595702485343312\\
0.7	0.594818091794892\\
0.71	0.59396133653757\\
0.72	0.593130985598999\\
0.73	0.592325870788863\\
0.74	0.591544885456041\\
0.75	0.590786980570195\\
0.76	0.590051161099329\\
0.77	0.589336482657622\\
0.78	0.588642048400429\\
0.79	0.587967006145426\\
0.8	0.587310545701021\\
0.81	0.586671896384804\\
0.82	0.586050324716486\\
0.83	0.585445132271193\\
0.84	0.584855653680229\\
0.85	0.584281254767648\\
0.86	0.583721330811953\\
0.87	0.583175304923218\\
0.88	0.582642626526779\\
0.89	0.582122769945367\\
0.9	0.581615233072313\\
0.91	0.581119536129002\\
0.92	0.580635220500419\\
0.93	0.580161847643043\\
0.94	0.579698998059897\\
0.95	0.579246270337938\\
0.96	0.578803280243383\\
0.97	0.578369659870906\\
0.98	0.577945056842964\\
0.99	0.577529133555825\\
1	0.577121566469098\\
1.01	0.576722045435849\\
1.02	0.576330273070582\\
1.03	0.575945964152599\\
1.04	0.575568845062403\\
1.05	0.575198653249009\\
1.06	0.574835136726189\\
1.07	0.574478053595776\\
1.08	0.574127171596369\\
1.09	0.573782267675797\\
1.1	0.573443127585926\\
1.11	0.573109545498387\\
1.12	0.572781323640002\\
1.13	0.572458271946675\\
1.14	0.572140207734693\\
1.15	0.571826955388362\\
1.16	0.571518346063062\\
1.17	0.571214217402792\\
1.18	0.570914413271393\\
1.19	0.570618783496663\\
1.2	0.570327183626627\\
1.21	0.570039474697299\\
1.22	0.569755523011273\\
1.23	0.569475199926574\\
1.24	0.569198381655192\\
1.25	0.568924949070776\\
1.26	0.568654787525009\\
1.27	0.568387786672189\\
1.28	0.568123840301593\\
1.29	0.567862846177215\\
1.3	0.567604705884497\\
1.31	0.56734932468369\\
1.32	0.567096611369519\\
1.33	0.566846478136817\\
1.34	0.566598840451852\\
1.35	0.566353616929042\\
1.36	0.566110729212808\\
1.37	0.565870101864315\\
1.38	0.565631662252851\\
1.39	0.565395340451642\\
1.4	0.565161069137879\\
1.41	0.564928783496756\\
1.42	0.564698421129346\\
1.43	0.564469921964122\\
1.44	0.564243228171969\\
1.45	0.56401828408452\\
1.46	0.563795036115661\\
1.47	0.563573432686076\\
1.48	0.563353424150694\\
1.49	0.563134962728894\\
1.5	0.562918002437374\\
1.51	0.562702499025545\\
1.52	0.562488409913356\\
1.53	0.562275694131433\\
1.54	0.562064312263457\\
1.55	0.561854226390653\\
1.56	0.561645400038337\\
1.57	0.561437798124408\\
1.58	0.561231386909722\\
1.59	0.561026133950265\\
1.6	0.560822008051053\\
1.61	0.560618979221686\\
1.62	0.560417018633497\\
1.63	0.560216098578229\\
1.64	0.560016192428176\\
1.65	0.559817274597741\\
1.66	0.55961932050634\\
1.67	0.559422306542622\\
1.68	0.559226210029928\\
1.69	0.55903100919297\\
1.7	0.558836683125661\\
1.71	0.558643211760061\\
1.72	0.558450575836403\\
1.73	0.558258756874152\\
1.74	0.558067737144055\\
1.75	0.557877499641162\\
1.76	0.557688028058759\\
1.77	0.557499306763203\\
1.78	0.55731132076961\\
1.79	0.557124055718372\\
1.8	0.556937497852475\\
1.81	0.556751633995586\\
1.82	0.556566451530882\\
1.83	0.556381938380596\\
1.84	0.556198082986257\\
1.85	0.556014874289595\\
1.86	0.55583230171409\\
1.87	0.555650355147147\\
1.88	0.555469024922868\\
1.89	0.555288301805403\\
1.9	0.555108176972869\\
1.91	0.554928642001797\\
1.92	0.554749688852118\\
1.93	0.55457130985264\\
1.94	0.554393497687021\\
1.95	0.554216245380214\\
1.96	0.554039546285362\\
1.97	0.553863394071142\\
1.98	0.553687782709528\\
1.99	0.553512706463973\\
2	0.553338159877984\\
2.01	0.553164137764081\\
2.02	0.55299063519314\\
2.03	0.55281764748408\\
2.04	0.552645170193913\\
2.05	0.552473199108126\\
2.06	0.552301730231381\\
2.07	0.552130759778549\\
2.08	0.551960284166034\\
2.09	0.551790300003396\\
2.1	0.551620804085267\\
2.11	0.551451793383532\\
2.12	0.551283265039792\\
2.13	0.551115216358071\\
2.14	0.55094764479779\\
2.15	0.550780547966965\\
2.16	0.550613923615661\\
2.17	0.550447769629657\\
2.18	0.550282084024337\\
2.19	0.550116864938798\\
2.2	0.549952110630157\\
2.21	0.549787819468065\\
2.22	0.549623989929406\\
2.23	0.549460620593194\\
2.24	0.549297710135641\\
2.25	0.549135257325404\\
2.26	0.548973261019006\\
2.27	0.548811720156414\\
2.28	0.548650633756784\\
2.29	0.548490000914351\\
2.3	0.548329820794474\\
2.31	0.548170092629829\\
2.32	0.54801081571673\\
2.33	0.547851989411596\\
2.34	0.547693613127545\\
2.35	0.547535686331113\\
2.36	0.547378208539098\\
2.37	0.547221179315521\\
2.38	0.547064598268706\\
2.39	0.546908465048462\\
2.4	0.546752779343382\\
2.41	0.546597540878244\\
2.42	0.546442749411507\\
2.43	0.546288404732911\\
2.44	0.546134506661167\\
2.45	0.545981055041742\\
2.46	0.545828049744729\\
2.47	0.545675490662803\\
2.48	0.545523377709261\\
2.49	0.54537171081614\\
2.5	0.545220489932412\\
2.51	0.545069715022257\\
2.52	0.544919386063405\\
2.53	0.544769503045548\\
2.54	0.54462006596882\\
2.55	0.544471074842343\\
2.56	0.544322529682835\\
2.57	0.54417443051328\\
2.58	0.544026777361657\\
2.59	0.543879570259722\\
2.6	0.543732809241851\\
2.61	0.543586494343933\\
2.62	0.543440625602314\\
2.63	0.543295203052792\\
2.64	0.543150226729657\\
2.65	0.543005696664783\\
2.66	0.542861612886757\\
2.67	0.542717975420059\\
2.68	0.542574784284275\\
2.69	0.542432039493359\\
2.7	0.542289741054929\\
2.71	0.542147888969598\\
2.72	0.542006483230347\\
2.73	0.541865523821928\\
2.74	0.541725010720303\\
2.75	0.541584943892115\\
2.76	0.541445323294188\\
2.77	0.541306148873063\\
2.78	0.541167420564551\\
2.79	0.541029138293331\\
2.8	0.540891301972559\\
2.81	0.540753911503511\\
2.82	0.540616966775249\\
2.83	0.540480467664313\\
2.84	0.540344414034434\\
2.85	0.540208805736264\\
2.86	0.540073642607141\\
2.87	0.539938924470863\\
2.88	0.539804651137481\\
2.89	0.539670822403121\\
2.9	0.539537438049816\\
2.91	0.539404497845355\\
2.92	0.539272001543158\\
2.93	0.539139948882152\\
2.94	0.539008339586678\\
2.95	0.538877173366404\\
2.96	0.538746449916252\\
2.97	0.538616168916343\\
2.98	0.538486330031952\\
2.99	0.538356932913477\\
3	0.538227977196419\\
3.01	0.538099462501371\\
3.02	0.537971388434027\\
3.03	0.537843754585188\\
3.04	0.537716560530789\\
3.05	0.537589805831933\\
3.06	0.537463490034927\\
3.07	0.53733761267134\\
3.08	0.537212173258054\\
3.09	0.537087171297339\\
3.1	0.536962606276918\\
3.11	0.536838477670056\\
3.12	0.536714784935642\\
3.13	0.536591527518288\\
3.14	0.536468704848426\\
3.15	0.536346316342416\\
3.16	0.536224361402658\\
3.17	0.536102839417708\\
3.18	0.535981749762399\\
3.19	0.535861091797969\\
3.2	0.53574086487219\\
3.21	0.535621068319505\\
3.22	0.535501701461165\\
3.23	0.535382763605369\\
3.24	0.535264254047414\\
3.25	0.535146172069841\\
3.26	0.535028516942585\\
3.27	0.534911287923132\\
3.28	0.534794484256676\\
3.29	0.534678105176274\\
3.3	0.534562149903011\\
3.31	0.534446617646162\\
3.32	0.534331507603356\\
3.33	0.534216818960743\\
3.34	0.534102550893163\\
3.35	0.533988702564312\\
3.36	0.533875273126917\\
3.37	0.533762261722903\\
3.38	0.533649667483567\\
3.39	0.533537489529753\\
3.4	0.533425726972022\\
3.41	0.533314378910829\\
3.42	0.533203444436694\\
3.43	0.533092922630382\\
3.44	0.532982812563073\\
3.45	0.532873113296541\\
3.46	0.532763823883325\\
3.47	0.532654943366907\\
3.48	0.532546470781887\\
3.49	0.532438405154155\\
3.5	0.532330745501065\\
3.51	0.532223490831612\\
3.52	0.532116640146603\\
3.53	0.532010192438827\\
3.54	0.531904146693232\\
3.55	0.531798501887092\\
3.56	0.53169325699018\\
3.57	0.531588410964935\\
3.58	0.531483962766634\\
3.59	0.531379911343555\\
3.6	0.53127625563715\\
3.61	0.531172994582204\\
3.62	0.531070127107004\\
3.63	0.530967652133503\\
3.64	0.530865568577477\\
3.65	0.530763875348693\\
3.66	0.530662571351065\\
3.67	0.530561655482812\\
3.68	0.530461126636618\\
3.69	0.530360983699785\\
3.7	0.530261225554392\\
3.71	0.530161851077444\\
3.72	0.530062859141024\\
3.73	0.529964248612449\\
3.74	0.529866018354411\\
3.75	0.529768167225131\\
3.76	0.529670694078502\\
3.77	0.529573597764234\\
3.78	0.529476877127997\\
3.79	0.529380531011565\\
3.8	0.529284558252951\\
3.81	0.529188957686551\\
3.82	0.529093728143277\\
3.83	0.528998868450694\\
3.84	0.528904377433155\\
3.85	0.528810253911929\\
3.86	0.528716496705337\\
3.87	0.528623104628877\\
3.88	0.528530076495355\\
3.89	0.528437411115006\\
3.9	0.528345107295625\\
3.91	0.528253163842683\\
3.92	0.528161579559452\\
3.93	0.528070353247125\\
3.94	0.527979483704931\\
3.95	0.527888969730254\\
3.96	0.527798810118745\\
3.97	0.527709003664441\\
3.98	0.527619549159869\\
3.99	0.52753044539616\\
4	0.52744169116316\\
4.01	0.527353285249533\\
4.02	0.527265226442865\\
4.03	0.527177513529775\\
4.04	0.52709014529601\\
4.05	0.527003120526548\\
4.06	0.5269164380057\\
4.07	0.526830096517205\\
4.08	0.526744094844326\\
4.09	0.526658431769947\\
4.1	0.526573106076663\\
4.11	0.526488116546875\\
4.12	0.52640346196288\\
4.13	0.526319141106958\\
4.14	0.526235152761459\\
4.15	0.526151495708892\\
4.16	0.526068168732008\\
4.17	0.525985170613882\\
4.18	0.525902500137997\\
4.19	0.525820156088322\\
4.2	0.525738137249393\\
4.21	0.52565644240639\\
4.22	0.525575070345214\\
4.23	0.525494019852558\\
4.24	0.525413289715987\\
4.25	0.525332878724006\\
4.26	0.52525278566613\\
4.27	0.525173009332959\\
4.28	0.525093548516242\\
4.29	0.525014402008943\\
4.3	0.524935568605313\\
4.31	0.524857047100948\\
4.32	0.524778836292856\\
4.33	0.52470093497952\\
4.34	0.524623341960957\\
4.35	0.524546056038777\\
4.36	0.524469076016244\\
4.37	0.524392400698335\\
4.38	0.52431602889179\\
4.39	0.524239959405175\\
4.4	0.52416419104893\\
4.41	0.524088722635426\\
4.42	0.524013552979014\\
4.43	0.523938680896078\\
4.44	0.523864105205086\\
4.45	0.523789824726632\\
4.46	0.523715838283493\\
4.47	0.523642144700668\\
4.48	0.523568742805429\\
4.49	0.523495631427361\\
4.5	0.52342280939841\\
4.51	0.523350275552923\\
4.52	0.52327802872769\\
4.53	0.523206067761985\\
4.54	0.523134391497605\\
4.55	0.523062998778913\\
4.56	0.522991888452868\\
4.57	0.52292105936907\\
4.58	0.522850510379793\\
4.59	0.522780240340019\\
4.6	0.522710248107473\\
4.61	0.522640532542659\\
4.62	0.522571092508889\\
4.63	0.52250192687232\\
4.64	0.522433034501977\\
4.65	0.522364414269793\\
4.66	0.522296065050631\\
4.67	0.522227985722315\\
4.68	0.52216017516566\\
4.69	0.522092632264497\\
4.7	0.522025355905697\\
4.71	0.521958344979204\\
4.72	0.521891598378053\\
4.73	0.521825114998395\\
4.74	0.521758893739525\\
4.75	0.521692933503901\\
4.76	0.521627233197167\\
4.77	0.521561791728173\\
4.78	0.521496608008999\\
4.79	0.521431680954974\\
4.8	0.521367009484693\\
4.81	0.521302592520039\\
4.82	0.5212384289862\\
4.83	0.521174517811686\\
4.84	0.521110857928347\\
4.85	0.52104744827139\\
4.86	0.520984287779391\\
4.87	0.520921375394317\\
4.88	0.520858710061533\\
4.89	0.520796290729824\\
4.9	0.5207341163514\\
4.91	0.520672185881917\\
4.92	0.520610498280486\\
4.93	0.520549052509682\\
4.94	0.520487847535561\\
4.95	0.520426882327669\\
4.96	0.520366155859051\\
4.97	0.520305667106261\\
4.98	0.520245415049374\\
4.99	0.520185398671993\\
5	0.520125616961257\\
5.01	0.520066068907853\\
5.02	0.520006753506019\\
5.03	0.519947669753552\\
5.04	0.519888816651817\\
5.05	0.519830193205754\\
5.06	0.519771798423879\\
5.07	0.519713631318296\\
5.08	0.519655690904695\\
5.09	0.519597976202365\\
5.1	0.519540486234192\\
5.11	0.519483220026664\\
5.12	0.519426176609878\\
5.13	0.519369355017538\\
5.14	0.519312754286964\\
5.15	0.519256373459089\\
5.16	0.519200211578464\\
5.17	0.51914426769326\\
5.18	0.519088540855269\\
5.19	0.519033030119903\\
5.2	0.518977734546199\\
5.21	0.518922653196817\\
5.22	0.51886778513804\\
5.23	0.518813129439775\\
5.24	0.518758685175554\\
5.25	0.518704451422528\\
5.26	0.518650427261472\\
5.27	0.518596611776782\\
5.28	0.518543004056472\\
5.29	0.518489603192173\\
5.3	0.518436408279131\\
5.31	0.518383418416206\\
5.32	0.518330632705868\\
5.33	0.518278050254194\\
5.34	0.518225670170865\\
5.35	0.518173491569166\\
5.36	0.518121513565977\\
5.37	0.518069735281773\\
5.38	0.518018155840619\\
5.39	0.517966774370168\\
5.4	0.517915590001651\\
5.41	0.517864601869879\\
5.42	0.517813809113233\\
5.43	0.517763210873664\\
5.44	0.517712806296682\\
5.45	0.517662594531355\\
5.46	0.517612574730304\\
5.47	0.517562746049692\\
5.48	0.517513107649223\\
5.49	0.517463658692135\\
5.5	0.517414398345193\\
5.51	0.517365325778681\\
5.52	0.517316440166399\\
5.53	0.517267740685654\\
5.54	0.517219226517251\\
5.55	0.517170896845493\\
5.56	0.517122750858164\\
5.57	0.51707478774653\\
5.58	0.517027006705328\\
5.59	0.516979406932758\\
5.6	0.516931987630475\\
5.61	0.516884748003584\\
5.62	0.516837687260627\\
5.63	0.516790804613582\\
5.64	0.516744099277847\\
5.65	0.516697570472237\\
5.66	0.516651217418972\\
5.67	0.516605039343673\\
5.68	0.516559035475346\\
5.69	0.516513205046382\\
5.7	0.51646754729254\\
5.71	0.516422061452944\\
5.72	0.516376746770071\\
5.73	0.516331602489741\\
5.74	0.516286627861109\\
5.75	0.516241822136658\\
5.76	0.516197184572184\\
5.77	0.51615271442679\\
5.78	0.516108410962878\\
5.79	0.516064273446135\\
5.8	0.516020301145526\\
5.81	0.515976493333283\\
5.82	0.515932849284896\\
5.83	0.515889368279102\\
5.84	0.515846049597876\\
5.85	0.51580289252642\\
5.86	0.515759896353153\\
5.87	0.5157170603697\\
5.88	0.515674383870884\\
5.89	0.515631866154712\\
5.9	0.515589506522368\\
5.91	0.515547304278201\\
5.92	0.515505258729715\\
5.93	0.515463369187557\\
5.94	0.515421634965508\\
5.95	0.515380055380472\\
5.96	0.515338629752463\\
5.97	0.515297357404599\\
5.98	0.515256237663087\\
5.99	0.515215269857213\\
6	0.515174453319333\\
6.01	0.515133787384861\\
6.02	0.515093271392256\\
6.03	0.515052904683015\\
6.04	0.515012686601658\\
6.05	0.514972616495721\\
6.06	0.514932693715741\\
6.07	0.514892917615248\\
6.08	0.514853287550752\\
6.09	0.514813802881734\\
6.1	0.514774462970631\\
6.11	0.514735267182829\\
6.12	0.51469621488665\\
6.13	0.51465730545334\\
6.14	0.514618538257061\\
6.15	0.514579912674874\\
6.16	0.514541428086736\\
6.17	0.514503083875478\\
6.18	0.514464879426807\\
6.19	0.514426814129281\\
6.2	0.514388887374308\\
6.21	0.514351098556131\\
6.22	0.514313447071815\\
6.23	0.514275932321239\\
6.24	0.514238553707082\\
6.25	0.514201310634815\\
6.26	0.514164202512686\\
6.27	0.514127228751711\\
6.28	0.514090388765661\\
6.29	0.514053681971053\\
6.3	0.514017107787139\\
6.31	0.51398066563589\\
6.32	0.513944354941991\\
6.33	0.513908175132824\\
6.34	0.513872125638463\\
6.35	0.513836205891656\\
6.36	0.513800415327819\\
6.37	0.513764753385023\\
6.38	0.513729219503981\\
6.39	0.51369381312804\\
6.4	0.513658533703168\\
6.41	0.513623380677942\\
6.42	0.513588353503539\\
6.43	0.513553451633725\\
6.44	0.513518674524839\\
6.45	0.513484021635788\\
6.46	0.513449492428034\\
6.47	0.513415086365581\\
6.48	0.513380802914965\\
6.49	0.513346641545244\\
6.5	0.513312601727986\\
6.51	0.513278682937258\\
6.52	0.513244884649616\\
6.53	0.513211206344091\\
6.54	0.513177647502183\\
6.55	0.513144207607844\\
6.56	0.513110886147475\\
6.57	0.513077682609905\\
6.58	0.51304459648639\\
6.59	0.513011627270596\\
6.6	0.512978774458589\\
6.61	0.512946037548827\\
6.62	0.512913416042147\\
6.63	0.512880909441754\\
6.64	0.512848517253211\\
6.65	0.512816238984429\\
6.66	0.512784074145654\\
6.67	0.512752022249459\\
6.68	0.512720082810733\\
6.69	0.512688255346669\\
6.7	0.512656539376753\\
6.71	0.512624934422757\\
6.72	0.512593440008725\\
6.73	0.512562055660962\\
6.74	0.512530780908028\\
6.75	0.512499615280722\\
6.76	0.512468558312078\\
6.77	0.512437609537348\\
6.78	0.512406768493997\\
6.79	0.512376034721687\\
6.8	0.512345407762274\\
6.81	0.512314887159791\\
6.82	0.512284472460444\\
6.83	0.512254163212595\\
6.84	0.512223958966759\\
6.85	0.512193859275587\\
6.86	0.512163863693861\\
6.87	0.512133971778482\\
6.88	0.512104183088461\\
6.89	0.512074497184906\\
6.9	0.512044913631018\\
6.91	0.512015431992074\\
6.92	0.511986051835421\\
6.93	0.511956772730468\\
6.94	0.511927594248672\\
6.95	0.51189851596353\\
6.96	0.51186953745057\\
6.97	0.511840658287341\\
6.98	0.511811878053402\\
6.99	0.511783196330315\\
7	0.511754612701632\\
7.01	0.511726126752888\\
7.02	0.511697738071591\\
7.03	0.511669446247211\\
7.04	0.511641250871174\\
7.05	0.511613151536848\\
7.06	0.511585147839537\\
7.07	0.511557239376471\\
7.08	0.511529425746796\\
7.09	0.511501706551563\\
7.1	0.511474081393723\\
7.11	0.511446549878116\\
7.12	0.511419111611459\\
7.13	0.511391766202342\\
7.14	0.511364513261212\\
7.15	0.511337352400373\\
7.16	0.511310283233968\\
7.17	0.511283305377977\\
7.18	0.511256418450202\\
7.19	0.511229622070264\\
7.2	0.51120291585959\\
7.21	0.511176299441405\\
7.22	0.511149772440723\\
7.23	0.511123334484342\\
7.24	0.511096985200827\\
7.25	0.511070724220512\\
7.26	0.51104455117548\\
7.27	0.511018465699565\\
7.28	0.510992467428336\\
7.29	0.51096655599909\\
7.3	0.510940731050846\\
7.31	0.510914992224335\\
7.32	0.510889339161991\\
7.33	0.510863771507941\\
7.34	0.510838288908002\\
7.35	0.510812891009668\\
7.36	0.510787577462101\\
7.37	0.510762347916128\\
7.38	0.510737202024227\\
7.39	0.510712139440522\\
7.4	0.510687159820775\\
7.41	0.510662262822374\\
7.42	0.510637448104333\\
7.43	0.510612715327273\\
7.44	0.510588064153425\\
7.45	0.510563494246613\\
7.46	0.510539005272251\\
7.47	0.510514596897335\\
7.48	0.510490268790433\\
7.49	0.510466020621678\\
7.5	0.510441852062762\\
7.51	0.510417762786925\\
7.52	0.510393752468949\\
7.53	0.510369820785151\\
7.54	0.510345967413375\\
7.55	0.510322192032982\\
7.56	0.510298494324846\\
7.57	0.510274873971344\\
7.58	0.510251330656349\\
7.59	0.510227864065224\\
7.6	0.510204473884812\\
7.61	0.51018115980343\\
7.62	0.510157921510862\\
7.63	0.510134758698351\\
7.64	0.510111671058592\\
7.65	0.510088658285723\\
7.66	0.510065720075321\\
7.67	0.510042856124393\\
7.68	0.510020066131368\\
7.69	0.509997349796091\\
7.7	0.509974706819816\\
7.71	0.509952136905198\\
7.72	0.509929639756287\\
7.73	0.509907215078519\\
7.74	0.509884862578713\\
7.75	0.509862581965059\\
7.76	0.509840372947116\\
7.77	0.509818235235801\\
7.78	0.509796168543384\\
7.79	0.509774172583483\\
7.8	0.509752247071052\\
7.81	0.509730391722381\\
7.82	0.509708606255084\\
7.83	0.509686890388094\\
7.84	0.509665243841657\\
7.85	0.509643666337325\\
7.86	0.50962215759795\\
7.87	0.509600717347675\\
7.88	0.509579345311931\\
7.89	0.509558041217428\\
7.9	0.509536804792148\\
7.91	0.509515635765343\\
7.92	0.509494533867523\\
7.93	0.509473498830453\\
7.94	0.509452530387145\\
7.95	0.509431628271853\\
7.96	0.509410792220067\\
7.97	0.509390021968506\\
7.98	0.50936931725511\\
7.99	0.509348677819037\\
8	0.509328103400655\\
8.01	0.509307593741537\\
8.02	0.509287148584455\\
8.03	0.509266767673369\\
8.04	0.509246450753431\\
8.05	0.50922619757097\\
8.06	0.509206007873489\\
8.07	0.509185881409661\\
8.08	0.509165817929319\\
8.09	0.509145817183456\\
8.1	0.509125878924212\\
8.11	0.509106002904875\\
8.12	0.50908618887987\\
8.13	0.509066436604756\\
8.14	0.50904674583622\\
8.15	0.509027116332071\\
8.16	0.509007547851233\\
8.17	0.508988040153742\\
8.18	0.508968593000739\\
8.19	0.508949206154463\\
8.2	0.508929879378248\\
8.21	0.508910612436517\\
8.22	0.508891405094775\\
8.23	0.508872257119604\\
8.24	0.508853168278659\\
8.25	0.50883413834066\\
8.26	0.50881516707539\\
8.27	0.508796254253687\\
8.28	0.508777399647438\\
8.29	0.508758603029578\\
8.3	0.50873986417408\\
8.31	0.508721182855951\\
8.32	0.508702558851231\\
8.33	0.508683991936979\\
8.34	0.508665481891277\\
8.35	0.50864702849322\\
8.36	0.508628631522911\\
8.37	0.508610290761458\\
8.38	0.508592005990965\\
8.39	0.508573776994533\\
8.4	0.50855560355625\\
8.41	0.508537485461187\\
8.42	0.508519422495395\\
8.43	0.508501414445898\\
8.44	0.508483461100688\\
8.45	0.508465562248723\\
8.46	0.508447717679918\\
8.47	0.508429927185144\\
8.48	0.508412190556221\\
8.49	0.508394507585912\\
8.5	0.508376878067923\\
8.51	0.508359301796891\\
8.52	0.508341778568388\\
8.53	0.508324308178909\\
8.54	0.508306890425871\\
8.55	0.508289525107607\\
8.56	0.508272212023362\\
8.57	0.50825495097329\\
8.58	0.508237741758445\\
8.59	0.508220584180781\\
8.6	0.508203478043145\\
8.61	0.508186423149273\\
8.62	0.508169419303788\\
8.63	0.508152466312189\\
8.64	0.508135563980856\\
8.65	0.508118712117035\\
8.66	0.508101910528845\\
8.67	0.508085159025262\\
8.68	0.508068457416125\\
8.69	0.508051805512125\\
8.7	0.508035203124803\\
8.71	0.508018650066546\\
8.72	0.508002146150581\\
8.73	0.507985691190974\\
8.74	0.507969285002624\\
8.75	0.507952927401257\\
8.76	0.507936618203424\\
8.77	0.507920357226499\\
8.78	0.507904144288668\\
8.79	0.507887979208934\\
8.8	0.507871861807105\\
8.81	0.507855791903794\\
8.82	0.507839769320415\\
8.83	0.507823793879177\\
8.84	0.507807865403083\\
8.85	0.507791983715921\\
8.86	0.507776148642267\\
8.87	0.507760360007476\\
8.88	0.507744617637678\\
8.89	0.507728921359777\\
8.9	0.507713271001447\\
8.91	0.507697666391124\\
8.92	0.507682107358008\\
8.93	0.507666593732054\\
8.94	0.507651125343973\\
8.95	0.507635702025223\\
8.96	0.507620323608011\\
8.97	0.507604989925284\\
8.98	0.50758970081073\\
8.99	0.507574456098768\\
9	0.507559255624553\\
9.01	0.507544099223966\\
9.02	0.50752898673361\\
9.03	0.507513917990812\\
9.04	0.507498892833613\\
9.05	0.507483911100769\\
9.06	0.507468972631745\\
9.07	0.507454077266714\\
9.08	0.507439224846549\\
9.09	0.507424415212825\\
9.1	0.507409648207811\\
9.11	0.507394923674469\\
9.12	0.50738024145645\\
9.13	0.507365601398091\\
9.14	0.507351003344411\\
9.15	0.507336447141107\\
9.16	0.507321932634553\\
9.17	0.507307459671792\\
9.18	0.50729302810054\\
9.19	0.507278637769176\\
9.2	0.507264288526741\\
9.21	0.507249980222935\\
9.22	0.507235712708116\\
9.23	0.50722148583329\\
9.24	0.507207299450117\\
9.25	0.5071931534109\\
9.26	0.507179047568586\\
9.27	0.507164981776761\\
9.28	0.507150955889648\\
9.29	0.507136969762103\\
9.3	0.507123023249612\\
9.31	0.507109116208289\\
9.32	0.507095248494873\\
9.33	0.507081419966721\\
9.34	0.507067630481811\\
9.35	0.507053879898733\\
9.36	0.507040168076693\\
9.37	0.507026494875501\\
9.38	0.507012860155577\\
9.39	0.506999263777942\\
9.4	0.506985705604217\\
9.41	0.50697218549662\\
9.42	0.506958703317965\\
9.43	0.506945258931655\\
9.44	0.506931852201681\\
9.45	0.506918482992622\\
9.46	0.506905151169638\\
9.47	0.506891856598468\\
9.48	0.50687859914543\\
9.49	0.506865378677413\\
9.5	0.506852195061881\\
9.51	0.506839048166865\\
9.52	0.506825937860959\\
9.53	0.506812864013324\\
9.54	0.50679982649368\\
9.55	0.506786825172304\\
9.56	0.506773859920027\\
9.57	0.506760930608233\\
9.58	0.506748037108856\\
9.59	0.506735179294376\\
9.6	0.506722357037817\\
9.61	0.506709570212745\\
9.62	0.506696818693263\\
9.63	0.506684102354014\\
9.64	0.50667142107017\\
9.65	0.506658774717437\\
9.66	0.506646163172048\\
9.67	0.506633586310764\\
9.68	0.506621044010866\\
9.69	0.506608536150158\\
9.7	0.506596062606962\\
9.71	0.506583623260115\\
9.72	0.506571217988969\\
9.73	0.506558846673383\\
9.74	0.506546509193729\\
9.75	0.506534205430883\\
9.76	0.506521935266222\\
9.77	0.506509698581627\\
9.78	0.506497495259477\\
9.79	0.506485325182646\\
9.8	0.506473188234503\\
9.81	0.506461084298908\\
9.82	0.506449013260209\\
9.83	0.506436975003243\\
9.84	0.506424969413329\\
9.85	0.506412996376268\\
9.86	0.506401055778344\\
9.87	0.506389147506314\\
9.88	0.506377271447413\\
9.89	0.506365427489347\\
9.9	0.506353615520294\\
9.91	0.5063418354289\\
9.92	0.506330087104275\\
9.93	0.506318370435997\\
9.94	0.506306685314101\\
9.95	0.506295031629084\\
9.96	0.506283409271901\\
9.97	0.506271818133958\\
9.98	0.506260258107119\\
9.99	0.506248729083695\\
10	0.506237230956447\\
};
\end{axis}
\end{tikzpicture}%
  \caption{The posterior probability that the slope of the regression
    line is positive as a function of the noise standard deviation
    $\sigma$.  Note the limits on the $y$-axis.}
  \label{problem_4}
\end{figure}

\end{document}
